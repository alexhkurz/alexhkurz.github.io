\documentclass{entcs}
%
\usepackage[PostScript=dvips]{diagrams}
\usepackage{entcsmacro}
\usepackage{hyperref}
%
\usepackage[T1]{fontenc}
\usepackage[latin1]{inputenc}
%\usepackage[english]{babel}
%
%\usepackage{a4wide}
\usepackage{latexsym}
\usepackage{amsmath}
\usepackage{amssymb}
\usepackage{amsfonts}
\usepackage{stmaryrd}
%\usepackage{amsthm}
\usepackage{amsmath}
%\usepackage{multicol}
%\usepackage{mathrsfs}
%\usepackage{graphics}
\usepackage[matrix,arrow,curve]{xy}
\usepackage{url} 
\newcommand{\hypergetpageref}{0}
%
\input{include/meineamstheorems.tex}
\input{include/meinemakros.tex}
\input{include/makros-alex.tex}
%
%\setlength{\parindent}{0pt}
%\setlength{\parskip}{0pt}
\bibliographystyle{plain}

\def\lastname{Kurz and Pattinson}
\begin{document}
\begin{frontmatter}
  \title{Definability, Canonical Models, Compactness for
    Finitary Coalgebraic Modal Logic}%
  \author{Alexander Kurz\thanksref{myemail}}%
  \address{CWI\\ P.O.  Box 94079, 1090 GB Amsterdam\\ The Netherlands}%
  \author{Dirk Pattinson\thanksref{coemail}}%
  \address{Ludwig-Maximilians-Universit{\"a}t M{\"u}nchen\\
    Oettingenstr.~67, 80538 M{\"u}nchen\\ Germany}%
  \thanks[myemail]{Email: \href{mailto:kurz@cwi.nl}
    {\texttt{\normalshape kurz@cwi.nl}}}%
  \thanks[coemail]{Email:
    \href{mailto:pattinso@informatik.uni-muenchen.de}
    {\texttt{\normalshape pattinso@informatik.uni-muenchen.de}}}%
  \begin{abstract}
    This paper studies coalgebras %for a $\set$-endofunctor $T$ 
    from the perspective of the \emph{finitary} observations that can
    be made of their behaviours.
    %the behaviours of $T$-coalgebras.
    Based on the %finitary part of the 
    terminal sequence, notions of
    finitary behaviours and finitary predicates are introduced. A
    category $\Log(T)$ of coalgebras with morphisms preserving
    finitary behaviours is defined.
    
    We then investigate definability and compactness for finitary
    coalgebraic modal logic, show that the final object in $\Log(T)$
    generalises the notion of a canonical model in modal logic, and
    study the topology induced on a coalgebra by the finitary part of
    the terminal sequence.
  \end{abstract}
\end{frontmatter}


%DAS LETZTE THEOREM NICHT NUR FUER CANTORSPACE?????

%\setcounter{section}{-1}
\section*{Introduction}

Coalgebras for an endofunctor $\FT$ on $\Set$ encompass many types of
state base systems, including Kripke models and frames, labelled
transition systems, Moore- and Mealy automata and deterministic
systems, see eg.\ Rutten~\cite{rutten:uc-j}. The research on modal
logics as specification languages for coalgebras began with
Moss~\cite{moss:cl} and was then taken up in eg.\ 
\cite{kurz:cmcs98-j,roessiger:cmcs98-j,roessiger:cmcs00,jacobs:tl,jacobs:many-sorted}.

The relationship between modal logic and coalgebras
% %, ie.\ how
% %coalgebras give semantics to modal formulas, 
has been explained in \cite{kurz:aiml98} as follows. Denoting by $Z$
the carrier of the final coalgebra, we can consider as the semantics
of a modal formula $\phi$ the subset $\sem{\phi}\subseteq Z$
%of the carrier $Z$ of the final coalgebra 
satisfying $\phi$.
% %
% The relationship between modal logic and coalgebras
% %, ie.\ how coalgebras give semantics to modal formulas, has been explained in
% % %\cite{kurz:aiml98} as follows. 
% can be explained in different ways. First, following
% \cite{kurz:aiml98}, Denoting by $Z$ the carrier of the final
% coalgebra, we can consider as the semantics of a modal formula $\phi$
% the subset $\sem{\phi}\subseteq Z$
% %of the carrier $Z$ of the final coalgebra 
% satisfying $\phi$.
%
The intuition here is: The elements of $Z$ are the
%(complete infinite) 
behaviours %of the systems under consideration 
and the meaning of a modal formula $\phi$ is the property (ie.\ set)
of behaviours defined by $\phi$.
%
In case that the logics we are interested in are expressive in the
sense that they allow to define \emph{all} subsets of $Z$, we can
identify modal formulas and subsets of $Z$, resulting in a new way to
algebraically investigate modal logics,
see~\cite{kurz:aiml98,kurz:cmcs01}.

Unfortunately, modal logics given by a finitary syntax are in general
not expressive. 
% This is a consequence of the fact 
The reason for this is simply that not all properties of behaviours
can be described in a finite language. One of the main topics of this
paper is the quest for a semantics of modal logic that suits
\emph{finitary} modal logic as perfectly as the semantics sketched
above does for expressive modal logics.

% A second approach to a coalgebraic semantics of modal logics, due to
% \cite{pattinson:sem-mod,pattinson:cml-report}, relies on the so called
% \emph{terminal sequence} $(\FT^n 1)$ of the endofunctor $\FT$. The
% terminal sequence $(\FT^n 1)$ can be understood as approximating the
% final coalgebra. Intuively, the elements of the $n$-th approximant
% represent the behaviours that can be observed of a system in $n$
% steps. Accordingly, the semantics of a modal formula $\phi$ of depth
% $n$ is given by a subset $\sem{\phi}\subseteq\FT^n 1$.


The key concept used to account for finitariness is the so called
\emph{terminal sequence} $(\FT^n 1)$ of the endofunctor $\FT$. The
terminal sequence $(\FT^n 1)$ can be understood as approximating the
final coalgebra. Intuively, the elements of the $n$-th approximant
represent the behaviours that can be observed of a system in $n$
steps. Following \cite{pattinson:sem-mod,pattinson:cml-report}, the
semantics of a finitary modal formula $\phi$ of depth $n$ will be a
subset $\sem{\phi}\subseteq\FT^n 1$.

In case that the functor $\FT$ maps finite sets to finite sets, the
approach above results in a perfect match between finitary modal
logics and their semantics: For all finite ordinals $n$ and all subset
of $\FT^n 1$ there will be a formula defining this subset. In case
that $\FT$ maps finite sets to infinite sets, which is for example the
case for logics having infinitely many atomic propositions, there will
be subsets of $\FT^n 1$ which are not defined by a single finitary
formula.  This is the reason to introduce topologies $\tau_n$ on the
approximants $\FT^n 1$ in Section~\ref{section:topo-logical}, the idea
being that finitary formulas correspond to clopen subsets.
% Furthermore, in
% Section~\ref{section:compactness} we use topological compacteness to
% investigate compactness of modal logics for coalgebras.cdcd


The main novelty of the paper is probably the introduction, in
Section~\ref{section:finitary-predicates}, of the category $\Log(T)$
that has coalgebras as objects and functions that preserve
\emph{finitary} behaviours as morphisms. One of the claims of this
paper is that $\Log(T)$ plays a role for finitary modal logics as
$\Coalg{T}$ does for expressive modal logics. In
Section~\ref{section:canonical}, we show that $\Log(T)$ always has a
final object. Moreover, the final object in $\Log(T)$ is compared to
the canonical model as known from modal logic.  Assuming that $T$ maps
finite sets to finites sets and based on
Sections~\ref{section:canonical} and \ref{section:topo-logical},
Section~\ref{section:compactness} characterises those functors
$T$ for which finitary modal logics for $T$-coalgebras are compact.






\section{Preliminaries and Notation}\label{section:preliminaries}


We consider coalgebras for a $\Set$-endofunctor $\FT$. The category of
$\FT$-coalgebras and coalgebra morphisms is denoted by $\CoAlg(\FT)$.
%
The \emph{final $\Sigfun$-coalgebra}, denoted by $\Zsf=(Z,\zeta)$, is
-- if it exists -- defined up to isomorphism by the property that for
all $\Asf\in\CoalgFun$ there is a unique morphism
$\ineins_\Asf:\Asf\maps\Zsf$.  Given an element $a$ in $\Asf$, we call
$(\Asf,a)$ a \emph{process} and $\ineins_\Asf(a)$ its
\emph{behaviour}. Two processes $(\Asf,a)$ and $(\Bsf,b)$ are
\emph{behaviourally equivalent} (written $(\Asf,a)\sim(\Bsf,b)$), if
they can be identified by a morphism of coalgebras, ie.\ if there
exists $(\Csf, c) \in \CoalgFun$ and $f: \Asf \to \Csf$, $g:\Bsf \to
\Csf \in \CoalgFun$ such that $f(a) = g(b)$.

If the final coalgebra exists, this is clearly equivalent to 
$\ineins_\Asf(a)=\ineins_\Bsf(b)$.

\begin{example}[Streams]
  For a set $\Elem$ consider $\Sigfun X=\Elem\times X$. Given a coalgebra
  $\alpha=\langle\head,\tail\rangle: A\maps\Elem\times A$ the (complete)
  behaviour of an element $a\in A$ is the infinite list
  $(\head(a),\head(\tail(a)),\head(\tail(\tail(a))),\ldots)$. Accordingly, the
  final coalgebra $\Zsf=(\Elem^\nat,\langle\head,\tail\rangle)$ is given by the
  infinite lists over $\Elem$.
\end{example}

\begin{example}[Kripke models]
  Let $\Prop$ be a countably infinite set. Coalgebras for the functor
  $\Sigfun(X)=\Pow X\times \Pow\Prop$ are Kripke models. The coalgebraic
  notion of behavioural equivalence coincides with the standard notion of
  bisimulation in modal logic.
\end{example}

% \noindent
% Sometimes it is easy to give an explicit description of the final coalgebra:

We have seen how the final coalgebra (if it exists) provides a notion
of behaviour. In general, however, the behaviour of a process
represents an infinite amount of information.  This paper investigates
properties of processes, which can be specified by a finite amount of
information. Hence the final coalgebra (containing the infinite
behaviours of all processes) has to be replaced by finitary
approximations. These approximations are provided by the (finitary
part) of the so-called terminal sequence of the underlying endofunctor
$\FT$.

\subsection{The Terminal Sequence}

Terminal sequences can be thought of as approximating the final
coalgebra.  The following definition has been taken from
\cite{worrell:cmcs99}.

The \emph{terminal sequence} of $\FT$ is an ordinal indexed sequence of sets
$(Z_{\alpha})$ together with a family $(p^{\alpha}_{\beta})_{\beta \leq
\alpha}$ of functions $p^{\alpha}_{\beta}: Z_{\alpha} \to Z_{\beta}$ for all
ordinals $\beta \leq \alpha$ such that
\begin{itemize}
\item $Z_{\alpha+1} = \FT Z_{\alpha}$ and $p^{\alpha+1}_{\beta+1} = \FT
p^{\alpha}_{\beta}$ for all $\beta \leq \alpha$
\item
  $p^{\alpha}_{\alpha} = \id_{Z_{\alpha}}$ and $p^{\alpha}_{\gamma} =
        p^{\beta}_{\gamma} \circ p^{\alpha}_{\beta}$ for $\gamma \leq \beta \leq
        \alpha$
\item
  The cone $(Z_{\alpha}, (p^{\alpha}_{\beta}))_{\beta < \alpha}$ is limiting
        whenever $\alpha$ is a limit ordinal.
\end{itemize}

Thinking of $Z_{\alpha}$ as the $\alpha$-fold application of $\FT$ to
the limit $1$ of the empty diagram, we write $Z_{\alpha} =
\FT^{\alpha} 1$ in the sequel.  Intuitively, $\FT^n 1$ represents
behaviours which can be exhibited in $n$ steps.  For example, if $\FT X
= D \times X$, then $\FT^n 1 \cong D^n$ contains all lists of length
$n$.

Note that every coalgebra $(C, \gamma)$ gives rist to a cone $(C,
(\gamma_{\alpha}: C \to \FT^{\alpha} 1))$ over the terminal
sequence:

%$
\begin{defn}\label{def:n-step-beh}
  Given $(C, \gamma) \in \CoalgFun$, let $\gamma_0 = \ineins_C: C \to 1$
  denote the unique mapping. For successor ordinals $\alpha = \beta +
  1$ let $\gamma_{\alpha}: C \to \FT^{\alpha}1 = \FT\gamma_{\beta}
  \circ \gamma$.  If $\alpha$ is a limit ordinal, let
  $\gamma_{\alpha}$ be the unique map for which $\gamma_{\beta} =
  p^{\alpha}_{\beta} \circ \gamma_{\alpha}$ for all $\beta < \alpha$.
\end{defn}
We will often use without further mentioning the following easy
\begin{prop}\label{prop:easy} Let $n$ be an ordinal.
\begin{enumerate}
\item Let $f:(A,\alpha)\to(B,\beta)$ be a coalgebra morphism. Then
  $\beta_n\circ f=\alpha_n$.
\item Let $(A,\alpha)\in\CoalgFun$. Then
  $p^{n+1}_n\circ\FT(\alpha_n)\circ\alpha=\alpha_n$.
\end{enumerate}
\end{prop}


\subsection{Coalgebraic Modal Logic}

Following \cite{pattinson:sem-mod,pattinson:cml-report},
we consider modal logics for coalgebras as given by predicate liftings.

\begin{defn}
A \emph{predicate lifting} for $\FT$ is a natural transformation $\lambda: 2 \to
2 \circ \FT$, where $2$ denotes the contravariant powerset functor.
\end{defn}

\begin{example} \label{example:kripke-liftings}
Consider $\FT X = \Pow(X) \times \Pow(A)$, where $A$ is a set of atomic
propositions. Then $\FT$-coalgebras are in 1-1 correspondence with Kripke models
over the set $A$ of atoms. We demonstrate how to capture the interpretation of
atoms and modalities using predicate liftings. Let $a \in A$ and consider the
liftings $\lambda$ and $\lambda_a$ given by
\begin{align*}
\lambda(X)(\ssx) & = \lbrace (\ssx', \ssa) \in \Pow(X) \times \Pow(A)
  \sbars \ssx' \subseteq \ssx \rbrace \\
\lambda_a(X)(\ssx) & = \lbrace (\ssx', \ssa) \in \Pow(X) \times \Pow(A)
  \sbars a \in \ssa \rbrace.
\end{align*}
%
Given $(C, \gamma) \in \CoAlg(\FT)$, we write $c \to c'$ if $c, c' \in C$ and
$c' \in \pi_1 \circ \gamma(c)$. Also, if $c \in C$ and $a \in A$, we write
$c \models a$ iff $a \in \pi_2 \circ \gamma(c)$. 

\sskip For the case of modalities, consider a subset $\ssc \subseteq C$, which
we think of as the interpretation $\ssc = \lsem \phi \rsem$ of a modal formula
$\phi$.  Then
\[
  \gamma\inv \circ \lambda(C)(\ssc)  = 
        \lbrace c \in C \sbars \forall c' \in C. c \to c' \implies c' \in \ssc \rbrace
\]
corresponding to the interpretation of $\Box \phi$. The same formula with
$\lambda$ replaced by $\lambda_a$ yields
\[
  \gamma\inv \circ \lambda_a(C)(\ssc) = 
        \lbrace c \in C \sbars c \models a \rbrace.
\]
Hence, given any $\ssc \subseteq C$, we can capture the set of worlds which
satisfy $a$ using the lifting $\lambda_a$.\qed
\end{example}

\begin{defn}[Syntax and semantics of $\Lang(\FT,\PredLift)$]
  Given a set $\PredLift$ of predicate liftings for $\FT$, we consider the
  language $\Lang(\FT,\PredLift)$, often abbreviated to $\Lang(\PredLift)$,
  which is given by the grammar
\[
  \phi ::= \false \sbars \phi \to \psi \sbars \um{\lambda} \phi 
        \qquad (\lambda \in \PredLift)
\]
For a structure $(C, \gamma) \in \CoAlg(\FT)$, the semantics $\lsem \phi
\rsem_{\gamma} \subseteq C$ is given by
\[
\lsem \false \rsem_{\gamma} = \emptyset %
\qquad \lsem \phi \to \psi \rsem_{\gamma} = (C \setminus \lsem \phi \rsem_{\gamma}) \cup \lsem \psi \rsem_{\gamma}%
\qquad \lsem \um{\lambda} \phi\rsem_{\gamma} = \gamma\inv \circ
\lambda(C)(\lsem \phi \rsem_{\gamma})
\]
\end{defn}

\begin{notn}[The modal logic $\ml$]
  In case of $\FT=\Pow\times\Pow\Prop$, $\Prop$ countably infinite, we denote
  $\Lang(\FT,\PredLift)$ with $\PredLift$ as in the example above by $\ml$.
  $\ml$ is the standard modal logic for Kripke models.
\end{notn}

% Given the two basic predicates $\true = \lbrace 0 \rbrace$ and
% $\false = \emptyset$ on $1 = \lbrace 0 \rbrace$, we use predicate liftings
% (and boolean operations) to obtain predicates on the sets $\FT^n 1$ for any $n
% \in \Nat$. 
% This provides the handle to study the Cantor
% space topology from a logical viewpoint.

In the remainder of this section we show that each formula $\phi \in
\Lang(\PredLift)$ gives rise to a predicate $t \subseteq \FT^n 1$ for
some $n \in \Nat$. Given any structure $(C, \gamma) \in \CoAlg(\FT)$,
we show that $\lsem \phi \rsem_\gamma = \gamma_n\inv(t)$, where
$\gamma_n$ is defined as in Definition~\ref{def:n-step-beh}.

\begin{defn}[Formulas of depth $n$]\label{def:depth}
Let $L_0$ denote the set of formulas of propositional logic, that is, $L_0$ is
given by the grammar 
\[
  L_0 \ni \phi, \psi  ::= \false \sbars \phi \to \psi
\]
The mapping $d_0: L_0 \to \Pow(1)$ is given by $d_0(\phi) = 1$ iff $\phi$ is a
tautology, and $d_0(\phi) = \emptyset$, otherwise.
%
For $n \geq 0$ the set $L_{n+1}$ is given by the grammar
\[
  L_{n+1} \ni \phi, \psi ::= \false \sbars \phi \to \psi \sbars \um{\lambda} \rho
        \qquad (\lambda \in \PredLift, \rho \in L_n)
\]
The mapping $d_{n+1}: L_{n+1} \to \Pow(\FT^{n+1} 1)$ is given inductively by
\[
  \false \mapsto \emptyset \qquad
        \phi \to \psi \mapsto (\FT^n 1 \setminus d_{n+1}(\phi)) \cup d_{n+1} (\psi)
        \qquad
        \um{\lambda} \rho \mapsto \lambda(\FT^n 1)(d_n(\rho))
\]
We call formulas in $L_n$ \emph{formulas of depth $n$}. For $\phi\in L_n$, we
also write $\sem{\phi}_n$ or simply $\sem{\phi}$ instead of $d_n(\phi)$.
\end{defn}

\begin{example}\label{exle:depth-ml}
  In case of $\ml$, an equivalent definition of the depth of a formula
  is the following: $\depth(\false)=0$,
  $\depth(\phi\imp\psi)=\max\{\depth(\phi),depth(\psi)\}$,
  $\depth(p)=1$ for $p\in\Prop$, $\depth(\Box\phi)=\depth(\phi)+1$.
  This is a slight variation of the standard definition in modal logic
  (where $\depth(p)=0$) but can be used to the same effect.
\end{example}

Intuitively, a formula $\phi \in L_n$ describes behaviour which take $n$
transition steps into account. When thinking of the sets $\FT^n 1$ as
representing the behaviour which can be exhibited in $n$ transition steps, the
value $d_n(\phi)$ is a model-independent interpretation of $\phi$. We note that
every formula is eventually contained in one of the $L_n$'s:

\begin{propn} \label{propn:all-of-L}
$\Lang(\PredLift) = \bigcup_{n \in \Nat} L_n$.
\end{propn}
% \begin{proof}
% One shows $L_n \subseteq \Lang(\PredLift)$ by induction on $n$. The inclusion
% $\Lang(\PredLift) \subseteq \bigcup_{n \in \Nat} L_n$ is established using
% structural induction.
% \end{proof}

The following proposition supports the intuition that $d_n(\phi)$ is a
semantical model-independent representation of the $\phi$, for $\phi \in L_n$.
% It also establishes the link between coalgebraic modal logic and the Cantor
% space topology on coalgebras.

\begin{propn} \label{propn:semantics}
Suppose $(C, \gamma) \in \CoAlg(\FT)$. Then
\[
\lsem \phi \rsem_\gamma = \gamma_n\inv \circ d_n (\phi)
\]
for all $n \in \Nat$ and all $\phi \in L_n$.
\end{propn}
\begin{proof}
By induction on $n$ using the naturality of predicate liftings.
\end{proof}

% To be able to express topological properties logically, we have to see to the
% fact that each $d_n$ is a surjection, that is,
In order to obtain definability results, we have to assume that the logic
under consideration is reasonably expressive. We deal with two notions of
expressiveness: logics, which allow the denotation of every $t \subseteq \FT^n
1$ by a formula, and those which allow to carve out every predicate $t
\subseteq \FT^n 1$ by a \emph{set} of formulas. The formal definition is as
follows:

\begin{defn}[formula-(set-)expressive] \label{defn:expressive}
The language $\Lang(\FT,\PredLift)$ is called
\begin{itemize}
\item \emph{formula-expressive}, if every $d_n$ is a surjection
\item \emph{formula-set-expressive}, if every $D_n$ is a surjection
\end{itemize}
where  $d_n$ is as in Definition~\ref{def:depth} and $D_n$ is defined by
\[
  D_n:\Pow L_n\to \Pow\FT^n 1, \quad\Phi\mapsto
  \bigcap\{d_n(\phi):\phi\in\Phi\}.
\]
\end{defn}
%
For $\Lang(\FT,\PredLift)$ to be expressive, we have to put a completeness
condition on the set of of predicate liftings.

\begin{defn}[Separation]
\begin{enumerate}
\item
  Suppose $C$ is a set and $\mathcal C \subseteq \Pow(C)$ is a system of
  subsets of $C$. We call $\mathcal C$ \emph{separating}, if the map
  $s: C \to \Pow(\mathcal C)$, $s(c) = \lbrace \ssc \in \mathcal C \sbars
  c \in \ssc \rbrace$ is monic.
\item
  A set $\PredLift$ of predicate liftings for $C$ is called \emph{separating},
  if $\lbrace \lambda(C)(\ssc) \sbars \lambda \in \PredLift,
  \ssc \subseteq C \rbrace$ is a separating set of subsets of $\FT C$, for all
  sets $C$.
\end{enumerate}
\end{defn}

The idea of separation is that the individual points of $C$ can be
distinguished by the predicates $P \in \mathcal C$. Passing to predicate
liftings, we can distinguish individual successors $x \in \FT X$ by means of
predicate liftings, assuming that all points of $X$ can be distinguished.

The separation property is present in many examples, notably also in Example~\ref{example:kripke-liftings}.

\begin{example}
Let $\FT X = \Pow(X) \times \Pow(A)$ for some set $A$ of atomic propositions.
Consider $\PredLift = \lbrace \lambda  \rbrace \cup \lbrace \lambda_a \sbars a
\in A \rbrace$ as defined in Example~\ref{example:kripke-liftings}. Then
$\PredLift$ is separating.
\end{example}

Assuming that $\FT$ maps finite sets to finite sets, which corresponds
to the fact of having only finitely many propositional variables in
the case of Kripke models, one easily establishes

\begin{propn} \label{propn:surjection}
  Suppose $\FT$ maps finite sets to finite sets and $\PredLift$ is separating.
        Then $\Lang(\FT,\PredLift)$ is formula-expressive.
\end{propn}
% \begin{proof}
% Easy induction.
% \end{proof}

\noindent
Given a structure $(C, \gamma) \in \CoAlg(\FT)$, the above proposition
says that every predicate $\ssc \subseteq C$ on $C$, which arises as
$\gamma_n\inv(t)$ for some $t \subseteq \FT^n 1$ can actually be
denoted by a formula.

In the case of Kripke models with a countably infinite set of propositional
variables we have

\begin{propn} \label{propn:surjection-weak}
  Let $\FT=\Pow\times\Pow\Prop$, $\Prop$ countably infinite, and
  $\Lang(\FT,\PredLift)=\ml$. 
  Then $\Lang(\FT,\PredLift)$ is formula-set-expressive.
\end{propn}
\noindent
Given a structure $(C, \gamma) \in \CoAlg(\Pow\times\Pow\Prop)$, the above
proposition says that every predicate $\ssc \subseteq C$ on $C$, which arises
as $\gamma_n\inv(t)$ for some $t \subseteq \FT^n 1$ can be denoted by
a set of formulas.








\section{Finitary Predicates and the Category $\Log(\FT)$}
\label{section:finitary-predicates}

Behavioural predicates, ie.\ predicates on coalgebras which are
invariant under observational equivalence, can be considered as
subsets of the carrier of the final coalgebra. Here, we are interested
in finitary behavioural predicates and we propose to consider them as
subsets of the $\FT^n 1$, $n<\omega$, given by the terminal sequence.


First, recalling Definition~\ref{def:n-step-beh}, we define a notion
of $n$-step behavioural equivalence and of predicates of depth $n$.

\begin{defn}[$n$-Behavioural equivalence]
  Let $n$ be an ordinal. For two coalgebras $\Asf=(A,\alpha)$,
  $\Bsf=(B,\beta)$ define $(\Asf,a)\sim_n (\Bsf,b)$ iff
  $\alpha_n(a)=\beta_n(a)$. Similarly, $\Asf\sim_n\Bsf$ iff
  $\alpha_n(A)=\beta_n(B)$ and $\Asf\sim_{<\omega} \Bsf$ iff
  $\alpha_n(A)=\beta_n(B)$ for all $n<\omega$.
\end{defn}

Under the assumoption that the final coalgebra exists, we consider two
points $x$ and $y$ as behaviourally equivalent, if they are identified
by the unique morphism into the final coalgebra.  As shown in
\cite{adamek-koubek:gfp}, this is equivalent to $\alpha_n(x) =
\alpha_n(y)$ for all ordinals $n$. The notion of finitary behavioural
predicates, which we are about to introduce, restricts the validity of
the above equation to \emph{finite} ordinals.


\begin{remark}
  While $(\Asf,a)\sim_\omega(\Bsf,b) \ssi \fa n<\omega \tp
  (\Asf,a)\sim_n(\Bsf,b) $, we only have $\Asf\sim_\omega\Bsf \oif \fa
  n < \omega \tp \Asf\sim_n \Bsf$. For an example refuting the converse, let
  $\FT X=\{a,b\}\times X$, $\Asf$ the final coalgebra with carrier
  $\{a,b\}^\omega$ and $\Bsf$ the subcoalgebra with carrier $\{s\cdot
  a^\omega:s\in\{a,b\}^*\}$. 
% Then $\fa n<\omega \tp \Asf\sim_\omega\Bsf$ but
%   not $\Asf\sim_\omega\Bsf$.
\end{remark}

\begin{example}
  For $\FT X = \Pow X\times\Pow\Prop$, $n$-behavioural equivalence is
  (a slight variation of) the bounded bisimulation of modal logic as
  studied in \cite{gerbrandy:diss}.
\end{example}

\begin{defn}[Behavioural predicates of depth $n$]\label{def:predicates-depth-n}
  A set $S\subseteq \Sigfun^n 1$ is called a behavioural predicate of
  depth $n$. A process $(A,\alpha,a)$ satisfies $S$, written
  $(A,\alpha,a)\models S$ and often abbreviated as $a\models S$, iff
  $\alpha_n(a)\in S$.
\end{defn}

\noindent 
We also use standard notation such as $(A,\alpha)\models S\ssi \fa a\in A\tp
a\models S$ and $\sem{S}_{(A,\alpha)}=\{a\in A:a\models S\} $ and
$\Mod(S)=\{\Asf\in\CoalgFun:\Asf\models S\}$.

\begin{example}
  Let $\phi\in\Lang(\FT,\PredLift)$ be a formula of depth $n$. Then
  the semantics of $\phi$ is determined by the predicate
  $\sem{\phi}\subseteq\FT^n 1$ (cf.\ Definition~\ref{def:depth}).  If
  $\FT$ maps finite sets to finite sets and $\PredLift$ is separating,
  every predicate $S\subseteq\FT^n 1$ is denoted by a formula (cf.\ 
  Proposition~\ref{propn:surjection}). In case of $\ml$ every
  predicate is denoted by a set of formulas (cf.\ 
  Proposition~\ref{propn:surjection-weak}).
\end{example}

\begin{remark}
  In \cite{kurz:aiml98,kurz:diss} it was proposed to investigate modal
  logics by considering subsets of the final coalgebra as the
  semantics of modal formulas. From this perspective, the approach
  presented here is a special case. Let $(Z,\zeta)$ be the final
  coalgebra. Then every $S\subseteq\FT^n 1$ is logically equivalent to
  $\zeta_n\inv(S)\subseteq Z$. Indeed, $(\Asf,a)\models S \ssi
  \alpha_n(a)\in S \ssi \zeta_n(\ineins_\Asf(a))\in S \ssi
  \ineins_\Asf(a)\in\zeta_n\inv(S)$ which was the definition of
  satisfaction for modal formulas as subsets of the final coalgebra.
  This will be used in the next section.
\end{remark}

Predicates of depth $n$ and $n$-behavioural equivalence are related by the
following propositions. 

% \begin{prop}
%   $(\Asf,a)\sim_n(\Bsf,b)\Ssi (\fa S\subseteq\FT^n 1\tp (\Asf,a)\models S \ssi
%   (\Bsf,b)\models S)$.
% \end{prop}

\begin{prop}
  Suppose $\Lang(\FT,\PredLift)$ is formula-expressive.
  Then, for all $\Asf$ and
  $a$ in $\Asf$, there exists $\phi_{(\Asf,a)}\in\Lang(\FT,\PredLift)$
  such that $(\Asf,a)\sim_n(\Bsf,b)\ssi (\Bsf,b)\models\phi_{(\Asf,a)}$.
\end{prop}

The following is a result well-known in modal logic (cf.\ 
\cite{gerbrandy:diss}, Proposition~2.8). 

\begin{prop}\label{prop:bounden-beh-equ-ml}
  Suppose $\Lang(\FT,\PredLift)$ is formula-set-expressive and
  consider $\Asf$, $\Bsf \in \CoalgFun$ and elements $a$ in $\Asf$,
  $b$ in $\Bsf$. Then $$(\Asf,a)\sim_n(\Bsf,b) \Ssi (a\models\phi\ssi
  b\models\phi \text{ for all $\phi\in \Lang(\FT,\PredLift)$ of depth
    $n$}).$$
\end{prop}

\mmskip $\CoalgFun$ can be considered as the category which has as
morphisms $f:(A,\alpha)\to (B,\beta)$ precisely those functions
$f:A\to B$ satisfying 
%
$$\ineins_{(B,\beta)}\circ f = \ineins_{(A,\alpha)}$$
%
where $\ineins$ denotes the morphisms into the final coalgebra.
Equivalently, a function $f:A\to B$ is a morphism $f:(A,\alpha)\to
(B,\beta)$ iff
%
$$\fa S\subseteq Z\tp f(a)\models S \ssi a\models S$$
%
where $Z$ is the carrier of the final coalgebra.
%where $(Z,\zeta)$ is the final coalgebra. 
% Equivalently, a function
% $f:A\to B$ is a morphism $f:(A,\alpha)\to (B,\beta)$ iff
% $$\ineins_{(B,\beta)}\circ f = \ineins_{(A,\alpha).}$$
% % \begin{diagram}%[small]
% % A && \rTo{f} && B\\
% % % & \rdTo{\ineins_{(A,\alpha)}} && \ldTo{\ineins{(B,\beta)}}&\\
% % % &&Z&&
% % \end{diagram}
% %
We now define the category $\Log(\FT)$ of coalgebras that has the
analogous properties for finitary behaviours and predicates.
% $S\subseteq\FT^n1$, $n<\omega$.

\begin{defn}[$\Log(\FT)$]\label{def:log}
  The category $\Log(\FT)$ has $\FT$-coalgebras as objects. Morphisms
  $f:(A,\alpha)\to (B,\beta)$ are those functions $f:A\to B$ such that,
  for all $n<\omega$,
%
$\beta_n\circ f = \alpha_n$.
%
\end{defn}

\begin{remark} 
  For $f:A\to B$ each of the following is equivalent to $f$ being a
  morphism $f:(A,\alpha)\to (B,\beta)$ in $\Log(\FT)$
\begin{gather*}
  \beta_\omega\circ f = \alpha_\omega\medskip\\
  \fa n<\omega\tp \fa S\subseteq \FT^n1\tp\fa a\in A \tp f(a)\models S
  \ssi a\models S
\end{gather*}
\end{remark}

\begin{remark}
  Some remarks relating the categories $\Log(\FT)$ and $\CoalgFun$ are
  in order. Clearly, every morphism of coalgebras $f: (A, \alpha) \to
  (B, \beta) \in \CoAlg(\FT)$ is also a morphism $f \in \Log(\FT)$. We
  hence obtain a functorial inclusion $\CoAlg(\FT) \to \Log(\FT)$. Now
  suppose $(A, \alpha)$ and $(B, \beta)$ are $\FT$-coalgebras and $(a,
  b) \in A \times B$. 
% If we take $a$ and $b$ to be observationally
%   equivalent iff they can be identified by a morphism of coalgebras
%   (ie. there exist morphisms $f, g \in \CoAlg(\FT)$ with the same
%   codomain such that $f(a) = g(b)$), then $(a, b)$ observationally
  Clearly, $a, b$ behaviourally equivalent implies $a \sim_{<\omega}
  b$. The converse holds if one assumes $\FT$ to be
  $\omega$-accessible:  Under this assumption, $\FT$ adimits a final
  coalgebra, which is a subset of $\FT^{\omega} 1$ (see
  \cite{worrell:cmcs99} for details). Hence, if $\FT$ is
  $\omega$-accessible then $\Log(\FT)=\Coalg{T}$.
\end{remark}

One of the claims of this paper is that studying finitary modal logics
for coalgebras, it may be a profitable approach to transfer techniques
and ideas known from $\CoalgFun$ to $\Log(\FT)$. For example, in
Section~\ref{section:canonical} we will show that $\Log(\FT)$
always has a final object.




\section{Definability Results}\label{section:definability}

For the purpose of definability, we assume the existence of a final coalgebra
$\Zsf = (Z, \zeta)$.
We
first note the following easy proposition relating predicates
over $\FT^n1$ to predicates over $Z$.

\begin{prop}\label{prop:fin}\label{prop:app}\label{prop:finapp}
\begin{enumerate}
\item  For any $S\subseteq \Sigfun^n 1$ it holds $\Asf\models S$
  iff $\Asf\models\zeta_n\inv(S)$.
\item  For any $S\subseteq Z$ it holds $\Asf\models S$ only if
  $\Asf\models\zeta_n(S)$.
\item  $\fa S\subseteq Z \tp S \subseteq \zeta_n\inv(\zeta_n(S))$ and
  $\fa S\subseteq\FT^n 1\tp\zeta_n(\zeta_n\inv(S))=S$.
\end{enumerate}
\end{prop}

We are now able to prove

\begin{thm}
  A class $\bcal\subseteq\CoalgFun$ is definable by a subset
  $S\subseteq\Sigfun^n 1$ iff $\bcal$ is closed under images, coproducts,
  domains of morphisms, and $\sim_n$.
\end{thm}

\begin{proof}
  `only if': Closure under images, coproducts, domains of morphisms is
  standard (and easy to check) and closure under $\sim_n$ is immediate
  from
  the definitions.\\ %
  For `if' let $\Ssf=(S,\sigma)$ be the coalgebra given by the union of the
  images of all $\ineins_\Bsf:\Bsf\maps\Zsf$, $\Bsf\in\bcal$.  We show that
  $\bcal=\Mod(\zeta_n(S))$.  For $\Bsf\in\bcal$ we have, by definition of
  $S$, $\Bsf\models S$, hence
  $\Bsf\models \zeta_n(S)$ by Proposition~\ref{prop:app}.2.
  %
  To show $\bcal\supset\Mod(\zeta_n(S))$, define $\Ssfbar=(\Sbar,\sigmabar)$
  as the largest subcoalgebra of $\zeta_n\inv(\zeta_n(S))$. It follows from
  Proposition~\ref{prop:finapp}.3 that $\zeta_n(S)=\zeta_n(\Sbar)$,
  hence  $\Ssf\sim_n\Ssfbar$.
  %
  Since $\bcal$ is closed under images and coproducts, $\bcal$ is also closed
  under unions, hence $\Ssf\in\bcal$.  Since $\bcal$ is closed under $\sim_n$,
  $\Ssfbar\in\bcal$. Now assume $\Asf\models \zeta_n(S)$. By
  Proposition~\ref{prop:fin}.1, $\ineins_\Asf$ factors through
  $\zeta_n\inv(\zeta(S))$ and hence through $\Ssfbar$, ie.\ there is a morphism
  $\Asf\maps\Ssfbar$.  Since $\bcal$ is closed under domains of morphisms,
  $\Asf\in\bcal$.
\end{proof}

The following corollary is an immediate consequence.

\begin{cor}
\begin{enumerate}
\item  Suppose $\Lang(\FT,\PredLift)$ is formula-expressive. Then
  a class
  $\bcal\subseteq\CoalgFun$ is definable by a formula of depth $n$ iff
  $\bcal$ is closed under images, coproducts, domains of morphisms,
  and $\sim_n$.
\item If $\Lang(\FT,\PredLift)$ is formula-set-expressive, then
  a class $\bcal \subseteq \CoalgFun$ is definable by a set
  of formulas of depth $n$ iff $\bcal$ is closed under images,
  coproducts, domains of morphisms, and $\sim_n$.
\end{enumerate}
\end{cor}

\noindent
A similar proof gives an expressiveness result for sets of finitary formulas.


\begin{thm}
  A class $\bcal$ of coalgebras is definable by a set $\scal$ with
  $S\in\scal\oif \ex n\, (S\subseteq \Sigfun^n 1 \y n<\omega)\}$ iff $\bcal$ is
  closed under images, coproducts, domains of morphisms, and $\sim_{<\omega}$.
\end{thm}

\begin{cor}
  If $\Lang(\FT,\PredLift)$ is formula-set-expressive, then a class
  $\bcal\subseteq\CoalgFun$ is definable by a set of formulas iff
  $\bcal$ is closed under images, coproducts, domains of morphisms,
  and $\sim_{<\omega}$.
\end{cor}

For sufficient conditions ensuring formula-expressiveness and formula-set expressiveness, see Propositions \ref{propn:surjection} and \ref{propn:surjection-weak}.





\section{A Canonical Model Construction for Coalgebras} \label{section:canonical}

Reasoning about behaviours, the final coalgebra plays a central role
because, given the unique coalgebra morphism
$\ineins_\Asf:\Asf\maps\Zsf$ from a coalgebra $\Asf$ into the final
coalgebra $\Zsf$, for every element $a$ of $\Asf$, we can consider
$\ineins_\Asf(a)$ as the behaviour of $a$. Similarly, coalgebras final
in $\Log(\FT)$ (cf.\ Definition~\ref{def:log}) consist of the finite
behaviours. We first show that final coalgebras exist in $\Log(\FT)$
and then show how they generalise the canonical model construction
known from modal logic.


\subsection{Coalgebras Final in $\Log(\FT)$}

A coalgebra final in $\Log(\FT)$ should ``realise'' precisely all
$n$-behaviours, $n<\omega$. Accordingly, the carrier of a final
object in $\Log(\FT)$ will be a subset of $\FT^\omega 1$.

Recall that, given any structure $(C, \gamma)$, we write
$\gamma_{\omega}$ for the unique mediating map $\gamma_{\omega}: C \to
\FT^{\omega} 1$. That is, all $\omega$-behaviours appear as some
$\gamma_{\omega}(c)$ in $\FT^{\omega} 1$.
%
On the other hand, not every point $t \in \FT^{\omega} 1$ can be
presented as $t = \gamma_{\omega}(c)$ by some structure $(C, \gamma)$
and some $c \in C$.  Consider for example the finite powerset functor
$\FT = \Powfin$.  Worrell~\cite{worrell:cmcs99} shows, that for the
final $\FT$-coalgebra $(Z, \zeta)$ the morphism $\zeta_{\omega}: Z \to
\FT^{\omega} 1$ is not surjective.

\pskip Hence we construct the carrier of the coalgebra final in
$\Log(\FT)$ as consisting of all $t \in \FT^{\omega} 1$, which can be
``realised'' by some structure, ie.\ for which there are $(C, \gamma)
\in \CoAlg(\FT)$ and $c \in C$ such that $\gamma_{\omega}(c) = t$. It
then remains to find an appropriate coalgebra structure on the
carrier. 

% The present section makes this idea formal in that it shows, that the above
% construction yields a final object in $\Log(\FT)$. 
\mmskip Throughout, we fix the set $K$ of ``realisable'' elements $t
\in \FT^{\omega} 1$, which is given by
%
\[
  K = \lbrace t \in \FT^{\omega} 1 \sbars \exists (C, \gamma) \in \CoAlg(\FT)\tp \ex c
        \in C\tp \gamma_{\omega}(c) = t \rbrace.
\]
%
For each $k \in K$, we can now choose $(C^k, \gamma^k) \in \CoAlg(\FT)$ and $c^k
\in C^k$ such that $\gamma_{\omega}^k(c_k) = k$. Note that $K$ is a set, which
enables us to consider
\[
  (C,\gamma) = \coprod_{k \in K} (C^k, \gamma^k)
\]
where the coproduct is taken in $\CoAlg(\FT)$.
% (for the cocompleteness of $\CoAlg(\FT)$ see [Rutten]).
%
Denoting the coproduct injections by $\inj_k: C_k \to C$ (which, by
the construction of coproducts in $\CoAlg(\FT)$ are also coproduct
injections in the category of sets), we are ready to note:

\begin{lemma}
$\gamma_{\omega} \circ \inj_k(c) = \gamma^k_{\omega} (c)$ for all $k \in K$ and
$c \in C_k$.
\end{lemma}
%
\begin{proof}
  Since $\gamma^k_{\omega}$ is the unique mediating map into the limiting cone
  with vertex $\FT^{\omega} 1$, it suffices to prove that $\gamma_n \circ
  \inj_k (c) = \gamma_n^k(c)$ for all $n \in \Nat$. For $n = 0$, this is
  obvious. For the induction step we calculate 
%
$\gamma_{n+1} \circ \inj_k (c)%
  = %\text{(def.\ of $\gamma_{n+1}$)} 
    \FT \gamma_n \circ \gamma \circ \inj_k(c)%
  = %\text{(coproduct)}
    \FT \gamma_n \circ \FT \inj_k \circ \gamma^k(c) %
  = %\text{(ind.hyp.)} 
    \FT \gamma_n^k \circ \gamma^k(c) %
  = \gamma_{n+1}^k $
% \begin{align*}
%   \gamma_{n+1} \circ \inj_k (c)
%         & = \FT \gamma_n \circ \gamma \circ \inj_k(c) 
%         && \mbox{(By the construction of coproducts in $\CoAlg(\FT)$)} \\
%         & = \FT \gamma_n \circ \FT \inj_k \circ \gamma^k(c) \\
%         & = \FT \gamma_n^k \circ \gamma^k(c) \\
%         & = \gamma_{n+1}^k
% \end{align*}
establishing the claim.
\end{proof}

We obtain the following immediate corollary:

\begin{cor}
For all $k \in K$ there exists $c \in C$ with $\gamma_{\omega}(c) = k$.
\end{cor}

\noindent
In other words, $\gamma_{\omega}$ factors through $K$ as $\gamma_{\omega} = m
\circ e$, $m$ injective, $e$ surjective. Now consider the diagram
%
\[\xymatrix{
  \FT \FT^{\omega} 1 & \FT K \ar[l]_{\FT m} &
  \FT C \ar[l]_{\FT e} \\
  \FT^{\omega} 1 & K \ar[l]^m \ar@{-->}[u]^{\kappa} \ar@/_/[r]_{o} & C
  \ar[l]_e \ar[u]_{\gamma} }\] 
%
where $o$ is any one-sided inverse of $e$, ie.\
$e \circ o = \id_K$, the existence of which is guaranteed by $e$ being a
surjection.  We let 
%
$$\kappa = \FT e \circ \gamma \circ o.$$
%
Note that $\kappa: K \to \FT K$ makes $K$ into a $\FT$-coalgebra.  Denoting
the limit projections by $p^{\omega}_n: \FT^{\omega} 1 \to \FT^n 1$, we obtain

\begin{lemma} \label{lemma:kappa-n}
  For all $n \in \Nat$, $\kappa_n = p^{\omega}_n \circ m$, hence
  $m=\kappa_\omega$.
\end{lemma}
%
\begin{proof}
We proceed by induction on $n$, where the case $n = 0$ is evident. We calculate
  $\kappa_{n+1}%
        = \FT \kappa_n \circ \kappa %
        = \FT(p^{\omega}_n \circ m) \circ \FT e \circ \gamma \circ o %
        = \FT p^{\omega}_n \circ \FT (m \circ e) \circ \gamma \circ o %
        = \FT p^{\omega}_n \circ \FT \gamma_{\omega} \circ \gamma \circ o %
        = \FT \gamma_n \circ \gamma \circ o %
        = \gamma_{n+1} \circ o %
        = p^{\omega}_{n+1} \circ \gamma_{\omega} \circ o %
        = p^{\omega}_{n+1} \circ m \circ e \circ o %
        = p^{\omega}_{n+1} \circ m$
% \begin{align*}
%   \kappa_{n+1}
%         & = \FT \kappa_n \circ \kappa \\
%         & = \FT(p^{\omega}_n \circ m) \circ \FT e \circ \gamma \circ o \\
%         & = \FT p^{\omega}_n \circ \FT (m \circ e) \circ \gamma \circ o \\
%         & = \FT p^{\omega}_n \circ \FT \gamma_{\omega} \circ \gamma \circ o \\
%         & = \FT \gamma_n \circ \gamma \circ o \\
%         & = \gamma_{n+1} \circ o \\
%         & = p^{\omega}_{n+1} \circ \gamma_{\omega} \circ o \\
%         & = p^{\omega}_{n+1} \circ m \circ e \circ o \\
%         & = p^{\omega}_{n+1} \circ m
% \end{align*}
for the induction step, as desired.
\end{proof}

The proof of the main theorem of this section is now straightforward.

\begin{thm} \label{thm:log-final}
$\Log(\FT)$ has a final object.
\end{thm}
%
\begin{proof}
  We show that $(K, \kappa)$, as constructed above, is final in
  $\Log(\FT)$. Take any object $(D, \delta) \in \Log(\FT)$.
% We first construct a morphism $h: (D,
% \delta) \to (K, \kappa) \in \Log(\FT)$ and then show its uniqueness.
%
Consider the mapping $\delta_{\omega}: D \to \FT^{\omega} 1$, which is the
unique mediating map between the cones $(D, (\delta_n)_{n \in \Nat})$ and
$(\FT^{\omega} 1, (p^{\omega}_n)_{n \in \Nat})$. By construction,
$\delta_{\omega}$ factors as $\delta_{\omega} = m\circ h$ where $m: K \to
\FT^{\omega} 1$ is as above. By Lemma~\ref{lemma:kappa-n}
$$\delta_{\omega} = \kappa_\omega\circ h,$$
which implies that $h$ is
a $\Log(\FT)$-morphism.
% where $m: K \to
% \FT^{\omega} 1$ is as above. We claim that $h: (D, \delta) \to (K, \kappa)$ is a
% morphism of $\Log(\FT)$, which is the case iff $\delta_n = \kappa_n \circ h$ for
% all $n \in \Nat$. Using standard properties of the limit projections and Lemma
% \ref{lemma:kappa-n}, we
% calculate
% \begin{align*}
%   \kappa_n \circ h
%         & = p^{\omega}_n \circ m \circ f \\
%         & = f^{\omega}_n \circ \delta_{\omega} \\
%         & = \delta_n
% \end{align*}
% establishing $h: (D, \delta) \to (K, \kappa) \in \Log(\FT)$.
$h$ is unique since $\kappa_\omega$ is injective.
% For uniqueness, suppose $h_1, h_2: (D, \delta) \to (K, \kappa) \in \Log(\FT)$.
% Again invoking Lemma \ref{lemma:kappa-n}, we obtain for both $h \in \lbrace h_1,
% h_2 \rbrace$
% \begin{align*}
%   f^{\omega}_n \circ m \circ h_i 
%         & = \kappa_n \circ h_i \\
%         & = \delta_n
% \end{align*}
% for all $n \in \Nat$. Hence both $m \circ h_1$ and $m \circ h_2$ are morphisms
% between the cones $(D, (\delta_n)_{n \in \Nat})$ and $(FT^{\omega} 1,
% (f^{\omega}_n)_{n \in \Nat})$. Since the latter cone is limiting, we obtain $m
% \circ h_1 = m \circ h_2$. By $m$ being an injection, we finally establish $h_1 =
% h_2$.
%%
\end{proof}

\begin{remark}\label{rem:p-surjective}
  Note that a coalgebra final in $\Log(\FT)$ is not determined uniquely up
  to \emph{coalgebra}-isomorphisms. In case that
  $p^{\omega+1}_\omega:\FT\FT^\omega 1\to \FT^\omega 1$ is
  surjective\footnote{Which is the case for all examples in this paper with
    the exception of $\Sigfun=\Powfin$. A sufficient condition for
    $p^{\omega+1}_\omega$ to be surjective is that $\FT$ weakly preserves
  limits of $\omega\op$-chains.}, %
the coalgebras final in $\Log(\FT)$ have a simple description, since then
$m:K\maps \FT^\omega 1$ as above is iso.
%  and $\kappa$ as a one-sided inverse
% $p^{\omega+1}\circ\kappa=\id$. 
Indeed, if $p^{\omega+1}_\omega$ is surjective then a coalgebra is final in
$\Log(\FT)$ iff it is isomorphic in $\CoalgFun$ to a coalgebra
$\FT^\omega 1\stackrel{\kappa}{\mapss}{\FT\FT^\omega 1}$ with
$p^{\omega+1}_\omega\circ\kappa=\id$.
\end{remark}





\subsection{The Canonical Model}

Let $\mlfun$ be the functor $\Pow\times\Pow\Prop$, $\Prop$ a countably infinite
set.

\pskip The \emphdef{canonical model} (see for example
\cite{brv,goldblatt:ltc})
for the modal logic $\ml$ is the
$\mlfun$-coalgebra $(L,\langle \lambda_R,\lambda_V\rangle)$
%
{\renewcommand{\arraystretch}{1.2}
\begin{center}
\begin{tabular}{lll}
$L$ & & $\{\Phi\subseteq\ml: \Phi \text{ is maximally consistent}\}$\\
$\lambda_R:L\to\Pow L$ & &
$\Phi\mapsto\{\Psi:\psi\in\Psi\oif\Diamond\psi\in\Phi\}$\\
$\lambda_V:L\maps\Pow\Prop$ && $\Phi\mapsto\Phi\cap\Prop$
\end{tabular}
\end{center}
}
%
\noindent
The canonical model is final in the category $\Log_\ml$ that has
$M$-coalgebras as objects and morphisms $f:(A,\alpha)\to(B,\beta)$ are
functions $f:A\to B$ such that for all $a\in A$, $a$ and $f(a)$ have the same
modal theory.% (we write $\Th(a)=\Th(f(a))$).


From a coalgebraic viewpoint, finitary formulas correspond to subsets
of $\FT^{n} 1$. Taking inverse images along the limit projections,
each set of finitary formulas can be understood as a subset of
$\FT^{\omega} 1$, and maximally consistent subsets $\Phi \subseteq
\Lang$ correspond to minimal (ie.\  singleton) subsets $\lbrace t
\rbrace \subseteq \FT^{\omega} 1$ of $\FT^{\omega} 1$. We make this
precise by showing that the categories $\Log(M)$ and $\Log_\ml$ are
actually identical.

\begin{prop} $\Log(M) = \Log_\ml$. \end{prop}

\begin{proof}
  We have to show that for any coalgebras $(A,\alpha)$, $(B,\beta)$ and any
  function $f:A\to B$, 
%
$$\beta_\omega\circ f(a)=\alpha_\omega(a) \ssi
  \Th(a)=\Th(f(a)),$$
%
  which is equivalent to $[\fa n<\omega\tp \beta_n\circ
  f(a)=\alpha_n(a)] \ssi [\fa n<\omega\tp \fa\phi\in\ml\tp
  \depth(\phi)=n\oif (a\models\phi\ssi f(a)\models\phi)]$ which in turn
  is a consequence of Proposition~\ref{prop:bounden-beh-equ-ml}.
\end{proof}

Since in the case of $\mlfun$, the projection
$p^{\omega+1}_\omega:\mlfun\mlfun^\omega 1\to \mlfun^\omega 1$ is
surjective, we know by Remark~\ref{rem:p-surjective} that all coalgebras
$(K,\langle\kappa_R,\kappa_V\rangle)$ final in $\Log(\mlfun)$ are
given---up to coalgebra isomorphisms---by $K=\mlfun^\omega 1$ and a
one-sided inverse of $p^{\omega+1}_\omega$. Since we have just shown
that the canonical model is final in $\Log(M)$, it is indeed one of
the $(K,\kappa)$ constructed in the previous subsection.


% We will now precisely relate the construction of final objects in
% $\Log(\mlfun)$ of the previous subsection to the definition of the canonical
% model above.  Since in the case of $\mlfun$, the projection
% $p^{\omega+1}_\omega:\mlfun\mlfun^\omega 1\to \mlfun^\omega 1$ is surjective,
% we know by Remark~\ref{} that all coalgebras
% $(K,\langle\kappa_R,\kappa_V\rangle)$ final in $\Log(\mlfun)$ are given---up
% to coalgebra isomorphisms---by $K=\mlfun^\omega 1$ and a one-sided inverse of
% $p^{\omega+1}_\omega$. This determines $\kappa_V$. 
% DAS SCHAFFE ICH NICHT BIS MORGE


















\section{The Topology of Finite Observations}\label{section:topo-logical}

We study the topology on a coalgebra induced by the terminal sequence and
relate logical and topological properties.

%%$
%To start with, consider a sequence of sets $(X_n)_{n<\omega}$ with
%projections $p^{n+1}_n:X_{n+1}\to X_n$ and a cone
%$(A,(\alpha_n))_{n<\omega}$ over that sequence. In case that each
%$X_n$ is equipped with a topology $\tau_n$ such that the projections
%are continuous, there is an \emph{induced topology $\tau_A$ on $A$}
%given by the basis
%%
%$$\{\alpha_n\inv(o_n) : n<\omega, o_n\in\tau_n\}.$$
%%
%$\tau_A$ is the smallest topology on $A$ such that the $\alpha_n$ are
%continuous. $((A,\tau_A),(\alpha_n)_{n<\omega})$ is a cone over
%$(X_n,\tau_n)_{n<\omega}$ in the category $\Top$ of topological
%spaces. In case that $(A,(\alpha_n)_{n<\omega})$ is a limit of
%$(X_n)_{n<\omega}$ in $\set$, $((A,\tau_A),(\alpha_n)_{n<\omega})$ is
%a limit of $(X_n,\tau_n)_{n<\omega}$ in $\Top$.%\footnote{More
%%  generally, the forgetful functor $U:\Top\to\set$ lifts limits
%%  uniquely, that is, for any diagra $D:I\maps\Top$ there is a unique
%%  limiting cone over the limit of $UD$.}
%
%
%
%We apply this to the situation where the sequence under consideration
%is the finitary part of the terminal sequence $(\FT ^n 1)_{n<\omega}$.
%Recall that any coalgebra $(A,\alpha)$ gives rise to a cone
%$(A,(\alpha_n)_{n<\omega})$ over the terminal sequence.
%
%\begin{defn}
%  Suppose we are given topologies $\tau_n$ on $\FT ^n 1$ and
%  $(A,\alpha)\in\CoalgFun$. Then $\tau_A$ is the topology on $A$
%  induced by the cone $(A,(\alpha_n)_{n<\omega})$. In case the
%  $\tau_n$ are discrete\footnote{A topology is \emph{discrete} iff it
%    contains all subsets.}, we call $\tau_A$ the Cantor space topology.
%\end{defn}
% Since for any
% coalgebra $(A,\alpha)$, we have a cone $(A,(\alpha_n)_{n<\omega})$, the
% terminal sequence induces a topology $\tau_A$ on $A$. 
% % We call $\tau_A$ the
% % \textbf{induced topology}.
\begin{defn}
  Suppose $\tau_n$ is a topology on $\FT^n 1$ for all finite ordinals.
  If $(A, \alpha) \in \CoalgFun$, then the topology \emph{induced by
    $(\tau_n)_{n \in \omega}$} is the topology generated by the base
\[
  \lbrace \alpha_n\inv(o) \sbars n \in \omega, o \in \tau_n \rbrace
\]
of open sets. If all $\tau_n$ are discrete, we call $\tau_A$ the
Cantor space topology.
\end{defn}

Clearly, $\tau_A$ makes all projections $\alpha_n$ continuous.
Viewing the Cantor set as the final coalgebra for $\FT X = 2 \times X$, the
topology induced by the discrete $\tau_n = \Pow(\FT^n 1)$, one recovers the
cantor discontinuum.
%
\begin{example} \label{example:mittel-drittel}
  Suppose $\FT X = 2 \times X$, where $2 = \lbrace 0, 1 \rbrace$.
  Consider the (final) $\FT$-coalgebra $(A, \alpha)$ with $A =
  2^{\Nat} = \lbrace f : \Nat \to 2 \rbrace$
  and $\alpha(f) = (f(0), \lambda n\tp f(n+1))$. Then $(A, \tau_A)$ is
  homeomorphic to the Cantor discontinuum $C$ (also known as
  middle-third set, see eg.\ \cite{kelley:gt}) via the mapping
  $2^{\Nat} \to C$, $f \mapsto \sum_{i = 0}^{\infty}
  \frac{2}{3^{i+1}} \cdot f(i)$.
%\[\begin{array}{rccc}
%  h: & 2^{\Nat} & \to & C \\
%           & f & \mapsto & \sum_{i = 0}^{\infty} \frac{2}{3^{i+1}}
%\end{array}\]
\end{example}


\begin{remark} \label{remark:metric-space}
\begin{enumerate}
\item Suppose $f: (A, \alpha) \to (B, \beta)$ is a homomorphism of coalgebras.
  Then $f$ is continuous wrt.\ the topologies on $A$ and $B$. Thus the passage
  from a coalgebra $(A, \alpha)$ to the topological space $(A, \tau_A)$
  induces a functor $\CoAlg(\FT) \to \Top$.
\item %Suppose the $\tau_n$ are discrete\footnote{A topology is discrete iff it
%    contains all subsets.}. 
  Let $(A, \alpha) \in \CoAlg(\FT)$ and let, for $a_0,a_1\in A$,
  $d(a_0, a_1) = \inf \lbrace 2^{-n} : \forall k < n \tp \alpha_k(a_0)
  = \alpha_k(a_1) \rbrace$. Then $d$ is an ultrametric on $A$ and the
  Cantor space topology
  %topology induced by $\tau_n = \Pow(\FT^n 1)$ discrete
  coincides with the topology induced by $d$, as studied in
  \cite{barr:tc,worrell:diss}.
\end{enumerate}
\end{remark}


%\noindent
The topologies $\tau_n$ on $\FT^n 1$ of interest to us are those given
by a basis of `finitely observable properties'
%
$$\bcal_n=\{\sem{\phi}_n: \phi\in\Lang, \phi\text{ a formula of depth
  $n$}\},$$
%
for some logic $\Lang$. To make precise the assumptions on $\Lang$
that are needed in the following we make the
%
\begin{conv}\label{convention}
  Given a functor $\FT$, a logic $\Lang$ consists of sets of formulas
  $\Lang_n, n<\omega,$ equipped with functions
  $\sem{\cdot}_n:\Lang_n\to\Pow(\FT^n 1)$ which assign to each formula
  of depth $n$ a predicate of depth $n$. The semantics of $\Lang$ is
  determined by the terminal sequence as in
  Definition~\ref{def:predicates-depth-n}. Moreover, we assume that
  each $\Lang_n$ is closed under boolean operators which are
  interpreted in the usual way on $\Pow(\FT^n 1)$.
\end{conv}


Since we require $\Lang$ to have boolean operators, $\bcal_n$ is
indeed closed under intersections and hence a basis.\footnote{A
  \emph{basis} is closed under finite intersections. A set is
  \emph{open} iff it is the union of sets in the basis, \emph{closed}
  iff it is the complement of an open set, and \emph{clopen} iff it is
  open and clopen.} Moreover, since $\sem{\phi}_n=\FT^n
1\setminus\sem{\neg\phi}_n$, every set in $\bcal_n$ is clopen.

\pskip That is, in our setting, the relationship between logic and
topology is given by the correspondence between formulas of depth $n$
and clopens of $\FT^n 1$. For most of the following results, relating
logical and topological properties, we therefore impose the following
%
\begin{condition}\label{cond1}
  Given a logic $\Lang$ as described by Convention~\ref{convention},
  the topologies $\tau_n$ on $\FT ^n 1$ given by the basis
  $\{\sem{\phi}_n: \phi\in\Lang_n\}$ make the projections continuous.
\end{condition}
%
\noindent
The condition ensures that $(\FT^n 1,\tau_n)_{n<\omega}$ is a sequence
of topological spaces, each space having a basis of clopens, and
clopens determining precisely the subsets definable by single
formulas. In most examples, the following stronger condition also
holds.
%
\begin{condition}\label{cond2}
  In addition to Condition~\ref{cond1} require that the topologies $\tau_n$
  are compact\footnote{A set is \emph{compact} iff any open cover has a finite
    subcover.  This is sometimes called quasi-compact.}.
\end{condition}
%
\begin{example}\label{exle:conditions}
\begin{enumerate}
\item In case that $\Lang$ is $\Lang(\FT,\PredLift)$ with $\FT$
  mapping finite sets to finite sets and a separating set of predicate
  liftings $\PredLift$, Condition~\ref{cond2} is satisfied.
\item In case that $\Lang$ is $\ml$, Condition~\ref{cond2} is satisfied. We
  skip the proof and only mention that compactness of the $\tau_n$ can be
  deduced from the compactness of $\ml$, similarly to the argument of `only
  if' in the proof of Proposition~\ref{prop:log-top-cmp}.
%\item In case that $\FT X=A\times X$, $\Lang(\FT,\PredLift)$ satisfies
%  Condition~\ref{cond1} for the predicate liftings ... .  But no logic can
%  satisfy Condition~\ref{cond2} IST DIESE BEHAUPTUNG ZU STARK ODER DOCH WAHR?
\end{enumerate}
\end{example}


\noindent
We give some topological characterisations of logical properties.

\begin{prop}\label{prop:definable=closed}
  Let $(A,\alpha)\in\CoalgFun$ and $\Lang$ a logic. Under
  Condition~\ref{cond1}, a subset of $A$ is definable by a set of formulas iff
  it is closed wrt.\ $\tau_A$.
\end{prop}

\begin{proof}
  It is straightforward to check that $S\subseteq A$ is closed iff there are
  basic opens $O_i\subseteq\FT^{n_i}1$, $i\in I$, such that $S=
  \bigcap\{\alpha_i\inv(\FT^{n_i}1\setminus O_i):i\in I\}$. The proposition
  now follows from Condition~\ref{cond1}.
% %Long Proof:
%   Assume that, given $S\subseteq A$, there is $\Phi$ such that $S=\{a\in A:
%   \fa \phi\in\Phi\tp a\models_\alpha\phi\}$. By Condition~\ref{cond1} there is
%   a family of closed sets $C_\phi\subseteq\FT^{n_\phi} 1$, $\phi\in\Phi$, such
%   that $S=\bigcap\{\alpha_i\inv(C_\phi):\phi\in\Phi\}$. Hence $S$ is closed.
% %  
%   For the converse assume that $S$ is closed. Hence $A\setminus
%   S=\bigcup\{\alpha_i\inv(O_i):i\in I\}$ for some family of basic opens
%   $O_i\subseteq\FT^{n_i}1$, that is, $S=
%   \bigcap\{\alpha_i\inv(\FT^{n_i}1\setminus O_i):i\in I\}$. By
%   Condition~\ref{cond1}, $\FT^{n_i}1\setminus O_i$ is the semantics of a
%   formula in $\Lang$, hence $S$ is definable by some $\Phi\subseteq\Lang$.
\end{proof}

\begin{prop}\label{prop:definable=clopen}
  Suppose $(A, \alpha) \in \CoAlg(\FT)$ and $(A, \tau_A)$ compact and $\Lang$
  a logic. Under Condition~\ref{cond1}, a subset $S \subseteq A$ is definable
  by a single formula in $\Lang$ iff $S$ is clopen.
\end{prop}

\begin{proof}
  Assuming that the topologies $\tau_n$ have a clopen basis, we claim that a
  subset $S\subseteq A$ is clopen iff $S=\alpha_n\inv(O)$ for some $n\in\Nat$
  and some clopen $O\subseteq\FT^n 1$. The proposition then follows from
  Condition~\ref{cond1}.
  
  It follows from the $\alpha_n$ being continuous that all subsets of the form
  $\alpha_n\inv(O)$ are clopen if $O$ clopen. Now suppose $S\subseteq A$ is
  clopen. Then $S=\bigcup\{\alpha_{n_i}\inv(O_i):i\in I\}$ and $A\setminus
  S=\bigcup\{\alpha_{n_j}\inv(P_j):j\in J\}$ for basic clopens
  $O_i\subseteq\FT^{n_i} 1$, $P_j\subseteq\FT^{n_j} 1$. Since $A$ is compact,
  the (disjoint) union $\bigcup\{\alpha_{n_i}\inv(O_i):i\in I\} \cup
  \bigcup\{\alpha_{n_j}\inv(P_j):j\in J\}$ has a finite subcover,
%
  yielding a finite set $I'\subseteq I$ of indices such that $S
  =\bigcup\{\alpha_{n_i}\inv(O_i):i\in I'\} $. Let $m = \max \lbrace n_i:i\in
  I'\rbrace$ and consider the projections $p^m_{n_i}:\FT^m 1\to\FT^{n_i}1$.
  Defining $O=\bigcup\{(p^m_{n_i})\inv(O_{i}): i\in I'\}$ establishes the
  claim.
\end{proof}


Compactness is a property which is unfortunately not present in all models.
% We give two examples of non compact structures. The first one demonstrates
% that arbitrary coalgebras need not be compact, although $\FT X$ is finite for
% every finite $X$ DIE REFERENZ AUF FINITE PASST JETZT NICHT MEHR. The second
% example shows that the finiteness assumption on $\FT$ cannot be dropped.



\begin{example}\label{exle:non-cmp:streams}
  Let $\FT X = D \times X$ and consider the final coalgebra $(Z,
  \zeta)$ given by $Z=D^\omega$.
%   where $Z$ is the set of functions $f: \Nat \to D$ and $\zeta(f) = (f(0),
%   \lambda n. f(n+1))$. 
\begin{enumerate}
\item $(Z, \zeta)$ is compact in the Cantor space
  topology iff $D$ is finite.
% the set $O_n = \lbrace f \in Z \sbars f(0) =
% n \rbrace$ is open for each $n \in \Nat$ and clearly $Z = \bigcup_{n \in \Nat}
% O_n$. It is easy to see that the cover $(O_n)_{n \in \Nat}$ does not admit a
% finite subcover.
\item Suppose $D=\{a,b\}$. Then examples of non-compact coalgebras are given by
  the carriers $Z\setminus\{b^\omega\}$ and $\{s\cdot a^\omega:
s\in\{a,b\}^*\}$ (and inheriting the structure from $\zeta$).
\end{enumerate}
\end{example}

\begin{example}
  Let $\FT X = \{a,b\} \times X + 1$ and consider the final coalgebra $(Z,
  \zeta)$ with $Z=\{a,b\}^*\cup\{a,b\}^\omega$. Then $(Z, \zeta)$ is not
  compact in the Cantor space topology: consider eg.\  the open cover given by
  $\{\{w\}: w\in\{a,b\}^*\}\cup\{\{a,b\}^\omega\}$.
\end{example}

\begin{example}\label{exle:non-cmp:finite-powerset}
Let $\FT=\Powfin$. Then the final coalgebra is not compact.
\end{example}


%\section{Topological and Logical Compactness} \label{section:topo-logical}


For the main result of this section, call a $\FT$-coalgebra
$(A,\alpha)$ \emphdef{logically compact}, if every set, which is
finitely satisfiable in $(A,\alpha)$ (that is, for every finite subset
$\Phi' \subseteq \Phi$ there exists $a \in A$ such that $a \models
\Phi'$) is satisfiable in $(A,\alpha)$ (ie.\ there exists $a \in A$
such that $a \models \Phi$). We are now ready to prove

\begin{prop} \label{prop:log-top-cmp}
Let $(A,\alpha)\in\CoalgFun$ and $\Lang$ a logic. Under
  Condition~\ref{cond1}, 
  $(A,\alpha)$ is logically compact iff 
  $(A,\alpha)$ is compact.
\end{prop}
\begin{proof}
  We use that $A$ is compact iff every system $\mathcal S \subseteq \Pow(A)$
  of closed subsets, which has the finite intersection
  property\footnote{$\mathcal S$ has the \emph{finite intersection property}
    iff $\bigcap\mathcal S'$ is non-empty for all finite $\mathcal
    S'\subseteq\mathcal S$.}, %
  has non-empty intersection.  
  
\sskip
  Assume that $(A,\alpha)$ is logically compact and that $\mathcal S
  \subseteq \Pow(A)$ is a system of closed sets having the finite
  intersection property. Every set $S \in \mathcal S$ is definable by
  some $\Phi_S \subseteq \Lang$ by
  Proposition~\ref{prop:definable=closed}. It follows from $\mathcal
  S$ having the finite intersection property that $\bigcup \lbrace
  \Phi_S \sbars S \in \mathcal S \rbrace$ is finitely satisfiable,
  thus satisfiable by logical compactness. That is, there exists $a
  \in A$ such that $a \models \bigcup \lbrace \Phi_S \sbars S \in
  \mathcal S \rbrace$ which implies $a \in \bigcap \mathcal S$ by
  construction.
  
  \sskip Now assume $(A,\alpha)$ is topologically compact and consider a set
  $\Phi \subseteq \Lang$ which is finitely satisfiable.  Since $\lsem \phi
  \rsem \subseteq A$ is closed by Proposition~\ref{prop:definable=closed}, the
  set $\lbrace \lsem \phi \rsem : \phi \in \Phi \rbrace$ is a
  system of closed sets having the finite intersection property. By
  compactness of $(A,\alpha)$, there exits $a \in \bigcap \lbrace \lsem \phi
  \rsem \sbars \phi \in \Phi \rbrace$, which is equivalent to $a\models \Phi$.
\end{proof}

\begin{example}\label{exle:log-top-cmp}
  We can now easily verify Examples~\ref{exle:non-cmp:streams}.2 and
  \ref{exle:non-cmp:finite-powerset}. For instance, in case of
  $\FT=\Powfin$, it is an easy exercise to write down formulas
  $\phi_n$ which force any point satisfying $\phi_n$ to have at least
  $n$ successors.  The set $\Phi = \lbrace \phi_n \sbars n \in \Nat
  \rbrace$ is then finitely satisfiable, but not satisfiable by a
  $\Powfin$-coalgebra.
\end{example}













\section{Compactness for Coalgebraic Modal Logic}\label{section:compactness}

In (standard) modal logic, compactness is not an issue, since it is inherited
from the corresponding result in first order logic via van Benthem's standard
translation \cite{benthem:mlcl,brv}.  Generalising to coalgebraic modal logic,
the standard translation is no longer available. We therefore have to resort
to different means in order to establish a compactness theorem. 
% What is
% interesting is that logical compactness
Moreover, compactness fails in the general case, for example in case of
image-finite Kripke models (ie $\FT=\Powfin$, cf.\ 
Examples~\ref{exle:non-cmp:finite-powerset} and \ref{exle:log-top-cmp}).
% Consider eg.\ the finite
% powerset functor $\FT = \Powfin$ and the modal logic $\ml$. 
% %  the predicate lifting $\lambda(X)(\ssx) =
% % \lbrace t \in \Powfin(X) \sbars t \subseteq x \rbrace$, which interprets the
% % $\Box$-modality of modal logic (cf. Example \ref{example:kripke-liftings}).
% % Putting $\PredLift = \lbrace \lambda \rbrace$, 
% It is an easy exercise to write down formulas $\phi_n$ which force any point
% satisfying $\phi_n$ to have at least $n$ successors.  The set $\Phi = \lbrace
% \phi_n \sbars n \in \Nat \rbrace$ is then finitely satisfiable, but not
% satisfiable by a $\Powfin$-coalgebra.

Hence we are drawn to investigate sufficient and necessary conditions for the
compactness theorem to hold. Building upon the work of Sections
\ref{section:topo-logical} and \ref{section:canonical}, we obtain that
validity of the compactness theorem is equivalent to the existence of a
\emph{compact} canonical model. We then characterise those endofunctors $\FT$
for which $\Log(\FT)$ has a final object compact in the Cantor space topology
as those endofunctors which weakly preserve the limit of the chain $(\FT^n
1)_{n \in \Nat}$.

% For the remainder of this exposition, we make the same assumptions as in
% Section \ref{section:topo-logical}. That is, we consider a set-endofunctor
% $\FT$, together with a set $\PredLift$ of predicate liftings assuming that
% \begin{enumerate}
% \item $\FT$ maps finite sets to finite sets
% \item $\PredLift$ is separating.
% \end{enumerate}

We say that a set $\Phi \subseteq \Lang(\PredLift)$ is \emph{satisfiable}, if
there exists a $\FT$-coalgebra $(A,\alpha)$ and $c \in C$ such that $c
\models_{\gamma} \phi$ for all $\phi \in \Phi$. We call $\Phi$ \emph{finitely
satisfiable}, if every finite subset of $\Phi$ is satisfiable. Using this
terminology, we are in the position to present the first version of the
compactness theorem:

\begin{thm}
  Under Condition~\ref{cond1}, a logic $\Lang$ for $\FT$-coalgebras is compact
  iff $\Log(\FT)$ has a compact final object.
\end{thm}
%
\begin{proof}
  `only if': By Theorem~\ref{thm:log-final} there exists a final object $(K,
  \kappa) \in \Log(\FT)$. We show that $(K, \kappa)$ is logically compact,
  from which the result then follows by
  Proposition~\ref{prop:log-top-cmp}.  So suppose $\Phi \subseteq \Lang$
  is finitely satisfiable in $(K, \kappa)$. By compactness $\Phi$ is
  satisfiable. Thus there is $(C, \gamma)$ and $c \in C$ such that $c
  \models_{\gamma} \Phi$. Since $(K, \kappa)$ is final in $\Log(\FT)$, there
  is a mapping $u: (C, \gamma) \to (K, \kappa) \in \Log(\FT)$. By definition
  of morphisms in $\Log(\FT)$, we obtain $u(c) \models_{\kappa} \Phi$. Hence
  $\Phi$ is finitely satisfiable in $(K, \kappa)$.
  
  `if': Let $(K, \kappa)$ be compact and final in $\Log(\FT)$ and suppose
  $\Phi \subseteq \Lang$ is finitely satisfiable. Then -- by finality and by
  definition of morphisms in $\Log(\FT)$ -- $\Phi$ is finitely satisfiable in
  $(K, \kappa)$, hence satisfiable in $(K, \kappa)$ by compactness and
  Proposition~\ref{prop:log-top-cmp}.
  %Thus $\Phi$ is satisfiable.
\end{proof}

We now proceed to characterise those endofunctors $\FT$ for which
$\Log(\FT)$ has a compact final object. It turns out that $\Log(\FT)$
has a weakly final object iff $\FT$ weakly preserves the limit of its
final sequence up to $\omega$. More precisely, consider the limiting
cone $\FT^{\omega} 1$ of the sequence
%
\[\xymatrix{
  1 & \FT 1 \ar[l]_{p^1_0} & \FT^2 1 \ar[l]_{p^2_1} & \FT^3 1 \ar[l]_{p^3_2} &
  \dots & \FT^\omega 1 }
\] 
%
with associated projections $p^\omega_n: \FT^{\omega} 1 \to \FT^n 1$, we say that $\FT$
\emph{weakly preserves} the limit of the sequence $(\FT^n 1)_{n \in \Nat}$, if
the cone $(\FT \FT^{\omega} 1, (\FT p^\omega_n)_{n \in \Nat})$ is weakly limiting.

% We pause to state
% \begin{lemma}
% Suppose $f: (C, \gamma) \to (D, \delta)$ is a morphism in $\Log(\FT)$. Then $f$
% is continuous wrt. the cantor space topology.
% \end{lemma}
% \begin{proof}
% Follows immediately from the definitions.
% \end{proof}
% The preceding lemma tells us in particular, that compactness in the cantor space
% topology is stable under isomorphism in $\Log(\FT)$. 

%The proof of the promised characterisation theorem is split in several steps.
We begin by noting that every element of an `approximant' $\FT^n 1$ can be
realised by a coalgebra.

\begin{prop} \label{lemma:gamma-n-n}
  Let $m$ be any mapping $1 \to \FT 1$ and $(C^n, \gamma^n) = (\FT^n 1, \FT^n
  m)$. Then $\gamma_n^n = \id_{C^n}$.
\end{prop}

\noindent
We now show that the carrier of a compact final object in $\Log(\FT)$ is
isomorphic to $\FT^{\omega} 1$. %= \Lim_{n \in \omega} \FT^n 1. 
This is the crucial step in our proof.

\begin{lemma}
  Suppose $(K, \kappa)$ is compact and final in $\Log(\FT)$ and $u: K
  \to \FT^{\omega} 1$ is the unique mediating morphism between the
  cones $(K, (\kappa_n)_{n \in \Nat})$ and $(\FT^{\omega} 1,
  (p^\omega_n)_{n \in \Nat})$. Then $u$ is iso.
\end{lemma}
\begin{proof}
  It follows from the construction in Section~\ref{section:canonical}
  that $u$ is mono. To see that $u$ is epi, it suffices to show that
  for all $t \in \FT^{\omega} 1$ there exists $k \in K$ such that
  $p^\omega_n (t) = \kappa_n(k)$.  Fix some $t \in \FT^{\omega} 1$ and
  let $u^n: (C^n, \gamma^n) \to (K, \kappa)$ denote the unique
  morphism into the final object (with $(C^n, \gamma^n)$ as in
  Proposition~\ref{lemma:gamma-n-n}).
  
  \sskip Define a sequence $(k_n)_{n \in \Nat}$ by $k_n = u^n \circ
  p^\omega_n (t)$.  It follows that $\kappa_n(k_n) = p^\omega_n(t)$
  for all $n \in \Nat$. Note that $(K, \kappa)$ is actually a metric
  space (cf.\ Remark~\ref{remark:metric-space}), and that $(k_n)_{n \in
    \Nat}$ is a Cauchy-sequence. Since compact metric spaces are
  complete, it follows
%by compactness of $K$ 
that $k=\lim k_n$ exists in $K$. Observing $\kappa_n(k)= \kappa_n\circ u^n
\circ p^\omega_n (t)= p^\omega_n (t)$ finishes the proof.
% %
%DAS GEHT NUR FUER METRISCHE RAEUME, ALSO NUR FUER TAU\_N DISKRET??
% the Bolzano-Weierstrass theorem provides us with a converging subsequence
% $k_{n_i} \to k$ as $i \to \infty$ for some $k \in K$. We claim that $p^\omega_n(t) =
% \kappa_n(k)$ for all $n \in \Nat$. Fix some $N \in \Nat$. By $k_{n_i} \to k$,
% there exists $j \in \Nat$ such that $\kappa_N(k_{n_i} = \kappa_N(k)$ for all
% $i \geq j$. Denoting the morphisms of the terminal sequence by $f^n_m: \FT^n 1
% \to \FT^m 1$, we obtain for sufficiently large $i$:
% \begin{align*}
%   \kappa_N(k)
%         & = \kappa_N(k_{n_i}) \\
%         & = f^{n_i}_N \circ \kappa_{n_i}(k_{n_i}) \\
%         & = {n_i}_N \circ p_{n_i} (t) \\
%         & = p^\omega_n(t)
% \end{align*}
% as required.
\end{proof}

Observing that $\FT$ weakly preserves the limit of $(\FT^n 1)_{n \in
  \Nat}$ iff $p^{\omega+1}_{\omega}$ has a one-sided inverse $i$,
$p^{\omega+1}_{\omega}\circ i = \id_{\FT^\omega 1}$, we are now able
to prove the following two propositions. 

\begin{prop}
  Suppose that $\Log(\FT)$ has a final model that is compact wrt.\ the
  Cantor space topology. Then $\FT$ weakly preserves the limit of
  $(\FT^n 1)_{n \in \Nat}$.
\end{prop}

\begin{proof}
  Let $(K,\kappa)$ be final and compact in $\Log(\FT)$ and $u: K \to
  \FT^{\omega} 1$ be the unique mediating morphism between the cones
  $(K, (\kappa_n)_{n \in \Nat})$ and $(\FT^{\omega} 1, (p^\omega_n)_{n
    \in \Nat})$. Due to the lemma above, we can define $i=\FT
  u\circ\kappa\circ u\inv$. It remains to check that indeed
  $p^{\omega+1}_{\omega}\circ i = p^{\omega+1}_{\omega}\circ \FT
  u\circ\kappa\circ u\inv = u\circ u\inv = \id_{\FT^\omega 1}$.
\end{proof}

\begin{prop}
  Suppose the topologies $\tau_n$ on $\FT^n 1$, $n<\omega$, are
  compact Hausdorff. Then $\FT$ weakly preserves the limit of $(\FT^n
  1)_{n \in \Nat}$ only if the final model of $\Log(\FT)$ is compact
  in the induced topology.
\end{prop}

\begin{proof}
Let $p^{\omega+1}_{\omega}\circ i = \id_{\FT^\omega 1}$.
% ... implies that 
% $p^{\omega+1}_{\omega}$ is surjective. 
It was shown in Remark~\ref{rem:p-surjective} that $(\FT^\omega 1,i)$
is final in $\Log(\FT)$. It is compact since $\FT^\omega 1$ is the
limit of compact Hausdorff spaces and the induced topology on a limit
of compact Hausdorff spaces is compact Hausdorff (see
\cite{engelking:gt}~3.2.13).
\end{proof}

We can summarise:

\begin{thm}
  Let $\FT$ map finite sets to finite sets. Then $\Log(\FT)$ has a
  final object that is compact in the Cantor space topology iff $\FT$
  weakly preserves the limit of $(\FT^n 1)_{n \in \Nat}$.
\end{thm}

%\begin{proof}
%Observe that $\FT$ weakly preserves the limit of $(\FT^n 1)_{n \in \Nat}$ iff
%$p^{\omega+1}_{\omega}$ has a one-sided inverse $i$,
%$p^{\omega+1}_{\omega}\circ i = \id_{\FT^\omega 1}$.

%\sskip `if': Let $p^{\omega+1}_{\omega}\circ i = \id_{\FT^\omega 1}$.
%% ... implies that 
%% $p^{\omega+1}_{\omega}$ is surjective. 
%It was shown in Remark~\ref{rem:p-surjective} that $(\FT^\omega 1,i)$
%is final in $\Log(\FT)$. It is compact since $\FT^\omega 1$ is the
%limit of compact Hausdorff spaces (see \cite{engelking:gt}~3.2.13).

%\sskip `only if': Let $(K,\kappa)$ be final and compact in $\Log(\FT)$
%and $u: K \to \FT^{\omega} 1$ be the unique mediating morphism between
%the cones $(K, (\kappa_n)_{n \in \Nat})$ and $(\FT^{\omega} 1,
%(p^\omega_n)_{n \in \Nat})$. Due to the lemma above, we can define $i=\FT
%u\circ\kappa\circ u\inv$. It remains to check that indeed
%$p^{\omega+1}_{\omega}\circ i = p^{\omega+1}_{\omega}\circ \FT
%u\circ\kappa\circ u\inv = u\circ u\inv = \id_{\FT^\omega 1}$.
%%the second
%%equation being due to Proposition~\ref{prop:easy}
%\end{proof}

%\begin{cor}
%  WELCHE LOGIKEN KOMAPKT SIND.   (keinen Bock mehr)
%\end{cor}

\section{Conclusions and Related Work}

We have studied definability and compactness for finitary coalgebraic modal 
logic. The main instrument through which finitary logics have been studied 
is the terminal sequence and -- based on the terminal sequence -- the
shift from the category $\CoAlg(\FT)$ to the category $\Log(\FT)$.




\bibliography{include/coalg}


\end{document}
