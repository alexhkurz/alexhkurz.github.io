% SEN-R0223

\begin{document}

\pagestyle{headings}

\title{Observational Logic, Constructor-Based Logic,\\ and their
       Duality}

\author{Michel Bidoit\\
       {\small email: bidoit@lsv.ens-cachan.fr} \\
       {\small Laboratoire Sp�cification et V�rification (LSV), CNRS \& ENS de
         Cachan, France} \\  \\
%
       Rolf Hennicker\\
       {\small email: hennicke@informatik.uni-muenchen.de}\\
       {\small Institut f�r Informatik, Ludwig-Maximilians-Universit�t M�nchen,
         Germany} \\ \\
%  
       Alexander Kurz\\
       {\small email: kurz@cwi.nl}\\
       {\small CWI, %P.O.Box 94079, 1090 GB 
       Amsterdam, The Netherlands} \\ 
}

\date{}

\sf    

\maketitle
\thispagestyle{empty}

\bigskip

\noindent
ABSTRACT\smallskip

\noindent
  Observability and reachability are important concepts for formal software
  development. While observability concepts are used to specify the
  required observable behavior of a program or system, reachability
  concepts are used to describe the underlying data in terms of datatype
  constructors. In this paper we first reconsider the
  observational logic institution which provides a logical framework for
  dealing with observability. Then we develop in a completely analogous way
  the constructor-based logic institution which formalizes a novel
  treatment of reachability. Both institutions are tailored to capture the
  semantically correct realizations of a specification from either the
  observational or the reachability point of view. We show that there is a
  methodological and even formal duality between both frameworks. In
  particular, we establish a correspondence between observer operations and
  datatype constructors, observational and constructor-based algebras,
  fully abstract and reachable algebras, and observational and inductive
  consequences of specifications. The formal duality between the
  observability and reachability concepts is established in a
  category-theoretic setting.


\bigskip
\bigskip

\noindent
\textsf{\emph{1998 ACM Computing Classification System:}} D.2.1,
D.2.4, F.3.1, F.3.2
\medskip

\noindent
\textsf{\emph{Keywords and Phrases:}} 
Algebraic specification,  observability,  reachability,  duality,  institution
% keywords here, in the form: keyword \sep keyword
% PACS codes here, in the form: \PACS code \sep code
%\PACS 
\medskip

\noindent
\textsf{\emph{Note:}} This work was partially supported by the ESPRIT
Working Group 29432 \CoFI{}, by the Bayer.\ Forschungsstiftung, and by
the German DFG-project InopSys. Preliminary results of this study have
been published in~\cite{bhk-fossacs-2001}, and preliminary results
about the observational logic institution have been published
in~\cite{hb:ol}.  


\rm

\newpage

\end{document}