% ----------------------------------------------------------------
% AMS-LaTeX Paper ************************************************
% **** -----------------------------------------------------------
\documentclass{book}
\usepackage{aiml08}
\usepackage{graphicx}
\usepackage{bussproofs}
\usepackage{amssymb}
\usepackage{amsfonts}
\usepackage{latexsym}
\usepackage{amstext}
\usepackage{amsmath}
\usepackage[all]{xy}
\usepackage{bbm} %Blackbord
\usepackage{enumerate}
%\usepackage{amsthm}
% ----------------------------------------------------------------
%\vfuzz2pt % Don't report over-full v-boxes if over-edge is small
%\hfuzz2pt % Don't report over-full h-boxes if over-edge is small
% THEOREMS -------------------------------------------------------
%\newtheorem{thm}{Theorem}[section]
%\newtheorem{cor}[thm]{Corollary}
%\newtheorem{lem}[thm]{Lemma}
%\newtheorem{prop}[thm]{Proposition}
%\theoremstyle{definition}
%\newtheorem{defn}[thm]{Definition}
%\theoremstyle{remark}
%\newtheorem{rem}[thm]{Remark}
%\newtheorem{ex}[thm]{Example}
%\numberwithin{equation}{section}
%\newenvironment{proofof}[1]{\begin{trivlist}\item[\hskip\labelsep{\sc
%Proof~of~{#1}.\ }]}{\hspace*{\fill} {\sc qed}\end{trivlist}}
% ----------------------------------------------------------------

%\bibliographystyle{plain}
\collection
\begin{document}
\paper[Completeness of the finitary Moss logic]{Completeness of the finitary Moss logic}
{Clemens Kupke\thanks{Supported by NWO under FOCUS/BRICKS grant 642.000.502.},
Alexander Kurz\thanks{Partially supported by EPSRC grant EP/C014014/1 }, 
Yde Venema\thanks{The research of this author has been made possible by 
    VICI grant 639.073.501 of the Netherlands Organization for Scientific
    Research (NWO).}}

%%%%%%%%%%% kkv macros %%%%%%%%%%%

% some abbreviations
\newcommand{\xxviom}{\omega}

% Functors & category-theoretic notions
\newcommand{\xxviSet}{\mathsf{Set}}
\newcommand{\xxviBA}{\mathsf{BA}}

\newcommand{\xxviId}{\mathit{Id}}
\newcommand{\xxviF}{T}
\newcommand{\xxviFom}{\xxviF_{\omega}}
\newcommand{\xxviPw}{\mathcal{P}} %covariantpowerset
\newcommand{\xxviPc}{P} %contravariant powerset
\newcommand{\xxviPom}{\xxviPw_{\omega}}
\newcommand{\xxviMoss}{M}
\newcommand{\xxviFree}{\mathbb{F}}
\newcommand{\xxviFalg}{\mathbb{L}_0}

\newcommand{\xxviPal}{\mathbb{P}}
\newcommand{\xxviLbb}{{\mathbb{L}}} %\L already in use
\newcommand{\xxviM}{\mathbb{M}}

\newcommand{\xxviol}[1]{\overline{#1}}
\newcommand{\xxviid}{{\mathrm{id}}}


% Language
\newcommand{\xxviLang}{\mathcal{L}}
\newcommand{\xxviLcal}{\mathcal{L}}
\newcommand{\xxviBase}{\mathit{Base}}
\newcommand{\xxvinb}{\nabla}
\newcommand{\xxvibw}{\bigwedge}
\newcommand{\xxvibv}{\bigvee}

\newcommand{\xxviSRD}{\mathit{SRD}}

\newcommand{\xxviis}{\approx}
\newcommand{\xxviissm}{\preceq}

% Proofs etc
\newcommand{\xxviD}{\mathcal{D}}
\newcommand{\xxvinax}{\mathbf{M}}

% Coalgebras
\newcommand{\xxvibbS}{\mathbb{S}}

% Algebras
\newcommand{\xxviLTn}{\xxviLang_{n}/{\equiv}}
%\newcommand{\xxviLTM}{\mathbb{H}_{\xxviM}}
\newcommand{\xxviLTM}{\mathcal{M}}
\newcommand{\xxvibbA}{\mathbb{A}}
\newcommand{\xxvibbB}{\mathbb{B}}
\newcommand{\xxvibbtwo}{\mathbbm{2}}

\newcommand{\xxvibap}[2]{\mathrm{BA}\langle #1, #2 \rangle}

\newcommand{\xxvilsem}{\mathopen{[\![}}
\newcommand{\xxvirsem}{\mathclose{]\!]}}
\newcommand{\xxvisem}[1]{\xxvilsem #1 \xxvirsem}
\newcommand{\xxvirst}[1]{\!\upharpoonright_{#1}\,}
\newcommand{\xxviterm}{\mathsf{T}_\xxviBA}
\newcommand{\xxviquo}{\mathfrak{q}}
\newcommand{\xxvicoloneqq}{\mathrel{:=}}
\newcommand{\xxviGraph}{\mathrm{Gr}}
\newcommand{\xxviconv}[1]{#1\breve{\hspace*{1mm}}}

\newcommand{\xxviproofspace}{\vspace{-8pt}} %one has to experiment a bit to find the best value. -10pt or -12 pt 
% \newcommand{\xxvisingleton}{\{\cdot\}}
\newcommand{\xxvisingleton}{\eta}

% Use the following only during the writing process

%\newenvironment{tbs}{%
%   \small\tt
%   \renewcommand{\labelitemi}{$\blacktriangleright$}%
%   \begin{itemize}}{\end{itemize}}
%\newcommand{\btbs}{\begin{tbs}}                       
%\newcommand{\etbs}{\end{tbs}}
%\newcommand{\shtbs}[1]{\begin{tbs} \item #1 \end{tbs}}

% \newcommand{\xxvilatjoin}{\bigsqcup}
\newcommand{\xxvilatjoin}{\xxvibv}
\newcommand{\xxvigen}{\mathcal{G}}
\newcommand{\xxviProp}{\mathrm{Prop}}
\newcommand{\xxviPres}{\mathsf{PRS}}
\newcommand{\xxvip}[1]{[#1]}

\newcommand{\xxviverberg}[1]{}

%%%%%%%%%%% kkv macros %%%%%%%%%%%

\begin{abstract}
  We give a sound and complete derivation system for the valid
  formulas in the finitary version of Moss' coalgebraic logic, for
  coalgebras of arbitrary type.
\end{abstract}

\textbf{Keywords:} coalgebra, modal logic, coalgebraic logic, 
   completeness


%\maketitle
% ----------------------------------------------------------------
\EnableBpAbbreviations

%\input{introduction_final}
\section{Introduction}

Generalizing Kripke models and frames, coalgebras provide a general,
category theoretic account of state-based evolving systems.  This
point of view was emphasized by Rutten~\cite{rutten:uc-j}, who
developed, in analogy with Universal Algebra, the basics of Universal
Coalgebra as a general theory of systems.  One of the strengths of the
coalgebraic approach is that a substantial part of the theory of
systems can be developed uniformly in a functor $\xxviF$ (on the
category $\xxviSet$ of sets and functions), which intuitively
represents the type of the transition system.  For example, as
discovered by Aczel~\cite{aczel:nwfs}, any functor $\xxviF$ induces a
canonical notion of bisimilarity on $\xxviF$-coalgebras.

The research programme of Coalgebraic Logic is to extend this uniform
approach to logics for specifying and reasoning about the behavior of
coalgebras.  This research direction was initiated by
Moss~\cite{moss:cl}, who described a logic for $\xxviF$-coalgebras
uniformly for all set functors $\xxviF$ (satisfying a mild condition).
Moss' fascinating idea was, roughly, to take $\xxviF$ itself as a
modality.  In the case of the power set functor $\xxviPw$, this
modality, denoted as $\xxvinb$,
% and known under the name of `cover modality', 
has surfaced in modal logic from time to time, for instance in Fine's
work~\cite{fine:norm75} on normal forms.  It can be defined using the
standard box and diamond: With $\alpha \in \xxviPw\xxviLang$ a set of
formulas, the formula $\xxvinb\alpha$ can be seen as an abbreviation:
$\nabla\alpha = \Box\bigvee \alpha\wedge \bigwedge \!\Diamond\alpha$,
where $\Diamond\alpha$ denotes the set $\{ \Diamond a \mid a \in
\alpha \}$.  The \emph{semantics} of $\xxvinb$ can be expressed in terms
of the so-called Egli-Milner \emph{lifting} of the satisfaction
relation ${\Vdash} \subseteq S \times \xxviLang$ between states and
formulas to a relation $\xxviol{\Vdash}$ between $\xxviPw S$ (sets of
states) and $\xxviPw\xxviLang$ (sets of formulas):
\begin{equation}
\label{eq:pwnb}
\xxvibbS,s \Vdash \xxvinb\alpha \mbox{ iff } \sigma(s) \;\xxviol{\Vdash}\; \alpha,
\end{equation}
where $\sigma: S \to \xxviPw S$ denotes the successor function.
Since one may associate a reasonable notion of relation lifting with other
set functors as well, the observation (\ref{eq:pwnb}) paves the way for
generalization to an arbitrary functor $\xxviF$.
Moss shows that his coalgebraic logic, based on a modality $\xxvinb_{\xxviF}$, is
invariant under bisimilarity, and, in the presence of infinitary conjunctions,
characterizes bisimilarity.

% Whereas the cover modality $\xxvinb_{\xxviPw}$ has surfaced in modal logic from time
% to time, the 
The operator $\xxvinb_{\xxviF}$ associated with an arbitrary functor
$T$ looks strikingly different from the usual $\Box$ and $\Diamond$
modalities.  Following on from \cite{moss:cl}, attention turned to the
question how to obtain modal languages for $\xxviF$-coalgebras which
use more standard modalities
\cite{kurz:cmcs98-j,roessiger:ml98-j,jacobs:many-sorted}, and how to
find derivation systems for these formalisms.  This approach is now
usually described in terms of predicate
liftings~\cite{pattinson:cml-j,schroeder:fossacs05} or, equivalently,
Stone duality~\cite{bons-kurz:fossacs05,kurz:sigact06}.  For a while,
this approach displaced the interest in Moss' logic and the
relationship between the two was not completely clear.

Interest in Moss' logic revived when it became clear that even in
standard modal logic, a $\xxvinb$-based approach has some advantages.
In fact, independently of Moss' work, Janin \&
Walukiewicz~\cite{jani:auto95} already observed that the connectives
$\xxvinb$ and $\lor$ may in some sense replace the set $\{
\Box,\Diamond,\land,\lor \}$.  This observation, which is closely
linked to fundamental automata-theoretic constructions, lies at the
heart of the theory of the modal $\mu$-calculus, and has many
applications, see for instance~\cite{dago:moda06,sant:comp07}.
Generalizing the link between fixpoint logics and automata theory to
the coalgebraic level of generality, Kupke \&
Venema~\cite{kupk:coal08} generalized some of these observations to
show that many fundamental results in automata theory are really
theorems of universal coalgebra.

This paper addresses the main problem left open by
Moss~\cite{moss:cl}.  Moss' approach focuses on semantics, and he
provides only some sound logical principles which do not constitute a
complete syntactic calculus.  As a first result in the direction of a
derivation system for $\nabla$ modalities, Palmigiano \&
Venema~\cite{palm:nabl07} gave a complete axiomatization for the cover
modality (i.e., $\nabla_{\xxviPw}$ for the power set functor $\xxviPw$).  This
calculus was streamlined by B{\'\i}lkov\'a, Palmigiano \&
Venema~\cite{bilk:proo08} into a formulation that admits a
straightforward generalization to an arbitrary set functor $\xxviF$.

Our main contribution here is a uniform completeness proof.  That is,
in this paper we provide, uniformly in the functor $\xxviF$, a derivation
system $\xxvinax$ which is sound and complete with respect to the
semantics of the coalgebraic language based on the modality
$\xxvinb_{\xxviF}$.  The main idea of the completeness proof is based on the
Stone duality approach to coalgebraic logic and, as a byproduct,
we also see how Moss' language fits into this approach.

In the Stone duality approach to coalgebraic logic, the relationship
between logic and semantic is based on the following situation
\begin{equation}\label{diag:duality}
\xymatrix{
  {\xxviM\textsf{-Alg}}\ar[d] %\ar@/_/[r]_{} 
  & {T\textsf{-Coalg}}\  \ar@/_/[l]_{} \ar[d]\\
  {\ \xxviBA} \ar@(dl,ul)[]^{\xxviM} %\ar@/_/[r]_{}
   & {\xxviSet\ } \ar@/_/[l]_{\xxviPal} \ar@(dr,ur)[]_{T}
   }
\end{equation}
where $\xxviM$ is the functor on Boolean algebras given by the proof
system of the logic under consideration and $\xxviPal$ is the
contravariant powerset functor. The semantics of the logic appears in
this setting as a natural transformation $\delta:\xxviM\xxviPal\to \xxviPal T$
(using $\delta$, $\xxviPal$ lifts to a functor in the upper row, which
maps a $T$-coalgebra to its `complex' $\xxviM$-algebra). The proof system
is complete if $\delta$ is injective (Proposition~\ref{p:3}). One
advantage of this approach is its flexibility. For example,
descriptive-general-frame semantics corresponds to replacing $\xxviSet$ by
Stone spaces. On the algebra side, one can treat positive logic by
replacing $\xxviBA$ by distributive lattices or infinitary logic (like in
Moss' original work) by replacing $\xxviBA$ by complete atomic Boolean
algebras. This paper treats the case of $\xxviBA$ and $\xxviSet$ which is of
particular interest to us and leaves the others for future work. This
means, in particular, that we will concentrate on the finitary version
of Moss' logic first introduced in \cite{venema:coalg-aut}.



%\input{preliminaries_final}
\section{Preliminaries}

In this section we settle on notation and terminology, and we introduce the
finitary version of Moss' logic.
For background on coalgebra the reader is referred to \cite{vene:alge06}.

\paragraph{General}
Two categories play a major role in our paper: the category $\xxviSet$ with
sets as objects and functions as arrows, and the category $\xxviBA$ of Boolean 
algebras and homomorphisms.
The categories $\xxviSet$ and $\xxviBA$ are related by the contravariant functor
$\xxviPal: \xxviSet\to\xxviBA$,
%that turns a set $X$ into its power set algebra $\xxviPal(X) \in \xxviBA$
%and that maps a function $f:X \to Y$ to the inverse image function $f^{-1}: 
%\xxviPal(Y) \to \xxviPal(X)$.
by the forgetful functor $U:\xxviBA\to\xxviSet$, and by the left adjoint
$\xxviFree$ of $U$ mapping a set $X$ to the free Boolean algebra over
$X$. We write $P$ for $U\xxviPal$, $\xxvibbtwo$ for the two-element Boolean
algebra and $1$ for a one-element set.

\paragraph{Coalgebra}
A coalgebra (over $\xxviSet$) for a functor $\xxviF:\xxviSet \to \xxviSet$, 
also called $\xxviF$-coalgebra, is a pair 
$(S,\sigma)$ where $S$ is a set (of ``states'') and 
$\sigma:S \to \xxviF S$ is a function (the ``transition structure'').
A $\xxviF$-coalgebra morphism from a $\xxviF$-coalgebra
$(S_1,\sigma_1)$ to a $\xxviF$-coalgebra $(S_2,\sigma_2)$ 
is a function $f:S_1 \to
S_2$ such that $\xxviF f \circ \sigma_1 = \sigma_2 \circ f$.
 
For a modal logician, the prime examples of coalgebras are Kripke frames
and Kripke models.
%% 
 % \begin{example}\label{ex:kripke}
 %    Let $\xxviProp$ be a set (of propositional variables) 
 %    and consider the functor $\xxviPw \Phi \times \xxviPw$, that maps
 %    a set $X$ to the set $\xxviPw \xxviProp \times \xxviPw X$ and a function
 %    $f:X \to Y$ to the function $\xxviid_{\xxviPw \xxviProp} \times \xxviPw (f)$.
 %    Then a Kripke model $(W,R,V:\xxviProp \to \xxviPw(W))$
 %    corresponds to the $\xxviPw \xxviProp \times \xxviPw$-coalgebra
 %         $(W,
 %         \langle V^\sharp , \lambda x.R[x] \rangle : W \to \xxviPw \xxviProp 
 %         \times \xxviPw W)$ 
 %         where $p \in V^\sharp(w)$ if $w \in V(p)$.
 %    Furthermore $f:(W_1,R_1,V_1) \to (W_2,R_2,V_2)$ is a bounded
 %    morphism between Kripke models iff $f$ is a 
 %    morphism between the corresponding coalgebras.
 %    A similar correspondence holds between $\xxviPw$-coalgebras 
 %    and Kripke frames.
 % \end{example}
 %%
Bisimulations between Kripke structures also have their natural coalgebraic
generalization: a relation $Z$ between the carrier sets of two
coalgebras is a bisimulation if for all $(s_{1},s_{2})\in Z$, the pair
$(\sigma_1(s_1),\sigma_2(s_2))$ belongs to the \emph{relation lifting}
$\xxviol{Z}$ of $Z$.

\begin{definition} \label{d:rellift}
Let $\xxviF$ be a set functor.
Given a binary relation $Z$ between two sets $S_1$
and $S_2$, we define the relation $\xxviol{Z} \subseteq \xxviF S_1 \times \xxviF S_2$
as follows:
\[
\xxviol{Z} := \{ ((\xxviF\pi_{1}) \phi, (\xxviF\pi_{2})\phi) \mid \phi \in \xxviF Z \},
\]
where $\pi_i: Z \to S_i$ for $i=1,2$ are the projection functions.
\end{definition}

In this paper we will confine attention to set functors that are
\emph{standard} (that is, inclusions are mapped to inclusions), and
that \emph{preserve weak pullbacks}.  We will not define the latter
property, but simply note that it is equivalent to requiring that the
composition of two bisimulations is again a bisimulation, or,
equivalently, that for all relations $Z_1,Z_2$ we have
$\xxviol{Z_1\circ Z_2}=\xxviol{Z_1}\circ\xxviol{Z_2}$ (and it will be
apparent from the development below that this property is essential to
work with the Moss modality).  The requirement of standardness is not
essential and only serves to keep the notation a bit smoother.  The
class of standard and weak pullback preserving functors includes the
ones that are used to model infinite words, infinite binary trees,
Kripke frames and probabilistic transition systems as coalgebras.  A
more detailed discussion of these examples can be found in
\cite{kupk:coal08}.  For reasons of space limitations we cannot go
into further detail here.

\begin{convention}
Throughout this paper we fix a standard and weak pullback preserving set
functor $\xxviF$.
\end{convention}

The following fact lists the properties of relation lifting that we use in our
paper. 
(Here $\xxviGraph(f) \subseteq S \times S'$ denotes the graph of a function $f:
S \to S'$.)
For proofs we refer to \cite{moss:cl}
%,baltag:cmcs00}, 
and references therein.

\begin{fact}\label{fact:wpb}
Let $\xxviF$ be a set functor that is standard and weak pullback preserving. 
Then relation lifting 
\\ (1)  extends $\xxviF$: $\xxviol{\xxviGraph(f)} = \xxviGraph(\xxviF f)$,  and 
  preserves the diagonal: $\xxviol{\xxviId_{S}} = \xxviId_{\xxviF S}$; 
\\ (2) is monotone: $R \subseteq Q$ implies $\xxviol{R} \subseteq \xxviol{Q}$;
%\item
\\ (3) commutes with taking restrictions:
$\xxviol{R\xxvirst{U\times U'}} = \xxviol{R}\xxvirst{\xxviF U \times \xxviF U'}$;
\\ (4) preserves composition: $\xxviol{R \circ Q} = \xxviol{R} \circ \xxviol{Q}$ ,
  and converse: $\xxviol{(\xxviconv{R})} = \xxviconv{(\xxviol{R})}$;
%, if $\xxviF$ preserves weak pullbacks.
%% 
 % \begin{enumerate}[(i)]
 % \item  $\xxviol{\xxviGraph(f)} = \xxviGraph(\xxviF f)$ (extends $\xxviF$), 
 % $\xxviol{(\xxviId_{S})} = \xxviId_{\xxviF S}$ (preserves the diagonal) and 
 % $\xxviol{(\xxviconv{R})} = \xxviconv{\xxviol{R})}$ (commutes with converse);
 % %(3) $\xxviol{(\_)}$ commutes with relation converse:
 % %$ \xxviol{(\xxviconv{R})} = \xxviconv{\xxviol{R})}$;
 % \item $R \subseteq Q$ implies $\xxviol{R} \subseteq \xxviol{Q}$
 %   (monotonicity);
 % %\item
 % \item\label{restr} $\xxviol{R\xxvirst{U\times U'}} =
 %   \xxviol{R}\xxvirst{\xxviF U \times \xxviF U'}$ (commutes with
 %   taking restrictions);
 % \item\label{comp} $\xxviol{R \circ Q} = \xxviol{R} \circ
 %   \xxviol{Q}$ (preserves composition)
 % %, if $\xxviF$ preserves weak pullbacks.
 % \end{enumerate}
 %%
\end{fact}
 
We let $\xxviFom$ denote the \emph{finitary}, or,
\emph{$\xxviom$-accessible}, version of $\xxviF$, that is, the set
functor $\xxviFom$ which agrees with $\xxviF$ on finite sets, while
for an infinite set $X$,
\[
\xxviFom(X) := \bigcup \{ \xxviF Y \mid Y \in \xxviPom(X) \}.
\]
On maps, $\xxviFom$ simply agrees with $\xxviF$.  It is not hard to
see that $\xxviFom$ is a well-defined subfunctor of $\xxviF$
(cf.~\cite[p.314]{adam-trnk:automata}) and that $\xxviFom X \subseteq
\xxviF X$ for all sets $X$.  Furthermore, as any standard set functor
preserves finite intersections (\cite[III,
Prop.~4.6]{adam-trnk:automata}), for any set $X$, and any element
$\alpha \in \xxviFom X$, there is a \emph{smallest, finite} subset
$X_{0} \subseteq X$ such that $\alpha \in \xxviFom X_{0}$.  This set
$X_{0}$ is called the \emph{base} of $\alpha$, notation:
$\xxviBase(\alpha)$.




\subsection*{Moss' language}

\begin{definition}\label{def:moss-lang}
Given a set $X$ of proposition letters, we define the
following.  $\xxviLang_{0}(X)$ is the smallest superset of $X$ which is
closed under taking negations and finitary conjunctions and
disjunctions.
% such that $\xxvibv\phi,\xxvibw\phi
% \in Y$ for all $\phi\in\xxviPom(Y)$, and $\neg a \in Y$ if $a \in Y$.
$\xxviLang_{n+1}(X):= \xxviLang_{0}(\{ \xxvinb\alpha \mid \alpha \in
\xxviFom\xxviLang_{n}(X)\})$ is the smallest set containing the formula
$\xxvinb\alpha$ for each $\alpha \in \xxviFom\xxviLang_{n}(X)$, which is closed
under taking negations and finitary conjunctions and disjunctions.
% and such that $\xxvibv\phi,\xxvibw\phi \in Y$ for all $\phi\in\xxviPom(Y)$.
$\xxviLang(X) := \bigcup_{n \in \xxviom} \xxviLang_{n}(X)$ is the set of
\emph{formulas in $X$}; in case $X = \emptyset$ we write $\xxviLang_{n}$
and $\xxviLang$ instead of $\xxviLang_{n}(\emptyset)$ or $\xxviLang(\emptyset)$.
The \emph{depth} of a formula $a$ is the smallest $n$ such that $\phi
\in \xxviLang_{n}$.
\end{definition}

We write $\top := \xxvibw\emptyset$ and $\bot := \xxvibv\emptyset$.  Then by
definition, $\top$ and $\bot$ belong to every layer of the language.
While it is not hard to prove that $\xxviLang_{n} \subseteq \xxviLang_{n+1}$,
for all $n \in \xxviom$, it is in general not the case that $X \subseteq
\xxviLang_{n}$ for $n>0$.

It will occasionally be useful to think of $\xxviLang_{0}(X)$ as the
(carrier of the) absolutely free algebra of Boolean type, or the
\emph{Boolean term algebra}, generated by $X$, and of $\xxviLang_{n+1}$ as
the Boolean term algebra generated by the set $\{ \xxvinb\alpha \mid
\alpha \in \xxviFom\xxviLang_{n} \}$.

%\begin{definition}
%\label{d:Falg}
%\marginpar{\tiny Apologies: \emph{this} notation is what I had in
%  mind.  Now $\xxviLang_{0}(X) = U \xxviFalg(X)$. (yv) If we consider
%  $\xxviFalg$ as a functor on set, then $\xxviLang_0$ would be more
%  in accordance with the use of $\xxviLang_1$ in the completeness
%  section (ak 22-3)} Let $\xxviFalg$ be the functor taking a set $X$
% to the Boolean term algebra.
%\end{definition}

The language can be seen as an initial algebra for a functor.

\begin{proposition}
\label{p:mossinit}
Let $\xxviMoss$ be the set functor $\xxviId + \xxviId\times\xxviId +
\xxviId\times\xxviId + \xxviFom$.  Then
$(\xxviLang,\neg,\wedge,\vee,\xxvinb)$ is the initial
$\xxviMoss$-algebra.
\end{proposition}

\begin{remark}\label{rmk:moss-lang-fun} 
  For the category theoretic minded reader we note that,
  identifying formulas up to Boolean equivalence, Moss' language
  $\xxviLang$ is the initial algebra for the functor
  $\xxviLbb=\xxviFree\xxviF_\omega U: \xxviBA \to \xxviBA$.
\end{remark}

While we will refer to the above language as \emph{Moss'} coalgebraic
language, there are actually some differences.  The most important of
these is that by defining $\xxvinb\alpha$ to be a formula only for
elements $\alpha \in \xxviFom\xxviLang$ (rather than for all $\alpha
\in \xxviF\xxviLang$), we construct a language that is \emph{finitary}
in the sense that every formula has a finite number of
\emph{subformulas}.  This notion can be defined inductively, the key
clause being that the subformulas of $\xxvinb\alpha$ are given as the
closure of the set $\xxviBase(\alpha)$ under subformulas.

Concerning the semantics of $\xxviLang$, we only give the clause for
the $\xxvinb$ modality.

\begin{definition}\label{def:Lang-sem}
Given a coalgebra $\xxvibbS=(S,\sigma)$, we define $s\Vdash\nabla\alpha$
if
$\sigma(s)\xxviol{\Vdash}\alpha$.
\end{definition}

\begin{example}
Let $\xxviProp$ be a set of propositional variables and recall 
% from Example~\ref{ex:kripke} 
that coalgebras for the functor $K = \xxviPw \xxviProp \times \xxviPw$
correspond to Kripke models.  Then any formula $\xxvinb_{K} \alpha$ is
of the form $\xxvinb_{K}\alpha=\xxvinb_{K}(P,A)$ where $P\subseteq
\xxviProp$ is a set of proposition letters and $A \subseteq \xxviLang$ is
a finite set of formulas.  If the set $\xxviProp$ is finite it is easy to
see that one can define a translation $t$ of formulas in $\xxviLang$
into the basic modal language by putting
\[ t(\xxvinb_{K} (P,A)) \xxvicoloneqq
\xxvibw_{p \in P} p \wedge \xxvibw_{p \notin P} \neg p
\wedge \xxvibw_{a \in A} \Diamond t(a) \wedge \Box (\xxvibv_{a \in A} t(a))
\]
such that $(S,\sigma),s \Vdash a$ iff $(S,\sigma),s \Vdash t(a)$ for all $a
\in \xxviLang$.
\end{example}

The semantics of a $\xxvinb$-formula can be also expressed using the
following natural transformation which plays a central role in our
paper.
\begin{definition}
\label{def:TP-PT}
We define a natural transformation $\rho: \xxviF\xxviPc \to
\xxviPc\xxviF$ by putting $\rho_{X}(\Phi) := \{ \alpha \in \xxviF X
\mid \alpha \xxviol{\in} \Phi \}$.
\end{definition}
\begin{remark}
  $\rho$ is natural if $T$ preserves weak-pullbacks. This is also true
  if one replaces the contravariant $P$ with the covariant $\cal P$.
\end{remark}
In order to gain some intuitions about the $\xxvinb$-operator and 
the transformation $\rho$, the reader is invited to prove the following
easy lemma.
\begin{lemma}
\label{l:altsem}
For any $\xxvinb \alpha \in \xxviLang$ we have $s\Vdash\nabla\alpha$ iff
$s\in\sigma^{-1}\circ\rho_S(\xxviF\mu (\alpha))$, where
$\mu: \xxviLang \to \xxviPw S$ is the function that maps a formula
to its semantics. 
\end{lemma}

\begin{remark}\label{rmk:moss-lang-fun-2}
Following on from  Remark~\ref{rmk:moss-lang-fun}, freely extending $\rho$
to Boolean algebras yields a natural transformation
$\gamma:\xxviLbb\xxviPal\to\xxviPal T$. $\gamma$  
allows us to associate with any coalgebra $(S,\sigma)$ a `complex
$\xxviLbb$-algebra' $\xxviLbb\xxviPal S \stackrel{\gamma_{S}}{\to} \xxviPal\xxviF S
\stackrel{\xxviPal\sigma}{\to} \xxviPal S$.  Denote by $\xxviLcal'$ the language
$\xxviLang$ quotiented by Boolean equivalence.  Then $\xxviLcal'$ is the
initial $\xxviLbb$-algebra. For each coalgebra $(S,\sigma)$, initiality of
$\xxviLcal'$ gives us a map $\xxvisem{\cdot}:\xxviLcal'\to\xxviPal S$ interpreting
elements of $\xxviLcal'$ as propositions on $S$. This definition agrees
with Definition~\ref{def:Lang-sem} (because $\gamma$ is the free
extension of $\rho$).
\end{remark}

% \btbs
% \item define validity
% \etbs

%%% Local Variables: 
%%% mode: latex
%%% TeX-master: "nax"
%%% End: 

%\input{derivation_final}
\section{The derivation system}

In this section we will define and discuss the derivation system $\xxvinax$.
Before we can provide the actual definition of $\xxvinax$, we need a few
preparatory remarks and definitions.

First of all, it will be convenient for us to have the derivation
system operating on \emph{inequalities}, that is, expressions of the
form $a \xxviissm b$, with $a,b \in \xxviLang$.  The main reason for this is
that we like our system to stay close to equational reasoning. Indeed,
in any logic with an underlying algebraic semi-lattice structure,
inequalities can be seen as (special) \emph{equations}: we may for
instance identify the inequality $a \xxviissm b$ with the equation $a
\land b \xxviis a$.  Conversely, we may think of an equation $a \xxviis b$ as
a \emph{pair} of inequalities $a\xxviissm b$, $b\xxviissm a$.

\begin{definition}
  An inequality $a \xxviissm b$ is \emph{valid in a coalgebra}
  $\xxvibbS = (S,\sigma)$, notation: $\xxvibbS \Vdash a \xxviissm b$,
  if $\xxvibbS,s \Vdash a$ implies $\xxvibbS,s \Vdash b$ for all $s
  \in S$, and \emph{valid} simpliciter if it is valid in every
  coalgebra, notation: $a \models b$.
\end{definition}

Note that the set of valid formulas can be obtained from the set of valid 
inequalities: a formula $a$ is true in every state in every coalgebra iff the 
inequality $\top \xxviissm a$ is valid.

In the sequel we will need symbols to refer to formulas ($\xxviLang$),
and to elements of the sets $\xxviPom\xxviLang$, $\xxviFom\xxviLang$,
$\xxviFom\xxviPom\xxviLang$ and $\xxviPom\xxviFom\xxviLang$.  For
convenience we fix our notation for such objects as follows:
\begin{center}
\begin{tabular}{|c|c||c|c|}
  % after \\: \hline or \cline{col1-col2} \cline{col3-col4} ...
  \hline
  $\xxviLang$       & $a,b,c,\ldots$ &
  $\xxviFom\xxviLang$   & $\alpha,\beta,\gamma\ldots$ \\
  $\xxviPom\xxviLang$   & $\phi,\psi,\ldots$ &
  $\xxviFom\xxviPom\xxviLang$ & $\Phi,\Psi,\ldots$ \\
  $\xxviPom\xxviFom\xxviLang$ & $A, B, C \ldots$ & &\\ \hline
\end{tabular}
\end{center}
The same notation will be used for variants where $\xxviLang$ is replaced
by an arbitrary set or $\xxviPom,\xxviFom$ are replaced by $\xxviPw,\xxviF$.

An important role in the definition of $\xxvinax$ is played by the notion of a
\emph{slim redistribution}.

\begin{definition}
  A set $\Phi \in \xxviF\xxviPw(X)$ is a \emph{redistribution} of a
  set $A \in \xxviPw\xxviF(X)$ if $A \subseteq \rho_{X}(\Phi)$.  In
  case $A \in \xxviPom\xxviFom(X)$, we call a redistribution $\Phi$
  \emph{slim} if $\Phi \in\xxviFom \xxviPom \big(\bigcup_{\alpha\in A}
  \xxviBase(\alpha) \big)$.  The set of slim redistributions of $A$ is
  denoted as $\xxviSRD(A)$.
\end{definition}

\subsection*{A special case} 

Our derivation system is given in Definition~\ref{def:nax}.  It turns
out, however, that we can give a somewhat simpler version in case the
functor $\xxviF$ restricts to finite sets (that is, if $\xxviF X$ is finite
whenever $X$ is finite).  This simpler system is the direct
generalization of the system for $\xxviF=\xxviPw$ (that is, where the
coalgebras are Kripke structures) given by B\'{i}lkov\'a, Palmigiano
and the third author in~\cite{bilk:proo08}. \footnote{In~\cite{bilk:proo08}
it was shown that for $\xxviF = \xxviPw$ axiom ($\xxvinb 4$) is derivable from
($\xxvinb 1$)-($\xxvinb 3$). We recently discovered that this is also 
true for the case of an arbitrary functor $\xxviF$.} 

$\xxvinax$ is given as follows.  On top of a complete set of axioms
and rules for classical propositional logic, and the cut rule (from $a
\xxviissm b$ and $b \xxviissm c$ derive $a \xxviissm c$), it has the
axioms and derivation rules given in Table~\ref{tb:naxfin}.

\begin{table}[bht]
\begin{center}
\begin{tabular}{|ll|}
\hline 
\\[-4mm]
% ($\xxvinb0$) & 
%   Axioms and rules for Classical Propositional Logic
($\xxvinb1$) & 
From $\alpha\xxviol{\xxviissm}\beta$ infer $\vdash \xxvinb\alpha\xxviissm\xxvinb\beta$
\\ ($\xxvinb2$) &
$\xxvibw \{ \xxvinb\alpha \mid \alpha \in A \} \xxviissm
\xxvibv \{ \xxvinb (\xxviF\xxvibw)(\Phi) \mid \Phi \in \xxviSRD(A) \}$
\\ ($\xxvinb3$) &
$\xxvinb(\xxviF\xxvibv)(\Phi) \xxviissm \xxvibv \{ \xxvinb\alpha \mid \alpha \xxviol{\in} \Phi \}$
\\ ($\xxvinb4$) &
From $\vdash \top \xxviissm \xxvibv\phi$ infer $\vdash \top \xxviissm \xxvibv \{ \xxvinb\alpha \mid 
\alpha \in \xxviF\phi \}$ 
\\ \hline
\end{tabular}
\caption{Axioms and rules of the system $\xxvinax$, if $\xxviF$ restricts to
finite sets}
\label{tb:naxfin}
\end{center}
\end{table}

Let us hasten to give some explanation of the system.  To start with,
the reader may be slightly puzzled by our formulation of the
derivation rule ($\xxvinb1$), since its premiss
`$\alpha\xxviol{\xxviissm}\beta$' uses syntax that has not been
defined as part of the object language.  The proper way to read this
premiss is as follows: `the relation $Z := \{ (a,b) \in
\xxviBase(\alpha) \times \xxviBase(\beta) \mid \vdash a \xxviissm b
\}$ is such that $(\alpha,\beta)\in \xxviol{Z}$'.  In order to see
this, note that using Fact~\ref{fact:wpb}(3) one can show that for all
$\alpha,\beta \in \xxviFom \xxviLang$ and all $Z \subseteq \xxviLang
\times \xxviLang$
\[ 
(\alpha,\beta) \in \xxviol{Z} \quad \mbox{iff} \quad 
(\alpha,\beta) \in \xxviol{Z'}, 
\]    
where $Z' := Z\xxvirst{\xxviBase(\alpha) \times \xxviBase(\beta)}$ is
the restriction of $Z$ to the finite sets $\xxviBase(\alpha)$ and
$\xxviBase(\beta)$.  An alternative formulation of this rule would
therefore say that `if there is a relation $Z \subseteq
\xxviBase(\alpha) \times \xxviBase(\beta)$ such that $(\alpha,\beta)
\in \xxviol{Z}$, and $\vdash a \xxviissm b$ for all $(a,b) \in Z$,
then infer $\vdash \xxvinb\alpha\xxviissm\xxvinb\beta$'.  But the
presentation in Table~\ref{tb:naxfin} is shorter and reveals more
clearly that the rule is in fact the inequality version of a
\emph{congruence} rule.  Our discussion shows that ($\xxvinb1$) is a
\emph{finitary} rule, because its set of premisses can be assumed to
be contained in the finite set $\xxviBase(\alpha) \times
\xxviBase(\beta)$ if we want to derive $\xxvinb \alpha \xxviissm
\xxvinb \beta$.



The axioms ($\xxvinb2$) and ($\xxvinb3$) could in fact both be
replaced with identities, since in both cases, the reverse inequality
of the axiom can be derived as a theorem.  In order to be able to
\emph{read} the axioms $\xxvinb2$ and $\xxvinb3$, recall that
$\xxvibw$ and $\xxvibv$ are maps from $\xxviPom\xxviLang$ to
$\xxviLang$, so that $\xxviF\xxvibw: \xxviF\xxviPom\xxviLang \to
\xxviF\xxviLang$, and likewise for $\xxviF\xxvibv$.  Hence for $\Phi
\in \xxviFom\xxviPom\xxviLang$, $(\xxviF\xxvibw)(\Phi)$ and
$(\xxviF\xxvibv)(\Phi)$ belong to $\xxviFom\xxviLang$, and thus
$\xxvinb(\xxviF\xxvibw)(\Phi)$ and $\xxvinb(\xxviF\xxvibv)(\Phi)$ are
well-formed formulas.  In addition, if $\xxviF$ restricts to finite
sets, every $A \in \xxviPom\xxviFom \xxviLang$ can have at most
finitely many slim redistributions, and every
$\Phi\in\xxviFom\xxviPom\Phi$ can have at most finitely many lifted
members.  So the two axioms ($\xxvinb2$) and ($\xxvinb3$) are at least
well-defined.  What these axioms have in common further is that they
can be seen as \emph{distributive principles}.  This is the clearest
in the case of ($\xxvinb3$), which states that $\xxvinb$ distributes
over certain disjunctions.  In the case of ($\xxvinb2$) the
distributivity is a bit more involved, but basically, the axiom states
that any conjunction of $\xxvinb$s can be replaced with a disjunction
of $\xxvinb$s of conjunctions.

Finally, although the formulation of ($\xxvinb4$) does not use the
actual symbol, it is here that the interaction of the coalgebraic
modality with negation is dealt with.  To see why this is so, observe
that the conclusion of ($\xxvinb4$) implies that $\neg \xxvinb\beta
\xxviissm \xxvibv \{ \xxvinb\alpha \mid \beta \neq \alpha \in
\xxviF\phi \}$.

\subsection*{The general case}

In the case of a general functor, that is, one that does not
necessarily restricts to finite sets, some of the axioms and rules in
Table~\ref{tb:naxfin} above may involve ill-defined syntax.  In
particular, none of the disjunctions on the right-hand side of the
axioms ($\xxvinb2$) and ($\xxvinb3$) will be taken over a \emph{finite} set.
(Algebraically, however, it will often be convenient to think of e.g.\
($\xxvinb2$) as stating that \emph{in case} that the least upper bound
given on the right hand side \emph{exists}, it is greater than the
object denoted by the left hand side.)  The solution is to replace the
axioms ($\xxvinb2$) and ($\xxvinb3$) with \emph{infinitary derivation rules}
(and to do something similar for the conclusion of ($\xxvinb4$)),
according to the following principle.  An axiom of the form $s \xxviissm
\xxvibv_{i\in I} t_{i}$ is replaced with the derivation rule: `from $\{
\vdash t_{i} \xxviissm a \mid i \in I \}$, infer $\vdash s \xxviissm a$'.
Applying this principle to the above axiom system, we obtain the
following derivation system.


\begin{definition}\label{def:nax}
The derivation system $\xxvinax$ is given by the axioms and derivation rules 
of Table~\ref{tb:nax}, on top of a complete set of axioms and rules
for classical propositional logic, and the cut rule.

\begin{table}
\begin{center}
\begin{tabular}{|c|}
%\begin{enumerate}
% \item[($\xxvinb0$)] a complete set of axioms for Boolean algebras
\hline \\
%\item[($\xxvinb1$)]
%\begin{prooftree}
\AXC{$\{ b_1 \xxviissm b_2 \mid (b_1,b_2) \in Z \}$}
\RL{$(\alpha,\beta) \in \xxviol{Z}$}
\LL{($\xxvinb1$) \hspace{0.2cm}}
\UIC{$\xxvinb\alpha \xxviissm \xxvinb\beta$}
\DisplayProof \\ \\
%\end{prooftree}

%\item[($\xxvinb2$)] 
%\begin{prooftree}
\AXC{$\{ \xxvinb (\xxviF\xxvibw)(\Phi) \xxviissm a \mid \Phi\in \xxviSRD(A)\}$}
\LL{($\xxvinb2$) \hspace{0.2cm}}
\UIC{$\xxvibw\{\xxvinb\alpha \mid \alpha\in A\} \xxviissm a$}
\DisplayProof \\ \\
%\end{prooftree}

%%\item[($\xxvinb3$)] 
%\begin{prooftree}
\AXC{$ \{ \xxvinb\alpha \xxviissm a \mid \alpha \xxviol{\in} \Phi \}$}
\LL{($\xxvinb3$) \hspace{0.2cm}}
\UIC{$\xxvinb(\xxviF\xxvibv)(\Phi) \xxviissm a$}
\DisplayProof \\ \\
%\end{prooftree}

%%\item[($\xxvinb4$)]
%\begin{prooftree}
\AXC{$ \{ a \land \xxvinb\alpha' \xxviissm \bot \mid \alpha' \in \xxviFom(\phi), 
   \alpha' \neq \alpha \}$}
\AXC{$\top \xxviissm \xxvibv\phi$}
\LL{($\xxvinb4$) \hspace{0.2cm}}
\BIC{$a \xxviissm \xxvinb\alpha$}
\DisplayProof \\ \\ 
\hline
%\end{prooftree}
% \item[(CUT)]
%\end{enumerate}
\end{tabular}
\end{center}
\caption{Axioms and rules of the system $\xxvinax$}
\label{tb:nax}
\end{table}

A \emph{derivation} is a well-founded tree, labelled with
inequalities, such that the leaves of the tree are labelled with
axioms of $\xxvinax$, whereas with each parent node we may associate a
derivation rule of which the conclusion labels the parent node itself,
and the premisses label its children.  If $\xxviD$ is a derivation of the
inequality $a\xxviissm b$, we write $\xxviD \vdash_{\xxvinax} a \xxviissm b$.  We
write $\vdash_{\xxvinax} a \xxviissm b$ if we want to suppress the actual
derivation and we write $a \equiv b$ if $\vdash_{\xxvinax} a \xxviissm b$ and
$\vdash_{\xxvinax} b \xxviissm a$.
\end{definition}


\subsection*{The main result}

\begin{theorem}
\label{t:main}
Let $\xxviF$ be a standard functor that preserves weak pullbacks. 
Then for any pair $a$ and $b$ of formulas in $\xxviLang$:
\[
\vdash_{\xxvinax} a \xxviissm b
\iff
 a \models b.
\]
\end{theorem}

% Soundness (the direction from left to right of Theorem~\ref{t:main})
% can be proved by a straightforward induction on the complexity of
% derivations.  The key steps in this proof are to show that the rules
% ($\xxvinb1$)--($\xxvinb4$) preserve validity.  Because of space
% limitations, the details are left to the reader.

%%% Local Variables: 
%%% mode: latex
%%% TeX-master: "nax"
%%% End: 


%\input{soundness_final}
\section{Soundness}
\label{s:sound}

Soundness is the direction from left to right of Theorem~\ref{t:main}.
It is proved by induction on the complexity of
derivations.
The key steps are to show that the rules ($\xxvinb1$)--($\xxvinb4$)
preserve validity.

First we consider the rule ($\xxvinb1$).  Suppose that $\xxvibbS
\Vdash a \xxviissm b$ for all pairs $(a,b)$ belonging to some relation
$Z \subseteq \xxviLang\times\xxviLang$ such that $(\alpha,\beta) \in
\xxviol{Z}$.  From the first assumption it follows that ${\Vdash}\circ
Z \subseteq {\Vdash}$, and so, by the properties of relation lifting,
we see that $\xxviol{\Vdash} \circ \xxviol{Z} \subseteq
\xxviol{\Vdash}$.  In order to show that $\xxvibbS \Vdash
\xxvinb\alpha \xxviissm \xxvinb\beta$, take an arbitrary state $s$
such that $\xxvibbS,s \Vdash \xxvinb\alpha$.  Hence, by the truth
definition of $\xxvinb$, we see that $\sigma(s)\;\xxviol{\Vdash}\;
\alpha$, and so from $(\alpha,\beta) \in \xxviol{Z}$ we may infer that
$(\sigma(s),\beta) \in \xxviol{\Vdash} \circ \xxviol{Z} \subseteq
\xxviol{\Vdash}$.  But then, again by the truth definition of
$\xxvinb$, we see that, indeed, $\xxvibbS,s \Vdash \xxvinb\beta$.


For the rule ($\xxvinb2$), fix a set $A \in
\xxviPom\xxviFom\xxviLang$, and some formula $a\in \xxviLang$.
Suppose that $\xxvibbS$ validates all the premisses of the rule, that
is, $\xxvibbS \Vdash \xxvinb (\xxviFom\xxvibw)(\Phi) \xxviissm a$, for
all slim redistributions $\Phi$ of $A$.  In order to prove that
$\xxvibbS$ validates the conclusion of ($\xxvinb2$), assume that
$\xxvibbS,s \Vdash \xxvibw \{ \xxvinb\alpha \mid \alpha \in A \}$.
Clearly it suffices to come up with a slim redistribution $\Phi_{s}$
of $A$ such that $\xxvibbS,s \Vdash \xxvinb (\xxviF\xxvibw)(\Phi)$.

For the definition of $\Phi_{s}$, first associate, with any state $t$
in $\xxvibbS$, the finite set 
\[
\phi(t) := \{ b \in \bigcup_{\alpha\in A} \xxviBase(\alpha) \mid \xxvibbS,t 
\Vdash b \},
\]
and define $\Phi_{s} := (\xxviF\phi)(\sigma(s))$.

First we show that $\xxvibbS,s \Vdash \xxvinb (\xxviF\xxvibw)(\Phi_{s})$.
For that purpose, observe that by definition of $\phi$, the map 
$\xxvibw\circ\,\phi:
S \to \xxviLang$ is such that $\xxviGraph(\xxvibw\circ\,\phi) \subset {\Vdash}$.
From this it follows by the properties of relation lifting that $\xxviGraph 
\big( (\xxviF\xxvibw) \circ (\xxviF\phi) \big) \subseteq \xxviol{\Vdash}$.
In other words, for every element $\tau \in \xxviF S$ we have that $\tau 
\,\xxviol{\Vdash}\, \big( (\xxviF\xxvibw) \circ (\xxviF\phi) \big) (\tau)$.
Taking $\tau = \sigma(s)$, we obtain immediately by the definitions that
$\xxvibbS,s \Vdash \xxvinb (\xxviF\xxvibw)(\Phi_{s})$.

In order to see that $\Phi_{s}$ is a slim redistribution of $A$, observe
%\marginpar{\tiny Please check this! (yv)}
that by definition of $\phi$, $\xxviGraph(\phi)\circ \xxviconv{{\in}}
= {\Vdash}$ when restricted to elements of $\bigcup_{\alpha\in A}
\xxviBase(\alpha)$.  Then by the properties of relation lifting, it
follows that $\xxviGraph(\xxviF\phi \circ \xxviconv{\xxviol{\in}}) =
\xxviol{\Vdash}$.  But then for every $\alpha\in A$ it follows from
$\sigma(s)\,\xxviol{\Vdash}\, \alpha$ that there is some object $\Psi$
such that the pair $(\sigma(s),\Psi)$ belongs to the relation
$\xxviGraph(\xxviF\phi)$, and $\alpha \xxviol{{\in}}\Psi$.  From the
first fact it follows that $\Psi = \Phi_{s}$, and so we find that each
$\alpha\in A$ is a lifted member of $\Phi_{s}$.  In other words,
$\Phi_{s}$ is a redistribution of $A$; but then by its definition it
is slim.

In order to understand the soundness of ($\xxvinb3$), first consider
the statement $\xxvibbS,s \Vdash \xxvibv\phi$.  This statement can be
reformulated equivalently by saying that the pair $(s,\phi)$ belongs
to the relation ${\Vdash} \circ {\in}$, since there is some element
$a\in\phi$ such that $s \Vdash a$.  Alternatively, $s \Vdash
\xxvibv\phi$ iff $(s,\phi) \in {\Vdash} \circ
\xxviconv{\xxviGraph(\xxvibv)}$.  In other words, we find that the
relations ${\Vdash} \circ {\in}$ and ${\Vdash} \circ
\xxviconv{\xxviGraph(\xxvibv)}$ coincide.  From this it follows that
\begin{equation}
\label{eq:soundbv}
\xxviol{{\Vdash} \circ {\in}} \;=\;
\xxviol{{\Vdash}\circ \xxviconv{\xxviGraph(\xxvibv)}}.
\end{equation}

Fix some object $\Phi \in \xxviFom\xxviPom\xxviLang$ and some formula
$a$, and suppose that the coalgebra $\xxvibbS$ validates all the
premisses of ($\xxvinb3$), i.e., $\xxvibbS \Vdash \xxvinb\alpha
\xxviissm a$, for all $\alpha \xxviol{\in} \Phi$.  In order to prove
that $\xxvibbS$ also validates the conclusion of the rule, take an
arbitrary state $s$ such that $\xxvibbS,s \Vdash \xxvinb
(\xxviF\xxvibv)(\Phi)$.  From this it follows that
$(\sigma(s),(\xxviF\xxvibv)(\Phi))$ belongs to the relation
$\xxviol{\Vdash}$, and so $(\sigma(s),\Phi)$ belongs to
$\xxviol{\Vdash} \circ \xxviconv{\xxviGraph(\xxviF\xxvibv)} =
\xxviol{{\Vdash}\circ \xxviconv{\xxviGraph(\xxvibv)}}$.  But then by
(\ref{eq:soundbv}), $(\sigma(s),\Phi)$ belongs to the relation
$\xxviol{{\Vdash} \circ {\in}} = \xxviol{\Vdash} \circ \xxviol{\in}$.
In other words, there is some object $\alpha$ such that $\sigma(s)
\,\xxviol{\Vdash}\,\alpha$ and $\alpha \xxviol{\in} \Phi$.  Clearly
then $\xxvibbS,s \Vdash \xxvinb\alpha$, and so by the assumption we
have $\xxvibbS,s \Vdash a$.

Finally, for the rule $(\xxvinb4$), fix some finite set $\phi$ of
formulas.  It suffices to prove that, for an arbitrary
$\xxviF$-coalgebra $\xxvibbS = (S,\sigma)$, if $\xxvibbS \Vdash \top
\xxviissm \xxvibv\phi$, then for every point $s \in \xxvibbS$ we can
find an $\alpha \in \xxviF(\phi)$ such that $\xxvibbS,s \Vdash
\xxvinb\alpha$.
% (In other words, it suffices to prove the axiom version of the rule.)
From the assumption it follows that every state in $\xxvibbS$
satisfies some formula in $\phi$.  We may formulate this using a
function $f: S \to \phi$ such that $\xxvibbS, s\Vdash f(s)$, for all
$s \in S$, or, equivalently, $\xxviGraph(f) \subseteq {\Vdash}$.  But
then by the properties of relation lifting, we find that
$\xxviGraph(\xxviF f) \subseteq \xxviol{\Vdash}$.  Now consider an
arbitrary state $s$ in $\xxvibbS$, and let $\alpha \in \xxviF(\phi)$
be the element $(\xxviF f)(\sigma(s))$.  Then $(\sigma(s),\alpha) \in
\xxviGraph(\xxviF f) \subseteq \xxviol{\Vdash}$, and so by the truth
definition of $\xxvinb$, we find that $\xxvibbS,s \Vdash
\xxvinb\alpha$.  That is, we have found our $\alpha$.



%\input{completeness_final}
\section{Completeness}
\newcommand{\xxviLift}{\mathfrak{L}}

The completeness proof will use a standard coalgebraic technique,
namely to prove completeness via one-step-completeness. This is
well-known in domain theory (see e.g.  Abramsky~\cite{abramsky:dtlf})
and was introduced to coalgebra by Pattinson~\cite{pattinson:cml-j}.
Subsequently, it was used in for instance
\cite{cirs-patt:concur04-j,kkp:cmcs04,kurz-petr:cmcs08}.

The main idea is the following.  First, we show that Moss' logic
$(\xxviLang,\equiv)$ can be stratified into layers
$(\xxviLang_n,\equiv_n)$, with all layers at $n+1$ arising in a
uniform way from layers at $n$ (Proposition~\ref{prop:equiv-n}).  This
uniform construction can be described by means of a `one-step version'
of the derivation system $\xxvinax$.  Technically, it is described by
a functor $\xxviM:\xxviBA\to\xxviBA$, which constructs
$(\xxviLang_{n+1}/{\equiv_{n+1}})$ as
$\xxviM(\xxviLang_{n}/{\equiv_{n}})$.  Our main technical result
consists of showing that this one-step proof system is complete in a
suitable sense (Proposition~\ref{p:3}).  Then, using a standard
argument, completeness follows from one-step completeness
(Proposition~\ref{prop:coalg-compl}).

\begin{remark}\label{rmk:stone-duality} 
  Continuing from Remark~\ref{rmk:moss-lang-fun-2}, the proof system
  $\xxvinax$ defines a quotient $\xxviLbb\to\xxviM$. Then
  $\delta_X:\xxviM\xxviPal X\to\xxviPal\xxviF X$ is given by factoring
  $\gamma:\xxviLbb\xxviPal\to\xxviPal T$.  $\xxviM$-algebras are the
  Boolean algebras with operator for the Moss modality. The initial
  $\xxviM$-algebra $\xxviLTM$ is Lindenbaum algebra of Moss' logic
  (Proposition~\ref{prop:Lang-M}) and $\delta_X$ is injective
  (Proposition~\ref{p:3}), which then implies completeness.
  % In the Stone duality approach to coalgebraic logic, a logic for
  % $\xxviF$-coalgebras is given by a functor $\xxviM$ on $\xxviBA$
  % and a natural transformation $\delta_X:\xxviM\xxviPal
  % X\to\xxviPal\xxviF X$. The initial $\xxviM$-algebra $\xxviLTM$ can
  % then be considered as an (abstract) logic for $\xxviF$-coalgebras,
  % since for any coalgebra $(S,\sigma)$ we have a `complex
  % $\xxviM$-algebra' $\xxviM\xxviPal S \stackrel{\delta_{S}}{\to}
  % \xxviPal\xxviF S \stackrel{\xxviPal\sigma}{\to} \xxviPal S$ and,
  % by initiality of $\xxviLTM$, a map
  % $\xxvisem{\cdot}:\xxviLTM\to\xxviPal S$ interpreting `formulas' in
  % $\xxviLTM$ as propositions on $S$.  This section shows that if we
  % take for $\xxviM$ the functor associated with the one-step proof
  % system of $\xxvinax$, then $\xxviLTM$ is the Lindenbaum algebra of
  % Moss' logic (Proposition~\ref{prop:Lang-M}) and $\delta_X$ is
  % injective (Proposition~\ref{p:3}), which then implies
  % completeness.
\end{remark}

\subsection{A one-step proof system}

Recall the definition of $\xxviLang_{n}(X)$ from
Definition~\ref{def:moss-lang}.

\begin{definition}
  Let $\xxviLift(X)=\xxviLang_0\{\xxvinb\alpha\mid\alpha\in\xxviFom X\}$.
  In the following we consider $\xxviLift$ to be a functor
  $\xxviSet\to\xxviSet$, which maps $f:X \to Y$ to the function
  $\xxviLift(f):\xxviLift(X)\to\xxviLift(Y)$ that extends the map $\xxvinb \alpha
  \mapsto \xxvinb (\xxviF f)(\alpha)$ via Boolean operations.
\end{definition}

\noindent $\xxviLift$ constructs formulas step-wise:
$\xxviLang_n=\xxviLift^n(\xxviLang_0)$.  Next we show how to construct
$\equiv_n$ step-wise.  In order to smooth our presentation, it is
convenient in the following definition to assume that the generators
are already closed under Boolean operations.

% \begin{definition}
% 	We denote by $\Sigma_\xxviBA$ the category of algebras for 
% 	the Boolean signature and the usual homormophisms betwen them. 
% \end{definition}

\begin{definition}
  Let $A$ be the carrier of an algebra for the Boolean signature and
  let $R \subseteq \xxviLang_0(A) \times \xxviLang_0(A)$ be a set of pairs
  called \emph{relations}.  Using the laws of Boolean algebra, with
  pairs $(a,a') \in R$ as additional axioms $a \xxviissm a'$, one may
  generate a congruence relation $\equiv_{R}$ on the set $A \times A$.
  We say that the pair $(A,R)$ is a \emph{presentation of} the Boolean
  algebra $A/{\equiv_{R}}$ and denote this algebra as $\xxvibap{A}{R}$.  A
  homomorphism $f:A \to B$ of algebras for the Boolean signature is a
  presentation morphism from $(A,\equiv_R)$ to $(B,\equiv_{S})$ if
  $a_1 \equiv_R a_2$ implies $f(a_1) \equiv_{S} f(a_2)$ for all
  $a_1,a_2 \in A$.  The category of presentations and presentation
  morphisms is denoted by $\xxviPres$.
\end{definition}

The notion of a presentation morphism is motivated by the following lemma
which is not difficult to prove.

\begin{lemma}\label{lem:presmor}
Let $f:(A,R) \to (B,S)$ be a presentation morphism. 
Then the function $\xxvip{f}: A/{\equiv_R} \to B/{\equiv_S}$ that maps the
equivalence class of an element $a\in A$ to the equivalence class of $f(a)$
is well-defined.
Moreover $\xxvip{f}$ is a Boolean homomorphism. 
\end{lemma}

\begin{example}
  The {\em standard presentation} of a Boolean algebra $\xxvibbB$ is the
  pair $(U\xxvibbB,\leq)$ where ${\leq}$ is the relation on terms over
  $U\xxvibbB$ induced by the partial order of $\xxvibbB$.
\end{example}

The derivation system $\xxvinax$ is essentially a `one-step' derivation system
since in every rule involving the modality, every occurrence of $\alpha$
is under the scope of exactly one $\xxvinb$.
The following definition makes this precise.

\begin{definition}\label{def:M-one-step}
  Let $(X,R)$ be a presentation.  The \emph{one-step proof system}
  $\xxvinax(X,R)$ is the version of $\xxvinax$ in which all inequalities
  $b_{1} \xxviissm b_{2}$ from $R$ (that is, with $(b_{1},b_{2}) \in R$)
  are additional axioms, and in which \emph{only} elements from $X$
  and $\xxviLift(X)$ may be used.\footnote{In $(\xxvinb1)-(\xxvinb4)$, the $b_i$
    range over $X$, the $a,a'$ over $\xxviLift(X)$, $\alpha,\alpha'\in\xxviFom
    X$, $\phi\subseteq X$.}  We denote the associated notion of
  derivability by $\vdash_{\xxvinax(X,R)}$.  Furthermore, for
  $a_{1},a_{2}$ in $\xxviLift(X)$, we write $a_{1} \xxviissm_{\xxvinax(X,R)}
  a_{2}$ if $\vdash_{\xxvinax(X,R)} a_{1} \xxviissm a_{2}$; and $a_{1}
  \equiv_{\xxvinax(X,R)} a_{2}$ iff $a_{1} \xxviissm_{\xxvinax(X,R)} a_{2}$ and
  $a_{2} \xxviissm_{\xxvinax(X,R)} a_{1}$.  We let $\xxviM(X,R)$ denote the
  Boolean algebra presented by $(\xxviLift(X),{\xxviissm_{\xxvinax(X,R)}})$.

  In case $(X,R)$ is the standard representation of a Boolean algebra
  $\xxvibbA$, we write $\xxvinax(\xxvibbA)$ for the one-step proof
  system based on the standard presentation of $\xxvibbA$, and
  $\vdash_{\xxvinax(\xxvibbA)}$, $\xxviissm_{\xxvinax(\xxvibbA)}$,
  $\equiv_{\xxvinax(A)}$ and $\xxviM(\xxvibbA)$ for the associated
  notions.
\end{definition}

The next subsection shows that $\xxviM(\xxvibbA)$ is not only a
Boolean algebra, but that $\xxviM$ is a functor on the category of
Boolean algebras. This will allow us, in
Section~\ref{sec:stratification}, to recover the Lindenbaum algebra of
$\xxvinax$ as the union of algebras $(\xxviM^n\xxvibbtwo)_{n<\omega}$,
where $\xxviM^{n+1}\xxvibbtwo=\xxviM(\xxviM^n\xxvibbtwo)$.

Sections \ref{sec:M-functor} and \ref{sec:stratification} are rather
technical and are needed mainly\footnote{But Propositions~\ref{p:1}
  (via \ref{prop:liftfun}), \ref{p:2} and \ref{prop:equiv-n} have a
  proof-theoretic interpretation and are of independent interest.} to
deduce completeness from one-step completeness. Our main result, the
one-step completeness (Proposition~\ref{p:3}), does not depend on
Sections \ref{sec:M-functor} and \ref{sec:stratification} and the
reader might wish to go directly to Section~\ref{sec:semantics-of-M}.

\subsection{Technical Interlude: $\xxviM$ as a functor of Boolean algebras}
\label{sec:M-functor}

We want to define a functor $\xxviM: \xxviBA \to \xxviBA$ associated with the
one-step proof system.  As a first step in this direction we note that
$\xxviLift$ can be seen as a functor on the category $\xxviPres$.  The
proof-theoretic content of this is that a morphism $f:(A,R) \to (B,S)$
between presentations extends to a map of derivations between the
one-step proof systems $\xxvinax(A,R)$ and $\xxvinax(B,S)$.

\begin{proposition}
\label{prop:liftfun}
Let $(A,R)$ and $(B,S)$ be presentations in $\xxviPres$ and let $f:(A,R)
\to (B,S)$ be a presentation morphism.  The function $\xxviLift(f)$ is a
presentation morphism from $(\xxviLift(A),\xxviissm_{\xxvinax(A,R)})$ to
$(\xxviLift(B),\xxviissm_{\xxvinax(B,S)})$, i.e., for all $a',a'' \in \xxviLift(A)$,
\begin{equation}\label{equ:hompres}
\vdash_{\xxvinax(A,R)} a' \xxviissm a'' \quad \mbox{implies}  \quad
\vdash_{\xxvinax(B,S)} \xxviLift(f)(a') \xxviissm \xxviLift(f)(a'').
\end{equation}
\end{proposition}
\xxviproofspace
\begin{proof}
  Let $c',c'' \in \xxviLift(A)$ or $c',c'' \in A$.  One shows by
  structural induction on the derivation of $c' \xxviissm_{\xxvinax(A,R)} c''$
  that substituting, in the derivation, each occurrence of $a\in A$
  with $f(a)\in B$ and each occurrence of $a' \in \xxviLift(A)$ with
  $\xxviLift f(a') \in \xxviLift(B)$ yields a proof of
  $\xxviLift(f)(c')\xxviissm_{\xxvinax(B,S)} \xxviLift(f)(c'')$ or $f(c')
  \xxviissm_{\xxvinax(B,S)} f(c'')$, respectively.
%Hence $\xxviquo_\xxvibbA(b')=\xxviquo_\xxvibbA(b'')$
%implies $\xxviquo_\xxvibbB(\xxviLift(f)(b'))=\xxviquo_\xxvibbB(\xxviLift(f)(b''))$.
  Let us give some details of the induction argument (the case in
  which the derivation ends by an application of ($\xxvinb2$) is similar
  to the cases ($\xxvinb 1$), ($\xxvinb 3$) and ($\xxvinb 4$) and has been omitted
  due to
  space limitations).\\

  {\bf Case} $c' \xxviissm_{\xxvinax(A,R)} c''$ is a (Boolean) axiom.  If
  $c',c'' \in \xxviLift(A)$ it is clear from the definition of $\xxviLift$
  that $\xxviLift(f)(c') \xxviissm_{\xxvinax(B,S)} \xxviLift(f)(c'')$ is an axiom as
  well. If $c,c'' \in A$, $f(c') \xxviissm_{\xxvinax(B,S)} f(c'')$ follows
  from the fact that
  $f$ is a morphism in $\xxviPres$. \\

  {\bf Case} $c' \xxviissm_{\xxvinax(A,R)} c''$ is obtained by a derivation
  that ends by the application of some rule $R$ for classical
  propositional logic.  Then $\xxviLift(f)(c') \xxviissm_{\xxvinax(B,S)}
  \xxviLift(f)(c'')$ or $f(c') \xxviissm_{\xxvinax(B,S)} f(c'')$ can be easily
  obtained by applying the inductive hypothesis to the
  premises of $R$. \\

  {\bf Case ($\xxvinb 1$)} Suppose $c'= \xxvinb \alpha$ and $c'' = \xxvinb \beta$
  and suppose that there is a derivation $\xxviD$ of $c' \xxviissm_{\xxvinax(A,R)}
  c''$ such that ($\xxvinb 1$) is the last ruled applied in $\xxviD$, i.e.,
  $\xxviD$ ends with
%    By the definition of $\xxvinax(\xxvibbA)$ this means that $\xxviD$ 
%    of the form
   \begin{prooftree}
	\AXC{$\{ b_1 \xxviissm b_2 \mid (b_1,b_2) \in Z \}$}
	\RL{$(\alpha,\beta) \in \xxviol{Z}$}
	\UIC{$\xxvinb\alpha \xxviissm \xxvinb\beta$}
   \end{prooftree}
   Because $f$ is a Boolean homomorphism we get $f(b_1) \leq f(b_2)$
   for all $(b_1,b_2) \in Z$. Moreover one can easily calculate that
   $(\xxvinb(\xxviF f)(\alpha), \xxvinb (\xxviF f)(\beta)) \in \xxviol{Z'}$ with $Z'
   \xxvicoloneqq \{(f(b_1),f(b_2))\mid (b_1,b_2) \in Z\}$.  Therefore we
   can apply rule ($\xxvinb 1$) again: from the premisses $\{a_1 \xxviissm a_2
   \mid (a_1,a_2) \in Z' \}$ we obtain $\xxviLift(f)(\xxvinb \alpha)=\xxvinb(\xxviF
   f)(\alpha) \xxviissm_{\xxvinax(B,S)} \xxvinb(\xxviF f)(\beta) = \xxviLift(f)(\xxvinb
   \beta)$
   as required.\\


%{\bf Case ($\xxvinb 2$)} Suppose $b' = \xxvibw \{\xxvinb \alpha \mid \alpha \in
%C\}$ for some $C \in \xxviPom \xxviF(A)$ 
%    and suppose we have a derivation in $\xxviD$ of $b' \xxviissm_{
%    \xxvinax(A,R)} b''$ ending with an application of the rule
%    ($\xxvinb 2$), i.e., the last proof step in $\xxviD$ is
%	\begin{prooftree}
%   \AXC{$\{ \xxvinb (\xxviF\xxvibw)(\Phi) \xxviissm b''\mid \Phi\in
%     \xxviSRD(C)\}$} \UIC{$\xxvibw\{\xxvinb\alpha \mid \alpha\in C\}
%     \xxviissm b''$}
%	\end{prooftree}
%	By I.H.\ we have 
%        \begin{equation}\label{equ:ih2}
%           \xxvinb (\xxviF f) \left((\xxviF\xxvibw)(\Phi)\right) \xxviissm \xxviLift(f)(b'')
%           %\quad \mbox{for all} 
%           \quad \Phi \in \xxviSRD(C).
%        \end{equation}
%        Let now $C' \xxvicoloneqq \{ (\xxviF f)(\alpha) \mid \alpha
%        \in C\}$.  If we manage to prove for an arbitrary $\Phi' \in
%        \xxviSRD(C')$ that $\xxvinb (\xxviF\xxvibw)(\Phi')
%        \xxviissm_{\xxvinax(B,S)} \xxviLift(f)(b'')$ we can conclude by
%        ($\xxvinb 2$) that $\xxviLift(f)(\xxvibw\{\xxvinb\alpha \mid
%        \alpha\in C\})=\xxvibw\{\xxvinb\alpha' \mid \alpha'\in C'\}
%        \xxviissm_{\xxvinax(B,S)} \xxviLift(f)(b'')$.

%        Consider now an arbitrary $\Phi' \in \xxviSRD(C')$. By the
%        definition of $C'$ and by the definition of a slim
%        redistribution we have $(\alpha,\Phi') \in
%        \xxviol{\xxviGraph(f)} \circ \xxviol{\in}$ for all $\alpha
%        \in C$.  Furthermore it is easy to check that $\xxviGraph(f)
%        \circ \in \; = \; \in \circ \xxviconv{\xxviGraph(f^{-1})}$
%        and thus we get $(\alpha,\Phi') \in \; \xxviol{\in} \circ
%        \xxviconv{\xxviol{\xxviGraph(f^{-1})}}$.  In other words
%        $\Phi\xxvicoloneqq (\xxviF f^{-1})(\Phi') \in \xxviSRD(C)$.
%        By (\ref{equ:ih2}) we therefore have $\xxvinb \alpha'
%        \xxviissm_{\xxvinax(B,S)} \xxviLift(f)(b'')$ with
%	\begin{eqnarray*} 
%	    %	   \alpha' & = & \xxviLang(f)(\xxviF \xxvibw (\Phi) = (\xxviF f)\left(\xxviF \xxvibw
%	    %(\Phi)\right)	
%	    \stackrel{\mbox{\tiny{$f$ homom.}}}{=}  (\xxviF \xxvibw)(\xxviF \xxviPw f)
%	    %\left(\Phi\right).
%	\end{eqnarray*}
%	Because $\xxviPw f \circ f^{-1} \subseteq \; {\subseteq}$ we
% get
%	\[ \xxviF \xxviPw f (\Phi) = \xxviF \xxviPw f (\xxviF
% (f^{-1})(\Phi')) \; \xxviol{\subseteq} \; \Phi'\] and thus
% $(\alpha',\Phi') \in \; \xxviconv{\xxviol{\xxviGraph(\xxvibw})}
% \circ \xxviol{\subseteq}$.  It follows from propositional logic that
%	\[(\xxviconv{\xxviGraph(\xxvibw)} \circ \subseteq) \quad  \subseteq \quad (\geq
%\circ \xxviconv{\xxviGraph(\xxvibw)}),\] 
%	because this inclusion is
%	just a complicated way of expressing that $X \subseteq X'$ implies
%$\xxvibw X \geq \xxvibw X'$ for all 
%	sets of formulas $X$ and $X'$. 
%	Hence $(\alpha',\Phi') \in \xxviol{\geq} \circ \xxviconv{\xxviol{\xxviGraph(\xxvibw)}}$
%which can be rewritten
%	as $(\xxviF \xxvibw)(\Phi') \xxviol{\xxviissm} \alpha'$. Using ($\xxvinb 1$) one can show
%that the latter
%	yields $\xxvinb (\xxviF \xxvibw)(\Phi') \xxviissm_{\xxvinax(B,S)} \xxvinb \alpha'$. Together
%with 
%	 $\xxvinb \alpha' \xxviissm_{\xxvinax(B,S)} \xxviLift(f)(b'')$ transitivity of $\xxviissm$
%	(the ``cut rule'') gives us that $\xxvinb (\xxviF \xxvibw)(\Phi') \xxviissm_{\xxvinax(B,S)}
%	\xxviLift(f)(b'')$ which proves our claim. \\
  
   {\bf Case ($\xxvinb 3$)} Suppose $b' = \xxvinb (\xxviF \xxvibv)(\Phi)$ and suppose
   that there is a derivation $\xxviD$ of $\xxvinb (\xxviF \xxvibv)(\Phi)
   \xxviissm_{\xxvinax(A,R)} b''$ that ends with the following rule:
	\begin{prooftree}
		\AXC{$ \{ \xxvinb\alpha \xxviissm b'' \mid \alpha \xxviol{\in} \Phi \}$}
		\UIC{$\xxvinb(\xxviF\xxvibv)(\Phi) \xxviissm b''$}
	\end{prooftree}
        By the inductive hypothesis we have 
	\begin{equation}\label{equ:ih3}
	   \xxvinb (\xxviF f)(\alpha) \xxviissm_{\xxvinax(B,S)} \xxviLift(f)(b'') \quad
\mbox{for all} \quad  \alpha \xxviol{\in} \Phi. 
	\end{equation}
	Furthermore we get
	\begin{eqnarray}
	   \xxviLift(f)\left( \xxvinb (\xxviF \xxvibv) (\Phi) \right) &
\stackrel{\mbox{\tiny by Def.}}{=} & \xxvinb (\xxviF f) ((\xxviF \xxvibv) (\Phi)) \\	
	   & \stackrel{\mbox{\tiny $f$ is hom.}}{=} & \xxvinb (\xxviF \xxvibv) \left(
(\xxviF \xxviPw f)(\Phi)\right).\label{equ:applyL}
	\end{eqnarray}
        Moreover the following chain of equivalences holds:
	\begin{eqnarray*}
	   \alpha' \xxviol{\in} (\xxviF \xxviPw f)(\Phi) & \mbox{iff} & (\alpha',\Phi)
\in \; (\xxviol{\in \circ \xxviconv{\xxviGraph(\xxviPw f)}})
	    = \xxviol{\xxviconv{\xxviGraph(f)} \circ \in}  \\
	    & \mbox{iff} & \alpha' = (\xxviF f)(\alpha) \; \mbox{for some} \;
\alpha \xxviol{\in} \Phi  
	\end{eqnarray*}
	The latter equivalence together with (\ref{equ:ih3}) yields 
	$ \xxvinb \alpha' \xxviissm_{\xxvinax(B,S)} \xxviLift(f)(b'') $ for all $\alpha'
\;\xxviol{\in}\; (\xxviF \xxviPw f)(\Phi)$.
By applying rule ($\xxvinb 3$) we obtain 
\[ 
\xxviLift(f)\left( \xxvinb (\xxviF \xxvibv) (\Phi) \right) 
  \stackrel{\mbox{\tiny (\ref{equ:applyL})}}{=}
  \xxvinb (\xxviF \xxvibv)  \left( (\xxviF \xxviPw f)(\Phi)\right) \xxviissm_{\xxvinax(B,S)} \xxviLift(f)(b'')
\]
as required.\\

{\bf Case($\xxvinb 4$)} Consider now the case that $b'' = \xxvinb \alpha$ and
that there is a derivation $\xxviD$ of $b' \le_{\xxvinax(A,R)} \xxvinb \alpha$
that ends as follows:
	\begin{prooftree}
		\AXC{$ \{ b' \land \xxvinb\alpha' \xxviissm \bot \mid \alpha' \in
\xxviFom(\phi), 
   		\alpha' \neq \alpha \}$}
		\AXC{$\top \xxviissm \xxvibv\phi$}
		\BIC{$b' \xxviissm \xxvinb\alpha$}
	\end{prooftree}
        Again we can inductively assume that
	\begin{equation}\label{equ:ih4}
	     \xxviLift(f)(b') \wedge \xxvinb (\xxviF f)(\alpha') \xxviissm \bot \quad
\mbox{for all} \; \alpha' \in \xxviFom (\phi) \; 
		\mbox{s.t.} \; \alpha' \not= \alpha.
	\end{equation}
 	Let us put $\psi \xxvicoloneqq f[\phi]$ . 
	Then by the fact that $f$ is a homomorphism
	of Boolean algebras we get $\top \xxviissm \xxvibv \psi$. It is clear  	
	that for all $\beta' \in \xxviFom(\psi)$ such that
	$\beta' \not= \xxviF f (\alpha)$ there is an $\alpha' \not= \alpha$
	such that $\xxviF f (\alpha') = \beta'$.
%	\[ \{ \beta' \in \xxviFom(\psi) \mid  \xxviF f (\alpha) \not= \beta' \}
%	\subseteq
%	   \{ \beta' \mid \exists \alpha' \in \xxviFom(\phi). \alpha \not=
%\alpha' \; \& \; (\xxviF f)(\alpha')= \beta' \}
%	\]
        Together with (\ref{equ:ih4}) this implies that $\xxviLift(f)(b')
        \wedge \xxvinb\beta' \xxviissm \bot$ for all $\beta' \in
        \xxviFom(\psi)$ such that $(\xxviF f)(\alpha) \not= \beta'$.
        Now we can apply rule ($\xxvinb 4$) which yields
        $\xxviLift(f)(b') \xxviissm \xxvinb (\xxviF f)(\alpha)$.  This is
        what we had to show, because $\xxviLift(f) (\xxvinb \alpha) =
        \xxvinb (\xxviF f) (\alpha)$.
%\end{description}
\end{proof}

As a consequence, one obtains that the value of $\xxviM$ does not depend
on a choice of presentation.

\begin{proposition}
\label{p:Mwelldef}
If $(X,R)$ and $(X',R')$ generate isomorphic Boolean algebras, then 
$\xxviM(X,R) \cong \xxviM(X',R')$.
%$(\xxviLang_{1}(X),{\leq_{(X,R)}})$ and $(\xxviLang_{1}(X'),{\leq_{(X',R')}})$
\end{proposition}
\xxviproofspace
\begin{proof}
   Clearly it suffices to prove that $\xxviM(X,R) \cong \xxviM(B,\leq)$, where
$(B,\leq)$
is the standard presentation of $\xxvibbB := \xxvibap{X}{R}$. 
Recall that $B$ consists of equivalence classes of the set
$\xxviLang_{0}(X)$, 
let $f: \xxviLang_{0}(X) \to B$ be the quotient map,
and let
$m: B \to \xxviLang_0(X)$ be a function such that 
$f \circ m = \xxviid_B$. 
Then it is clear that 
$f$ and $m$ are presentation morphisms between $(X,R)$ 
and $(B,\leq)$. Then by Proposition~\ref{prop:liftfun}
also $\xxviLift(f)$ and $\xxviLift(m)$ are 
presentation morphisms and $\xxvip{\xxviLift(f)}$ is
an surjective $\xxviBA$-homomorphism from
$\xxviM(X,R)$ to $\xxviM(B\leq)$. In order to prove the claim
we show that $\xxvip{\xxviLift(f)}$ is also injective.
Note first that for all $x \in X$
we have $m(f(x)) \equiv_R x$. Using axiom ($\xxvinb 1$)
it can be shown that this implies $\xxviLift(m)(\xxviLift(f)(\xxvinb \alpha))
\equiv \xxvinb \alpha$ for all $\xxvinb \alpha \in \xxviLift(X)$ which can
be inductively extended to $\xxviLift(m)(\xxviLift(f)(a)) \equiv a$
for all $a \in \xxviLift(X)$. But then for all $a,a' \in \xxviLift(A)$
such that $\xxviLift(f)(a) \equiv \xxviLift(f)(a')$ we have
\[
	a \equiv \xxviLift(m)(\xxviLift(f)(a)) \equiv
	\xxviLift(m)(\xxviLift(f)(a')) \equiv a',
\]
which implies that $\xxvip{\xxviLift(f)}$ has to be injective
as required.
\end{proof}

%\subsection{$\xxviM$ as a functor of Boolean algebras}

Recall that, given a Boolean algebra $\xxvibbA$ with standard presentation
$(A,{\leq})$, we let $\xxviM\xxvibbA$ denote the Boolean algebra
$\xxviM(A,{\leq})$
(Definition~\ref{def:M-one-step}).
$\xxviM$ is thus an operation on the class of Boolean algebras. 
We will now see that in fact it is (or can be extended to) a
\emph{functor} on
the category of Boolean algebras.

\begin{definition}  
The quotient map from $\xxviLift(U\xxvibbA)$ to
$\xxviM\xxvibbA = (\xxviLift(U\xxvibbA)/{\equiv_{\xxvinax(\xxvibbA)}})$
will be denoted by $\xxviquo_\xxvibbA$.
\end{definition}

Another consequence of Proposition~\ref{prop:liftfun} is:

\begin{proposition}
\label{p:1}
$\xxviM$ is a functor on the category of Boolean algebras. 
\end{proposition}
\xxviproofspace
\begin{proof}
%$\xxviLbb$ is a functor on $\xxviBA$. 
Let $f:\xxvibbA\to \xxvibbB$ be a Boolean homomorphism and let $A$ and $B$
denote the
underlying sets of $\xxvibbA$ and $\xxvibbB$, respectively.
%For $t \in \xxviM \xxvibbA$ 
%we define $\xxviM f(t)$ to be $\xxviquo_\xxvibbB \circ \xxviLift(f)(b')$ where $b'$
%is any element 
%in $\xxviLift(A)$ such that $\xxviquo_\xxvibbA(b')=t$.
Obviously any homomorphism 
$f:\xxvibbA \to \xxvibbB$ is a presentation morphism from
$(A,\leq)$ to $(B,\leq)$ where $(A,\leq)$ and $(B,\leq)$ denote the standard
presentations of $\xxvibbA$ and $\xxvibbB$. 
So by Proposition~\ref{prop:liftfun}, 
$\xxviLift(f): (\xxviLift(A),\xxviissm_{\xxvinax(A,\leq)}) \to
(\xxviLift(B),\xxviissm_{\xxvinax(B,\leq)})$ is also a presentation morphism and we can
define a Boolean homomorphism $\xxviM f: \xxviM \xxvibbA \to \xxviM \xxvibbB$ by putting
$\xxviM f \xxvicoloneqq \xxvip{\xxviLift(f)}$.
It is easy to see that this $\xxviM$ satisfies the usual functor conditions,
i.e., that $\xxviM (\xxviid) = \xxviid$ and $\xxviM(f \circ g) = \xxviM f \circ \xxviM g$.
% 
%We have to show that this is well-defined, i.e.\ assuming
%$\xxviquo_\xxvibbA(b')=
%\xxviquo_\xxvibbA(b'')$ we want to prove that $\xxviquo_\xxvibbB(\xxviLift(f)(b'))=
%\xxviquo_\xxvibbB(\xxviLift(f)(b''))$.
%It suffices to show that for all $b',b'' \in \xxviLift(B)$
%\begin{equation}\label{equ:hompres}
% 	\vdash_{\xxvinax(\xxvibbA)} b' \xxviissm b'' \quad \mbox{implies}  \quad
%\vdash_{\xxvinax(\xxvibbB)} \xxviLift(f)(b') 
%	\xxviissm \xxviLift(f)(b'').
%\end{equation}
%
%\shtbs{Check $\leq$ vs $\xxviissm$ in remainder of proof}
%
\end{proof}

\begin{remark}\label{rmk:L-M}
  Wrt Remark~\ref{rmk:moss-lang-fun-2}, we note that $\xxviM$ is
  a quotient $\xxviLbb\to\xxviM$.
\end{remark}

It turns out that $\xxviM$ has some nice properties, which will be of use
later on.  In particular, we will show that $\xxviM$ is \emph{finitary}
(or $\omega$-accessible) which means, proof-theoretically, that for any
Boolean algebra $\xxvibbA$, a derivation of $\vdash_{\xxviM(\xxvibbA)}a_{1}\xxviissm
a_{2}$ can be carried out in a \emph{finite} subalgebra of $\xxvibbA$.  A
fairly easy consequence of this is the second useful property given
below, namely, that $\xxviM$ \emph{preserves embeddings}.

\begin{proposition}
\label{p:2}
\label{prop:M-finitary}
$\xxviM$ is a finitary functor that preserves embeddings.
\end{proposition}
\xxviproofspace
\begin{proof}
Fix a Boolean algebra $\xxvibbA$ with carrier set $A := U\xxvibbA$.
Given two elements $a_{1},a_{2} \in \xxviLift(A)$, 
% let $A_{a_{1}a_{2}}$ be 
consider the collection of elements of $A$ that occur as
\emph{subformulas} of 
$a_{1}$ and $a_{2}$.
It follows from our earlier remarks on subformulas that this is a
\emph{finite}
set, which then generates a finite subalgebra $\xxvibbA'$ of $\xxvibbA$.
By definition we have $a_{1},a_{2} \in \xxviLift(A')$.

We claim \vspace{-0.5ex} 
\begin{equation}
\label{eq:2-4}
\vdash_{\xxvinax(\xxvibbA)} a_{1} \xxviissm a_{2} \mbox{ iff }
\vdash_{\xxvinax(\xxvibbA')} a_{1} \xxviissm a_{2}.
\end{equation}
The interesting direction of (\ref{eq:2-4}) is from left to right.
The key observation here is that from the fact that $\xxvibbA'$ is a
finite
subalgebra of $\xxvibbA$, we may infer the existence of a
\emph{surjective}
homomorphism $f: \xxvibbA \to \xxvibbA'$ such that $f(a') = a'$ for all $a'
\in A'$.
(In other words, $\xxvibbA'$ is a \emph{retract} of $\xxvibbA$.)
There are various ways to prove this statement; here we refer to
Sikorski's
theorem that complete Boolean algebras are
injective~\cite{siko:theo48}.
But if $f$ is a homomorphism, by Proposition~\ref{prop:liftfun} it follows
%\marginpar{$\xxviLift$: correct notation?}
from $\vdash_{\xxvinax(\xxvibbA)} a_{1} \xxviissm a_{2}$ that $\vdash_{\xxvinax(\xxvibbA')}
\xxviLift(f)(a_{1}) \xxviissm \xxviLift(f)(a_{2})$.
Since $a_{1},a_{2} \in \xxviLift(A')$ and $f$ restricts to the identity on
$A'$, we may conclude that $\xxviLift(f)(a_{i}) = a_{i}$, 
for both $i = 1,2$.
Thus, indeed, $\vdash_{\xxvinax(\xxvibbA')} a_{1} \xxviissm a_{2}$.
Using the fact that every  $\xxviBA$ is the directed colimit (union) of finite
Boolean algebras, the finitariness of $\xxviM$ follows by a standard argument.

For the second part of the proof, let $e: \xxvibbA \to \xxvibbB$ be an
embedding.
Without loss of generality we will assume that $e$ is actually the
inclusion (that is, $\xxvibbA$ is a subalgebra of $\xxvibbB$).
In order to prove that $\xxviM e: \xxviM\xxvibbA \to \xxviM\xxvibbB$ is also injective, it
suffices to prove the following, for all $a_{1},a_{2} \in A$:
\begin{equation}
\label{eq:2-5}
\vdash_{\xxvinax(\xxvibbB)} a_{1} \xxviissm a_{2} \mbox{ implies } 
\vdash_{\xxvinax(\xxvibbA)} a_{1} \xxviissm a_{2}.
\end{equation}
But the proof of (\ref{eq:2-5}) simply follows from two applications
of 
(\ref{eq:2-4}).
\end{proof}

As a straightforward corollary of Proposition~\ref{p:2}, we obtain the
existence of an \emph{initial} $\xxviM$-algebra.  Furthermore, this
initial algebra is obtained as the union of the initial $\xxviM$-sequence
to be defined now.

\begin{definition}
%Recall that $\xxvibbtwo$ denotes the initial (two-element) Boolean
%algebra.  
We define $j_0: \xxvibbtwo \to \xxviM \xxvibbtwo$ to be the unique
embedding of the two-element Boolean algebra 
$\xxvibbtwo$ into $\xxviM \xxvibbtwo$, and inductively we define
$j_{n+1}:\xxviM^n\mathbbm{2}\to \xxviM(\xxviM^n\mathbbm{2})$ to be $\xxviM j_n$.  Let
$\xxviLTM$ be the colimit of the sequence $(\xxviM^n\mathbbm{2})_{n<\omega}$.
\end{definition}

We take the liberty to consider $\xxviLTM$ as an $\xxviM$-algebra, a Boolean
algebra, or a set, depending on the context.  Since $\xxviM$ is finitary
(Proposition~\ref{prop:M-finitary}), we have the following.

\begin{corollary}
\label{c:LTM}
$\xxviLTM$ is the initial $\xxviM$-algebra.
\end{corollary}

Because $\xxviM$ preserves embeddings, all maps in the initial sequence
$(\xxviM^n\mathbbm{2})_{n<\omega}$ are injective. 
This means that
we can consider the initial $\xxviM$-algebra as a \emph{union} of its
approximants $\xxviM^n\mathbbm{2}$.

\subsection{Technical Interlude: Stratification of the Moss logic}
\label{sec:stratification}

As we will see now, the one-step version of $\xxvinax$ allows for a
layer-wise
construction of the (inter)derivability relation between formulas.

\begin{definition}\label{def:equiv-n}
For each $n$, we define relations $\le_{n}$ and $\equiv_{n}$ on
$\xxviLang_{n}$.  For $n=0$, we simply let $\leq_{0}$ on the set $\xxviLang_0$
of all closed Boolean formulas denote derivability (in Boolean logic).
% Note that $\xxviLang_0/{\leq_0}$ is the two-element Boolean algebra.
% Assuming that $\xxviLang_n/{\le_n}$ is a Boolean algebra, we define
% $\le_{n+1}$ as derivability in the one-step proof system
% $\xxvinax(\xxviLang_n/\le_n)$.
Inductively, we define $\le_{n+1}$ as derivability in the one-step
proof system $\xxvinax(\xxviLang_{n},{\le_{n}})$.  Finally, $a \equiv_n b
\ \Leftrightarrow \ (a\le_n b \ \textrm{ and } a \le_n b)$.
\end{definition}


The following proposition reveals the crucial role of $\xxviM$ in this
stratification.  Its proof proceeds via a straightforward inductive
argument, of which the inductive step is an immediate consequence of
Proposition~\ref{p:Mwelldef}.

\begin{proposition}
For all $n$, $\xxviLang_n/{\equiv_n}\cong \xxviM^{n} \xxvibbtwo$.
\end{proposition}

\begin{definition}\label{def:qn}
For every $n \in \omega$ we let $\xxviquo_n$ be the quotient map from
$\xxviLang_n$ onto $U \xxviM^n \xxvibbtwo \cong \xxviLang_n/{\equiv_n}$.  Furthermore we
let $i_{0}: \xxviLang_0 \to \xxviLang_1$ be the obvious embedding, and
inductively we define $i_{n+1}=\xxviLift (i_{n})$.
\end{definition}

\begin{proposition}
We have $\xxviLang_n = \xxviLift^n(\xxviLang_0)$
and for all $n \in \omega$ the map $i_{n}$ is the inclusion of $\xxviLang_n$ into $\xxviLang_{n+1}$.
\end{proposition}
%\begin{proof}
%	a simple induction
%\end{proof}
Due to lack of space we omit the simple induction argument.
The next proposition establishes a connection between 
the embeddings $j_{n}: \xxviM^n \xxvibbtwo \to \xxviM^{n+1} \xxvibbtwo$ and the inclusions
$i_{n}: \xxviLang_n \to \xxviLang_{n+1}$. The proof is based on the following lemma. (i) is proved using ($\xxvinb1$) and (ii) follows from Proposition~\ref{p:1}.

\begin{lemma}\label{lem:tech_use}
For all $n<\omega$, we have  (i) $\xxviquo_{n+1}  = \xxviquo_{\xxviM^{n} \xxvibbtwo} \circ \xxviLift
(\xxviquo_n)$ and (ii) $U j_{n} \circ \xxviquo_{\xxviM^{n-1} \xxvibbtwo}=\xxviquo_{\xxviM^{n}
\xxvibbtwo} \circ \xxviLift(U j_{n-1})$:\\
%
\centerline{\xymatrix@C=4pt{
\xxviLift(\xxviLang_n) \ar[rd]_{\xxviLift(\xxviquo_n)} \ar[rr]^{\xxviquo_{n+1}} && U \xxviM^{n+1} \xxvibbtwo  \\
& \xxviLift(U\xxviM^n \xxvibbtwo) \ar[ru]_{\xxviquo_{\xxviM^n \xxvibbtwo}} & } 
            \quad\quad 
  \xymatrix@C=14pt{\xxviLift(U \xxviM^{n-1} \xxvibbtwo) \ar[d]_{\xxviquo_{\xxviM^{n-1} \xxvibbtwo}} 
	\ar[rr]^{\xxviLift(U j_{n-1})} & & \xxviLift(U \xxviM^n \xxvibbtwo) 
	\ar[d]^{\xxviquo_{\xxviM^n \xxvibbtwo}} \\
	U \xxviM^n \xxvibbtwo \ar[rr]_{U j_n} & & U \xxviM(\xxviM^n \xxvibbtwo)}} 
\end{lemma}

\begin{proposition}
\label{p:sq1}
For all $n \in \omega$ we have 
$U j_{n} \circ \xxviquo_n = \xxviquo_{n+1} \circ i_{n}$:\\
\centerline{\xymatrix{\xxviLang_{n} \ar[d]_{\xxviquo_n} \ar[r]^{i_n} & \xxviLang_{n+1} 
\ar[d]^{\xxviquo_{n+1}} \\
U \xxviM^n \xxvibbtwo \ar[r]_{U j_n} & U\xxviM^{n+1} \xxvibbtwo}}
\end{proposition}
\xxviproofspace
\begin{proof}
  The case $n = 0$ can be easily checked.  Consider now $n=m+1$ and
  some $\xxvinb \alpha \in \xxviLang_{m+1} =
  \xxviLift^m(\xxviLang_0)$. Then
\begin{eqnarray*}
j_{m+1} (\xxviquo_{m+1} (\xxvinb \alpha )) 
   & \stackrel{\mbox{\tiny (i)}}{=}
   & j_{m+1}(\xxviquo_{\xxviM^{m} \xxvibbtwo}(\xxviLift (\xxviquo_m)(\xxvinb \alpha))
\\ &  \stackrel{\mbox{\tiny (ii)}}{=} 
   & \xxviquo_{\xxviM^{m+1}\xxvibbtwo} (\xxviLift(j_{m})(\xxviLift (\xxviquo_m)(\xxvinb
\alpha)) 
% \\		 & = & \xxviquo_{\xxviM^{m+2}\xxvibbtwo} (
%		 \xxviLift(\xxviLift^{m}(j) \circ \xxviquo_m)
%		 (\xxvinb\alpha)) 
\\ & = & \xxviquo_{\xxviM^{m+1}\xxvibbtwo}	(\xxviLift (j_{m}\circ \xxviquo_m)(\xxvinb
\alpha)) 
\\ & \stackrel{\mbox{\tiny I.H.}}{=}  
   & \xxviquo_{\xxviM^{m+1}\xxvibbtwo}(\xxviLift(\xxviquo_{m+1} \circ
i_{m})(\xxvinb\alpha))	 \\
		 & \stackrel{\mbox{\tiny (i)}}{=} &
		 (\xxviquo_{m+2} \circ i_{m+1}) (\xxvinb \alpha)
	\end{eqnarray*}\vspace{-1cm} \\
	\phantom{blabla} 
\end{proof}

The following proposition
%, of which the proof is sketched in the appendix,
is crucial. 
It shows that if we have
$\vdash_{\xxvinax} a \xxviissm b$ for formulas $a,b$ of depth $n$, there
always is a derivation that does not employ formulas of depth greater
than $n$. This is typical for axiomatisations where each variable is
under the scope of precisely one modal operator. The situation here is
slightly more complicated than usual since our rules allow infinite
sets of premises.


\begin{proposition}\label{prop:equiv-n}
Let $a$ and $b$ be formulas.
Then 
\begin{enumerate}
\item
If $a,b \in \xxviLang_{n}$, and $a \leq_{m} b$ for some $m>n$, then $a
\leq_{n} b$.
\item
If $a,b \in \xxviLang_{n}$, then $\vdash_{\xxvinax} a \xxviissm b$ iff $a \le_{n}
b$.
\end{enumerate}
\end{proposition}
\xxviproofspace
\begin{proof}
  For Part~1 of the proposition, it suffices to confine attention to
  the case where $m=n+1$, which is a consequence of
  Proposition~\ref{p:sq1} and the fact that the injective
  $\xxviBA$-morphism $j_n$ reflects the order.

  Part~2 of the proposition is proved by induction on the complexity
  of derivations in $\xxvinax$.  Here we discuss a sample case of the
  inductive step, namely, where the last applied rule was ($\xxvinb4$):
\begin{prooftree}
  \AXC{$ \{ a \land \xxvinb\alpha' \xxviissm \bot \mid \alpha' \in \xxviFom(\phi),
    \alpha' \neq \alpha \}$} \AXC{$\top \xxviissm \xxvibv\phi$} \BIC{$a \xxviissm
    \xxvinb\alpha$}
\end{prooftree}
Inductively, there is some natural number $k$ such that $\top \leq_{k}
\phi$.  Let $m := \max\big\{ d(b) \mid b \in \phi \cup \{ a \}
\big\}$, where $d(b)$ denotes the depth of the formula $b$.  Then
clearly $m$ is (well-defined as) a finite natural number since $\phi$
is a finite set by assumption.  Then $\phi \subseteq \xxviLang_{m}$, and
so $\xxvinb\beta\in \xxviLang_{m+1}$ for all $\beta\in\xxviFom\phi$.  Since also
$a \in \xxviLang_{m}$, by the first part of the proposition we obtain $a
\land \xxvinb\alpha' \leq_{m+1} \bot$ for all $\alpha' \in \xxviFom(\phi)
\setminus \{ \alpha \}$.  Thus, with $p = \max\{m+1,k\}$, we see that
all premisses of the final rule are $p$-derivable.  But then the
conclusion is $p+1$-derivable, and then, by part 1, $n$-derivable
(where we assumed that $a,\xxvinb\alpha \in \xxviLang_{n}$).
\end{proof}


The next proposition shows that $\xxviLTM$ is the Lindenbaum algebra of
$\xxviLang$ modulo the proof system $\xxvinax$.  To see this, recall from
Proposition~\ref{p:mossinit} that $(\xxviLang,\neg,\wedge,\vee,\xxvinb)$ is
the initial algebra for the functor $\xxviMoss =\xxviId + \xxviId\times\xxviId +
\xxviId\times\xxviId + \xxviFom$.  Since $\xxviLTM$ can also be seen as an algebra of
this kind, it follows from initiality of $\xxviLang$ that there is a
unique quotient $\xxviquo:\xxviLang\to\xxviLTM$.  The next proposition states that
the kernel of $\xxviquo$ is the interderivability relation $\equiv$
according to the proof system $\xxvinax$.

\begin{lemma}\label{lem:qrestriction}
    The quotient maps $\xxviquo_n: \xxviLang_n \to U \xxviM^n \xxvibbtwo$ 
    are the restrictions of $\xxviquo:\xxviLang \to \xxviLTM$. More precisely 
    denote  by $k_n:M^n\xxvibbtwo\to\xxviLTM$ the embeddings of the initial 	
    sequence of $\xxviLTM$ and by $l_n:\xxviLang_n\to\xxviLang$ the inclusions. 
    Then the claim is that $q\circ l_n=k_n\circ\xxviquo_n$. 
\end{lemma}
\xxviproofspace
\begin{proof}    
  To prove this we need to observe that, by definition,
  $\xxviquo:\xxviLang\to\xxviLTM$ is the unique map for which $\xxviquo\circ l_n=f_n$,
  where the $f_n:\xxviLang_n\to\xxviLTM$ are given inductively by
  $f_{n+1}=\mu\circ\xxviquo_\xxviLTM\circ\xxviLift(f_n)$ and $\mu:\xxviM\xxviLTM\to\xxviLTM$
  is the structure map of the $\xxviM$-algebra $\xxviLTM$. We then proceed to
  show by induction that $k_n\circ\xxviquo_n=f_n$. Indeed,
%
$f_{n+1} = %Der
\mu\circ\xxviquo_\xxviLTM\circ\xxviLift(f_n) \stackrel{\tiny\mathrm{ind hyp}}{=} %
\mu\circ\xxviquo_\xxviLTM\circ\xxviLift(k_n)\circ\xxviLift{\xxviquo_n} = %
%\stackrel{\textrm{naturality of $\xxviquo$}}{=} %
\mu\circ\xxviM k_n\circ \xxviquo_{\xxviM^n\xxvibbtwo}\circ\xxviLift{\xxviquo_n} = %initial seq
k_{n+1} \circ \xxviquo_{\xxviM^n\xxvibbtwo}\circ\xxviLift{\xxviquo_n} = %
k_{n+1} \circ \xxviquo_{n+1}$. 
% \footnote{ Wrt the third = : We need naturality of $\xxviquo$, mabye
%   this should be an extra lemma. Wrt the last = : I assumed that it
%   is clear that $\xxviquo_{M^n\xxvibbtwo}\circ\xxviLift{\xxviquo_n} = \xxviquo_{n+1}$
%   but maybe it should be an extra Lemma (ak)}
where the third equation holds by definition of $\xxviM k_n =
\xxvip{\xxviLift(k_n)}$ and the last equation is an instance of
Lemma~\ref{lem:tech_use}(i).
\end{proof}

\begin{proposition}\label{prop:Lang-M}
The kernel of the quotient map $\xxviquo:\xxviLang\to\xxviLTM$ is $\equiv$.
\end{proposition}
\xxviproofspace
\begin{proof}
Recall that $\xxviLTM$ can be seen as the union of the initial sequence 
$(\xxviM^n\mathbbm{2})_{n<\omega}$, that $\xxviLang=\bigcup\xxviLang_n$, and that
the maps $\xxviquo_n:\xxviLang_n\to\xxviM^n\xxvibbtwo$ map formulas to their equivalence
classes. By Lemma~\ref{lem:qrestriction}, each $\xxviquo_n$ is the restriction
of $\xxviquo:\xxviLang\to\xxviLTM$ to $\xxviLang_n\to\xxviM^n\xxvibbtwo$.
Then, $\xxviquo(a) = \xxviquo(b)$ iff there is an $n$ such that $\xxviquo_n(a)=\xxviquo_n(b)$.
This, in turn, is equivalent to $a\equiv_n b$ (by Definition~\ref{def:qn})
and then to $a\equiv b$ by Proposition~\ref{prop:equiv-n}. 
\end{proof}


\subsection{Semantics of $\xxviM$ and Moss algebras}
\label{sec:semantics-of-M}
Let us first summarise the two preceding sections in the following diagram:
\begin{equation*}
\xymatrix{
\xxviLang_0 \ar[r]^{i_0}\ar[d]_{\xxviquo_0} & 
\xxviLang_1\ar[d]_{\xxviquo_1} \ar[r]^{i_1} & 
\xxviLang_2 \ar[d]_{\xxviquo_2} & \ldots & 
\xxviLang \ar[d]_{\xxviquo}\\ 
%
\xxvibbtwo \ar[r]^{j_0} & 
\xxviM\xxvibbtwo \ar[r]^{j_1} & 
\xxviM^2\xxvibbtwo & \ldots & 
\xxviLTM \\ 
}
\end{equation*}
$\xxviLang$ is the union of the $\xxviLang_n$. $\xxviM$ is a functor on $\xxviBA$,
$\xxvibbtwo$ is the initial $\xxviBA$ and $\xxviLTM$ is the colimit of the
sequence $\xxviM^n\xxvibbtwo$. Since $\xxviM$ preserves injections and is
finitary, $\xxviLTM$ is the initial $\xxviM$-algebra and can be considered to
be the union of the $\xxviM^n\xxvibbtwo$. $\xxviquo$ is the quotient of $\xxviLang$
wrt interderivability $\equiv$ and the $\xxviquo_n$ are the restrictions
of $\xxviquo$.

Thus, up to interderivability, we can work with $\xxviM$ and $\xxviLTM$
instead of $\xxviLang$.  In this section, we define the semantics of the
logic directly in terms of $\xxviM$ and show that it agrees with the
previously given one.

The relationship between $\xxviM$ and $\xxviF$ is provided by a natural
transformation $\delta: \xxviM\xxviPal \to \xxviPal\xxviF$.  For the definition of
$\delta$, recall the natural transformation $\rho$ from
Definition~\ref{def:TP-PT}.

\begin{definition}
Given a set $S$, define the map $\tilde{\rho}: \xxviLift(\xxviPc S) \to
\xxviPc\xxviF S$ as follows. 
%\footnote{ the def is now much simpler with the new definition of $
%\xxviLift$. we should mention in the preliminaries that we can consider 
%\xxviFom Y$ as a subset of $\xxviF Y$ because this is used implicitly here 
%(ak) }
For $\alpha \in \xxviFom \xxviPc S$, we let
$\tilde\rho(\xxvinb\alpha)=\rho(\alpha)$ and then extend it freely to
Boolean terms.
\end{definition}

\noindent
The soundness of the one-step proof system is enshrined in the next
proposition.  The proof is essentially the same as that of the
soundness direction in Theorem~\ref{t:main}.

\begin{proposition}
\label{p:3-11}
$a_{1} \!\equiv_{\xxvinax(\xxviPal X)}\! a_{2}$ implies $\tilde{\rho}(a_{1}) =
\tilde{\rho}(a_{2})$, for $a_i \in \xxviLift(\xxviPc X)$.
\end{proposition}

\xxviverberg{\xxviproofspace
\begin{proof}
   One-step soundness can be proven by induction on the derivation of
   $a_{1} \!\equiv_{\xxvinax(\xxviPal X)}\! a_{2}$. The proof is 
   essentially the same as the soundness proof in 
   Section~\ref{s:sound} and is therefore omitted.
\end{proof}
}

By Proposition~\ref{p:3-11}, the following is well-defined.

\begin{definition}
\label{d:3-11}
Given a set $X$, let $\delta_{X}: \xxviLift(\xxviPc X)/{\equiv_{\xxvinax(\xxviPal
X)}} \to
\xxviPc\xxviF X$ be the map given by 
$
\tilde{\rho}_{X} = \delta_{X}\circ \xxviquo_{\xxviPal (X)}.
$
\end{definition}

\begin{proposition}
\label{p:3-12}
The collection of maps given by Definition~\ref{d:3-11} form a natural
transformation $\delta: \xxviM\xxviPal \to \xxviPal\xxviF$.
\end{proposition}

\xxviproofspace
\begin{proof}
  We need to show that for each $X$, $\delta_{X}$ is a (Boolean)
  homomorphism, and that $\delta$ is natural. Both proofs are
  straightforward.
\end{proof}

\begin{remark}
  Continuing from Remark~\ref{rmk:L-M}, $\delta$ is given by factoring
  $\gamma:\xxviLbb\xxviPal\to\xxviPal\xxviF$ (Remark~\ref{rmk:moss-lang-fun}) through
  $\xxviLbb\xxviPal\to\xxviM\xxviPal$.
\end{remark}
%

The natural transformation $\delta$ allows us to associate with a
coalgebra $(S,\sigma)$ its `complex $\xxviM$-algebra'
$\xxviPal\sigma\circ\delta_{S}: \xxviM\xxviPal S \to \xxviPal S$.  Recall that $\xxviLTM$
denotes the initial $\xxviM$-algebra.  For each coalgebra $(S,\sigma)$,
initiality of $\xxviLTM$ gives us a map
\begin{equation}\label{equ:M-sem}
\xxvisem{\cdot}:\xxviLTM\to\xxviPal S
\end{equation} 
interpreting elements of $\xxviLTM$ as propositions on $S$.  Note that
this map is an arrow in the category of Boolean algebras.

The next proposition ensures that the coalgebraic semantics of $\xxviLTM$
(see (\ref{equ:M-sem})) and of $\xxviLang$ (Definition~\ref{def:Lang-sem})
agree.

\begin{proposition}\label{prop:Lang-M-sem}
  Denote by $\xxviquo:\xxviLang\to\xxviLTM$ the quotient map.  Given a coalgebra
  $(S,\sigma)$ and $a\in\xxviLang$, we have $s\Vdash a$ iff
  $s\in\xxvisem{\xxviquo(a)}$.
\end{proposition}

\xxviproofspace
\begin{proof}
  The semantic map $\mu:\xxviLang \to \xxviPw S$ can be written as $\mu= f
  \circ \xxviquo$ for some $f: \xxviLTM \to \xxviPw S$ by putting $f(\xxviquo(a))
  \xxvicoloneqq \mu(a)$.  The function $f$ is well-defined because of
  soundness of our logic: if $\xxviquo(a) = \xxviquo(a')$ then $a \equiv a'$
  by Proposition~\ref{prop:Lang-M} and therefore by soundness we get
  $\mu(a) = \mu(a')$.  Using Lemma~\ref{l:altsem} it is not
  difficult to see that $f$ is in fact an $\xxviM$-algebra morphism from
  the initial $\xxviM$-algebra $\xxviLTM$ to the $\xxviM$-algebra $(\xxviPal S, \xxviPal
  \sigma \circ \delta_S)$.  Therefore by initiality we get
  $f=\xxvisem{\cdot}$ and thus $\mu(a)=\xxvisem{\xxviquo(a)}$.
\end{proof}

%
%\xxviproofspace
%\begin{proof}
%This is a straightforward induction. The case of $\xxvinb$ is handled by
%the following observations. 
%First, $s \Vdash \xxvinb\alpha$ iff $\sigma(s)\in \rho_{S}(\alpha)$, see
%Lemma~\ref{l:altsem}.
%By definition of $\delta$, this is equivalent to $\sigma(s) \in
%\delta_{S}\circ\xxviquo (\xxvinb\alpha)$,
%and hence, to $s \in \sigma^{-1}(\delta_{S}\circ\xxviquo (\xxvinb\alpha))$.
%Finally, by the definition of the algebraic semantics $\xxvisem{\cdot}$,
%the
%latter statement amounts to saying that $s \in \xxvisem{\xxviquo
%(\xxvinb\alpha)}$.
%\end{proof}

\begin{remark}
  We have now finished the functorial presentation of Moss' logic. A
  central role play the \emph{Moss algebras}, that is, the algebras
  for the functor $\xxviM$. In the case of $\xxviF=\xxviPw$, the category of Moss
  algebras is isomorphic to the category of Boolean algebras with
  operators. (\ref{equ:M-sem}) corresponds to the fact that formulas
  are evaluated on a Kripke frame $S$ by the morphism from the
  Lindenbaum BAO $\xxviLTM$ to the complex algebra $\xxviPal S$ of $S$. The
  completeness proof in the next section generalises the well-known
  fact that we have an injection (iso for finite $X$)
  $d_X:\mathbb{K}\xxviPal X \to \xxviPal\xxviPw X$ where $\mathbb{K}$ is the
  functor $\xxviBA\to\xxviBA$ mapping $\xxvibbA$ to the algebra freely generated
  by $\Box a, a\in\xxvibbA$, modulo the equations expressing that $\Box$
  preserves finite meets ($d_X$ is given by $\Box
  a\mapsto\{b\subseteq X \mid b\subseteq a\}$).
\end{remark}



\subsection{One-step completeness}

Completeness of $\xxvinax$ is enshrined in the injectivity of
$\delta_{X}$. To show this we use the following basic fact about
Boolean algebras.

\begin{lemma}\label{lem:bafact}
  Let $\xxvibbA$ and $\xxvibbB$ be Boolean algebras and $f:\xxvibbA \to \xxvibbB$ be a
  homomorphism. Furthermore assume that $\xxvibbA$ is join-generated by
  $\xxvigen \subseteq A$, i.e., assume that for every $a \in A$ we have $a
  = \xxvilatjoin \{b \in \xxvigen \mid b \leq a\}$.  Then $f(b) \not=
  \perp_\xxvibbB$ for all $b \in \xxvigen$ implies that $f$ is injective.
\end{lemma}
\xxviproofspace
\begin{proof}
%    The lemma follows from a standard Boolean reasoning.
  In order to prove the claim note first that for all $a \in A$ we
  clearly have $\perp_\xxvibbA < a$ implies $\perp_\xxvibbB < f(a)$.  Let now
  $a,a'$ be elements of $A$ such that $a \not= a'$.  By our assumption
  we have w.l.o.g. that there is some $b \in \xxvigen$ with $b\leq a$ and
  $b \not \leq a'$. Therefore $\perp_\xxvibbA < \neg a' \wedge b$ which
  implies by our first observation and the fact that $f$ is a
  homomorphism that $\perp_\xxvibbB < \neg f(a') \wedge f(b)$ and thus
  $f(b) \not\leq f(a')$. On the other hand we clearly have $f(b) \leq
  f(a)$ which yields $f(a) \not= f(a')$.  As $a,a'$ where assumed to
  be arbitrary we showed that $f$ is injective.
\end{proof}

%The proof of the following Proposition is given in the Appendix.
\begin{proposition}
\label{p:3}
For every set $X$, the map $\delta_{X}: \xxviM\xxviPal(X) \to \xxviPal\xxviF(X)$ is an
embedding.
\end{proposition}
\xxviproofspace
\begin{proof}
   The basic idea of the proof is to work with the map
$\xxviF\xxvisingleton:\xxviF X\to\xxviF P X$, 
%\marginpar{Notation $\eta$: just to see how it looks (YV)}
where we write $\xxvisingleton_X:X\to P X$ 
for the singleton map
$x\mapsto \{x\}$.
% (we drop the subscript on $\xxvisingleton_X$ if no
%confusion is likely). \footnote{careful: $\xxvisingleton$ is natural only
%  with $\xxviPw$ instead of $P$. I dont think it matters but keep in mind
%  (ak).} 
The crucial property is that 
% \vspace{-0.3cm}
\begin{equation}\label{equ:singleton}
  \rho_X\circ\xxviF(\xxvisingleton_X)=\xxvisingleton_{\xxviF X}.
\end{equation}
The proof of (\ref{equ:singleton}) is based on the observation that 
$\xxviGraph(\xxvisingleton_{X}) \circ \xxviconv{{\in}} = \xxviId_{X}$.
From this it follows by the properties of relation lifting, that 
$\xxviGraph(\xxviF\xxvisingleton_{X}) \circ \xxviconv{\xxviol{\in}} = \xxviId_{\xxviF X}$
and thus $\rho_X(\xxviF \xxvisingleton_X(\alpha)) = \{\beta \mid \beta 
\; \xxviol{\in} \;  
\xxviF \xxvisingleton_X(\alpha) \} = \{\alpha\}$.

%\marginpar{Is this what you had in mind? But how to move from here?}
%% 
 % which is immediate from (i) that $\xxviF$ preserves weak pullbacks and hence the
 % composition of relations and (ii) that `singleton relation composed with
 % membership relation is the identity'. 
 %%

We define the set of ``$\xxviF$-singletons'' by putting
$\xxvigen \xxvicoloneqq \{ \xxvinb \xxviF(\xxvisingleton_X)(\alpha)
\mid \alpha \in \xxviFom X\}$.
In order to prove the proposition it now suffices to show that the Boolean 
algebra $\xxviM\xxviPal(X)$ is \emph{join-generated} by the $\xxviF$-singletons: 
% every element $a \in \xxviM\xxviPal(X)$\footnote{by writing $a$ I mean
%   $\xxviquo_{\xxviPal(X)}(a)$ for some $a \in \xxviLift U \xxviPal(X)$} is the
% (possibly infinite) join of elements of $\xxvigen$:
\begin{equation}
\label{equ:joingen}
\forall a \in \xxviM\xxviPal(X). 
%   \exists A_a \subseteq \xxvigen \; \mbox{s.t.} 
\; a = \xxvilatjoin \{\xxvinb \beta \in \xxvigen \mid \xxvinb \beta \leq a\}.
\end{equation}
(Note that the algebra $\xxviM\xxviPal(X)$ need not be complete.  The intended
reading of (\ref{equ:joingen}) is that every element of $\xxviM\xxviPal(X)$ is
the join of the $\xxviF$-singletons below it, not that every set of
$\xxviF$-singletons has a join.)  To see why the injectivity of
$\delta_{X}$ follows from this, note that by (\ref{equ:singleton}) we
have $\delta_X(\xxvinb \xxviF\xxvisingleton(\alpha)) =
\rho_X(\xxviF\xxvisingleton(\alpha)) = \{ \alpha \} \not = \emptyset =
\bot_{\xxviPal(\xxviF X)}$ for all $\xxvinb \xxviF\xxvisingleton(\alpha)\in \xxvigen$.
%But a straightforward argument shows that if $f: \xxvibbA \to 
%\xxvibbB$ is a
%homomorphism between two Boolean algebras, such that $f(g) \neq \bot_{B}$
%for all $g$ belonging to some join-generating set for $\xxvibbA$, then
Therefore an application of Lemma~\ref{lem:bafact} yields that $f$ is 
injective.

\noindent
Turning to the proof of (\ref{equ:joingen}), we distinguish cases as to the 
nature of the element $a$.

{\em Case 1:} Consider first an element of $\xxviM \xxviPal(X)$ of the form
$\xxvinb \beta$ with $\beta \in \xxviFom U \xxviPal X$. 
It can be easily shown that
    \[ \xxvinb \beta = \xxvinb (\xxviFom \bigvee)(\xxviFom \xxviPom 
    \xxvisingleton_X (\beta))
    = \xxvilatjoin_{\gamma \in 
    \rho_{\xxviPw X}(\xxviFom \xxviPom \xxvisingleton (\beta))} 
    \xxvinb \gamma \]
    where the first equality follows from 
    the fact that $\bigvee {\circ} \xxviPom \xxvisingleton_X = \xxviid_{\xxviPom X}$
%     ($\xxvinb 1$)
%     and the fact that
%     $\bigvee {\circ} \xxviPw \xxvisingleton_X \subseteq 
%     \equiv$
    and the second equality is an instance of 
    axiom ($\xxvinb 3$). 
    Furthermore one calculates (a detailed proof can be found 
    in~\cite{jaco04:trace}) that
%    \footnote{To be expanded. 
%    Result can be found in Bart's paper.}  
    \[ \rho_{\xxviPw X}(\xxviFom \xxviPom \xxvisingleton (\beta)) = \xxviFom\xxvisingleton [ 
    \rho_X (\beta) ] = \{ \xxviFom\xxvisingleton(\alpha) \mid 
    \alpha \in \rho_X(\beta)  
    \}. \]
    By combining the latter equality with the
	preceding ones we obtain
	\[ \xxvinb \beta = \xxvilatjoin_{\alpha \in \rho_X(\beta)} 
	\xxvinb (\xxviF \xxvisingleton)(\alpha) ,\]
	which shows that $\xxvinb \beta$ is the join of elements of $\xxvigen$. 

{\em Case 2:} Consider now an element of $\xxviM \xxviPal(X)$ of the form
$\neg \xxvinb \beta$. Let
$\xxvibbB$ be the subalgebra of $\xxviPal(X)$ 
generated by $\xxviBase(\beta) \subseteq_\xxviom U \xxviPal(X)$.
We write $B$ for the carrier of $\xxvibbB$. 
As $\xxviBase(\beta)$ is finite, the Boolean algebra 
$\xxvibbB$ is finite as well. 
Let $\phi \in \xxviPom (X)$ be the
(finite) set of atoms of $\xxvibbB$.
Then clearly $\bigvee \phi = \top$, while $a \wedge a' = \bot$ for any two
distinct $a,a' \in \phi$.
Furthermore $\xxvibv$ induces an isomorphism from
$\xxviPom \phi$ to $B$ that lifts to
an isomorphism $T \xxvibv$ between $\xxviFom \xxviPom \phi$
and $\xxviFom B$. As $\xxviBase(\beta)\subseteq B$ we have   
$\beta \in \xxviFom B$ and thus there exists some
$\Phi_\beta \in \xxviFom \xxviPom \phi$ such that
$(\xxviF \xxvibv)(\Phi_\beta) = \beta$. 
Now axiom ($\xxvinb 3$) entails that
\begin{equation}\label{e:item1}
   \xxvinb \beta = \xxvibv \{ \xxvinb \gamma \mid \gamma \in \xxviFom\phi,
   \gamma \xxviol{\in} \Phi_\beta \}.
\end{equation}
Our claim is now that 
\begin{eqnarray}
   \top & \xxviissm & \xxvibv \{ \xxvinb\gamma \mid \gamma \in \xxviFom\phi\} \qquad \mbox{and} \label{e:item2} \\
   \xxvinb \gamma \wedge \xxvinb \gamma' & \xxviissm & \bot \qquad \mbox{for all} \; \gamma,\gamma' \in \xxviFom \phi
   \quad \mbox{s.t.} \; \gamma \not= \gamma'. \label{e:item3}
\end{eqnarray}
Items (\ref{e:item1}), (\ref{e:item2}) and (\ref{e:item3}) 
together entail that 
\begin{equation}\label{equ:equalityholds}
\neg \xxvinb \beta = \xxvibv \{ \xxvinb \gamma \mid 
\gamma \in \xxviFom \phi \mbox{\ and not} \; \gamma \xxviol{\in} \Phi_\beta \}.
\end{equation}
From Case~1 we know that all elements $\xxvinb \gamma$ that occur on the righthand side 
of equation (\ref{equ:equalityholds}) can be written as joins of
elements of $\xxvigen$ and therefore the same applies to $\neg\xxvinb\beta$.

We now turn to the proof of (\ref{e:item2}) and (\ref{e:item3}). 
First note that because $\bigvee \phi = \top$ an application
of ($\xxvinb 4$) shows that $\top \xxviissm \xxvilatjoin 
\{ \xxvinb\gamma \in \xxviFom \phi \mid \gamma \in \xxviFom \phi\} $
which proves (\ref{e:item2}).
% 
% 
% Because $\beta = (\xxviF\xxvisingleton)(\alpha)$ we have
% $\xxviBase(\beta)\subseteq \xxvisingleton[X]$ and thus
% $\xxviBase(\beta)\subseteq \phi$. 
% Therefore $\beta \in \xxviFom \phi$ and an application of ($\xxvinb 4$)
% %\footnote{At the moment I assume the 
% %``stricter'' version of this axiom.} 
% yields now $\neg \xxvinb \beta \xxviissm
% \xxvilatjoin \{ \xxvinb\beta' \in \xxviFom U \xxviPal(X) \mid \beta' \in \xxviFom \phi
% \; \mbox{and} \; \beta' \not= \beta \}$.
% We are going to show that in fact 
% \begin{equation}
% \neg \xxvinb \beta =
% \xxvilatjoin \{ \xxvinb\beta' \in \xxviFom U \xxviPal(X) \mid \beta' \in \xxviFom \phi
% \; \mbox{and} \; \beta' \not= \beta \}
% \end{equation}
% by proving that for all
% $\beta,\beta' \in \xxviFom \phi$, $\xxvinb \beta \wedge \xxvinb \beta' 
% \xxviissm \bot$ if $\beta \not= \beta'$. 
For the proof of (\ref{e:item3}) consider $\gamma,\gamma' \in \xxviFom \phi$ 
with $\gamma \not= \gamma'$.  
Let $\Phi \in \xxviSRD(C)$ be a slim redistribution of 
$C \xxvicoloneqq \{\gamma,\gamma'\}$. We want to show that
$\xxvinb(\xxviF \xxvibw)(\Phi) \xxviissm \bot$. As $\Phi$ is an arbitrary redistribution of $C$ this will imply by ($\xxvinb 2$) that
$\xxvinb \gamma \wedge \xxvinb \gamma' \xxviissm \bot$ as required.

By assumption we have $(\gamma,\Phi), (\gamma',\Phi) \in \xxviol{\in}$.
This shows that $\emptyset \not\in\xxviBase(\Phi)$. 
Suppose now for a contradiction that $B_\Phi \xxvicoloneqq 
\xxviBase(\Phi) \subseteq \xxviPw \phi$ contains only singleton sets
and put $\in' \xxvicoloneqq {\in\xxvirst{\phi \times B_\Phi}}$.
Then $\in' \circ \xxviconv{\in'} \subseteq \xxviId_{\phi}$.
As a consequence we get $(\gamma,\gamma') \in \xxviol{\in' \circ \xxviconv{(\in')}}
\subseteq \xxviId_{\xxviF\phi}$ which means $\gamma = \gamma'$ --- a contradiction.
Hence we can assume that $B_\Phi$ contains at least one set 
$\psi^* \subseteq \phi$ such that $|\psi^*|>1$.

In order to prove that $\xxvinb(\xxviF \xxvibw)(\Phi) \xxviissm \bot$, define a function
$d: \xxviPw \phi \to  \xxviPw \phi$ by letting 
\[	 d(\psi)  \xxvicoloneqq  \left\{ 
	    \begin{array}{lcl}
	    		\emptyset & \mbox{if} & |\psi| \geq 1 \\ 
			\psi & \mbox{if} & |\psi| = 1 \\
			\phi & \mbox{if} & \psi = \emptyset
		\end{array} \right.
\]
It follows from our assumptions on the set $\phi$ that $\vdash \xxvibw\psi \xxviissm
\xxvibv d(\psi)$, for all $\psi \in \xxviPw\phi$.
Then an application of axiom ($\xxvinb 1$) shows that 
$\xxvinb (\xxviF \xxvibw)(\Phi) \xxviissm \xxvinb (\xxviF \xxvibv)(\xxviF d (\Phi))$.
Because $\xxviBase(\xxviF d(\Phi)) = d[B_\Phi]$ and 
$d(\psi^*) = \emptyset$ we obtain $\emptyset \in \xxviBase(\xxviF d(\Phi))$.
It is now a matter of routine checking that
$A \xxvicoloneqq \{ \alpha \in \xxviFom U \xxviPal(X) \mid \alpha \xxviol{\in} 
\xxviF d(\Phi)\} = \emptyset$. By axiom $(\xxvinb 3)$ we have
$\xxvinb (\xxviF \xxvibv)(\xxviF d (\Phi)) \xxviissm \xxvilatjoin_{\alpha \in A} 
\xxvinb \alpha = \bot$, and thus $\xxvinb (\xxviF \xxvibw)(\Phi) \xxviissm \bot$
which finishes the argument for proving (\ref{e:item3}) and hence
also of (\ref{equ:equalityholds}).

% \[ 
% \neg \xxvinb \beta = 
% \xxvilatjoin \{ \xxvinb (\xxviF \xxvisingleton)(\alpha) \in \xxvigen \mid 
%  \xxvinb (\xxviF \xxvisingleton)(\alpha) \leq \neg \xxvinb \beta \} .
% \]
\medskip

{\em Case 3:} Consider an element of $\xxviM \xxviPal(X)$ of the form
$\xxvibw_{i \in I} \xxvinb \beta_i$ for some finite set $A=
\{ \xxvinb \beta_i \in
\xxviFom U \xxviPal(X) \mid i\in I\}$. Then by axiom ($\xxvinb 2$)
we have
\begin{eqnarray*}
 	 \xxvibw_{i \in I} \xxvinb \beta_i & = & \xxvilatjoin_{\Phi \in \xxviSRD(A)}
	\xxvinb (\xxviFom \xxvibw)(\Phi)\\
	& \stackrel{\mbox{\tiny Case 1}}{=} & \xxvilatjoin_{\Phi \in \xxviSRD(A)} \{ \xxvinb (\xxviF \xxvisingleton)(\alpha) \in \xxvigen \mid  
	\xxvinb (\xxviF \xxvisingleton)(\alpha) \leq \xxvinb (\xxviFom \xxvibw)(\Phi)\}.
\end{eqnarray*}

Finally, the general case (that is, for an arbitrary element of $\xxviM 
\xxviPal (X)$)
can be obtained 
%as follows: Let $a$ be an arbitrary element of
%$\xxviM \xxviPal (X)$, ie., $a$ is a Boolean combination of formulas
%%
%
%
from the cases above using standard Boolean reasoning.
% When dealing with formulas of the form $\neg\xxvinb \beta\in
% \xxviM \xxviPal (X)$ one uses that
% \[
% 	\neg\xxvinb\beta \stackrel{\mbox{\tiny Case~1}}{=} 
% 	\neg (\xxvibv \xxvinb(\xxviF\xxvisingleton)(\alpha)) = 
% 	\xxvibw \neg \xxvinb(\xxviF\xxvisingleton)(\alpha) \]
% and then Case~2 takes care of the negated $\xxvinb$-formulas
% on the righthand side.
\end{proof}	


As a corollary of the proof of Proposition~\ref{p:3} we obtain the
following one-step normal form theorem.

\begin{corollary}
For $a\in\xxviLift\xxviPc X$ we have 
\[ a=\xxvilatjoin\{\xxvinb(\xxviF \xxvisingleton)(\alpha)
\mid\alpha\in\delta_X(a) \cap \xxviFom X\}.\]
\end{corollary}

\noindent In case that $\xxviF$ preserves finite sets, the join is finite
for finite sets $X$ and can be expressed in the language.  Induction
along the sequence of the $\xxviLang_n$ then yields a normal form theorem
for $\xxviLang$.
% For the case $\xxviF=\xxviPw$ this yields the normal forms of
% Fine~\cite{fine:normal-forms}.  ((and the disjunctive formulas of
% (what is the righ reference?  Janin/Walukiewicz ??))).

\subsection{Completeness}

The following proposition, going back to \cite{kkp:cmcs04}, is a
standard result in coalgebra based on $\delta$ being injective and
$\xxviM$ being finitary and preserving injective maps.

\begin{proposition}\label{prop:coalg-compl}
  Suppose $a\not\le b$ in the initial $\xxviM$-algebra. Then there is
  a $\xxviF$-coalgebra $(S,\sigma)$ and $s\in S$ such that $s\Vdash a$
  and not $s\Vdash b$. 
\end{proposition}
\xxviproofspace
\begin{proof} (Sketch) \hspace{0.1cm} To explain the idea of the proof
  assume first that a final $\xxviF$-coalgebra $\zeta:Z\to \xxviF Z$ exists.
  Then we would prove that the unique $\xxviM$-algebra morphism
  $\xxvisem{\cdot}$ from the initial $\xxviM$-algebra to $\xxviM\xxviPal Z
  \stackrel{\delta}{\longrightarrow}
  \xxviF\xxviPal\stackrel{\zeta^{-1}}{\longrightarrow}\xxviPal Z$ is injective.
  Indeed, $a\not\le b$ then implies $\xxvisem{a}\not\subseteq\xxvisem{b}$, ie
  there is $z\in Z$ such that $(Z,\zeta),z\Vdash a$ and
  $(Z,\zeta),z\not\Vdash b$.

  Since the assumption of the existence of a final coalgebra excludes
  important examples such as Kripke frames or models, we replace the
  final coalgebra by the corresponding final sequence
  $(\xxviF^n1)_{n<\omega}$, which is defined as follows.  We denote by
  $1=\xxviF^01$ the final object in $\xxviSet$.  The map $p_0:\xxviF{1}\to{1}$ is
  given by finality and $p_{n+1}:\xxviF(\xxviF^n{1})\to \xxviF^n{1}$ is defined to
  be $\xxviF p_n$.  It is easy to see that each $p_{n}$ is surjective.  We
  think of the $\xxviF^n{1}$ as approximating the final coalgebra.
  (Indeed, if we let run the final sequence through all ordinals, we
  obtain the final coalgebra as a limit if it exists, 
  see~\cite{adam-koub:gfp}.)  In the same way as any coalgebra
  $\xi:X\to \xxviF X$ has a unique arrow into the final coalgebra, there
  are canonical `$n$-step behavior maps', that is, arrows $\xi_n:X\to
  \xxviF^n{1}$ to the approximants of the final coalgebra: $\xi_0:X\to 1$
  is given by finality and $\xi_{n+1}=\xxviF(\xi_n)\circ\xi$.

  Recall that we may consider $\xxviLTM$, the initial $\xxviM$-algebra, as a
  union of its approximants $\xxviM^n\mathbbm{2}$.  Elements of
  $\xxviM^n\mathbbm{2}$ correspond to formulas of depth $n$ and we define
  their semantics wrt the final sequence of $\xxviF$ as a $\xxviBA$-morphism
  $\xxvisem{-}_n:\xxviM^n\mathbbm{2}\to \xxviPal\xxviF^n 1$ as follows.
\begin{equation}
  \xymatrix{
    \xxviPal 1\ar[r]^{\xxviPal p_0} & \ldots
    & \xxviPal\xxviF^n 1\ar[r]^{\xxviPal p_n}
    & \xxviPal\xxviF^{n+1} 1 & \ldots\\
    \mathbbm{2}\ar[u]_{\xxvisem{-}_0}\ar[r]_{j_0} & \ldots
    & \xxviM^n\mathbbm{2}\ar[u]_{\xxvisem{-}_n}\ar[r]_{j_n}
    & \xxviM^{n+1}\mathbbm{2}\ar[u]_{\xxvisem{-}_{n+1}} & \ldots
  }
\end{equation}
$\xxvisem{-}_0$ is given by initiality (and is actually the identity).
For the definition of $\xxvisem{-}_{n+1}$, recall that $\delta_{\xxviF^{n}1}:
\xxviM\xxviPal \xxviF^{n}1 \to \xxviPal \xxviF^{n+1} 1$, and assume inductively that
$\xxvisem{-}_{n}: \xxviM^{n}\xxvibbtwo \to \xxviPal \xxviF^{n}1$ has been defined, so that
$\xxviM (\xxvisem{-}_{n}) : \xxviM^{n+1}\xxvibbtwo \to \xxviM\xxviPal \xxviF^{n}1$.  Composing
these two maps, we obtain $\xxvisem{-}_{n+1} := \delta_{\xxviF^n 1}
\circ\xxviM(\xxvisem{-}_n)$.

Observe that the semantics of a formula is independent of the
particular approximant we choose (all squares in the diagram commute).
Moreover, given a coalgebra $\xi:X\to \xxviF X$ and $a\in\xxviM^n\mathbbm{2}$,
the semantics via the initial $\xxviM$-algebra and the semantics via the
final sequence coincide: $\xxvisem{a}_{(X,\xi)}=\xi_n^{-1}(\xxvisem{a}_n)$.
Since $\delta$ is injective (Proposition~\ref{p:3}) and $\xxviM$ preserves
embeddings (Proposition~\ref{p:2}), a straightforward inductive proof
shows that all $\xxvisem{-}_n$, $n\in\omega$, are injective.

To show the claim now, suppose $a\not\le b$ in the initial
$\xxviM$-algebra.  We find an approximant $\xxviM^n\mathbbm{2}$, in which
$a\not\le_n b$.  Choosing a half-inverse $h$ of $p_0$, we let $\xi:
\xxviF^{n}1 \to \xxviF\xxviF^{n}1$ be $\xxviF^n(h)$. $\xi$ provides $\xxviF^n{1}$ with
$\xxviF$-coalgebra structure.  Now injectivity of $\xxvisem{-}_n$ shows that
$(\xxviF^n{1},\xi)$ provides a counter-example for $a\le b$.
\end{proof}

\noindent The proof of Theorem~\ref{t:main} is now a corollary.
Reasoning by contraposition, take formulas $a,b \in \xxviLang$ such that
$\not\vdash_{\xxvinax} a \leq b$. By Proposition~\ref{prop:Lang-M},
$a\not\le b$ in $\xxviLTM$. Now, completeness follows from
Propositions~\ref{prop:coalg-compl} and \ref{prop:Lang-M-sem}.



%%% Local Variables: 
%%% mode: latex
%%% TeX-master: "nax"
%%% End: 


%%
%%
% \bibliographystyle{abbrv}%same as plain, but first names abbreviated
% \bibliography{coalg-short,nabla}
%%

\begin{thebibliography}{10}

\bibitem{abramsky:dtlf}
S.~Abramsky.
\newblock Domain theory in logical form.
\newblock {\em Ann.\ Pure Appl.\ Logic}, 51, 1991.

\bibitem{aczel:nwfs}
P.~Aczel.
\newblock {\em Non-Well-Founded Sets}.
\newblock CSLI, Stanford, 1988.

\bibitem{adam-koub:gfp}
J.~Ad{\'a}mek and V.~Koubek.
\newblock On the greatest fixed point of a set functor.
\newblock {\em Theoret.\ Comput.\ Sci.}, 150, 1995.

\bibitem{adam-trnk:automata}
J.~Ad{\'a}mek and V.~Trnkov{\'a}.
\newblock {\em Automata and Algebras in Categories}.
\newblock Kluwer, 1990.

\bibitem{bilk:proo08}
M.~B\'{\i}lkov\'a, A.~Palmigiano, and Y.~Venema.
\newblock Proof systems for the coalgebraic cover modality, 2008.
\newblock Same volume.

\bibitem{bons-kurz:fossacs05}
M.~Bonsangue and A.~Kurz.
\newblock Duality for logics of transition systems.
\newblock In {\em FoSSaCS'05}.

\bibitem{cirs-patt:concur04-j}
C.~C{\^\i}rstea and D.~Pattinson.
\newblock Modular proof systems for coalgebraic logics.
\newblock {\em Theoret.\ Comp.\ Sci.}, 338, 2007.

\bibitem{dago:moda06}
G.~D'Agostino and G.~Lenzi.
\newblock On modal $\mu$-calculus with explicit interpolants.
\newblock {\em Journal of applied logic}, 338, 2006.

\bibitem{fine:norm75}
K.~Fine.
\newblock Normal forms in modal logic.
\newblock {\em Notre Dame Journal of Formal Logic}, 16, 1975.

\bibitem{jaco04:trace}
B.~Jacobs.
\newblock Trace {S}emantics for {C}oalgebras.
\newblock In {\em CMCS'04}.

\bibitem{jacobs:many-sorted}
B.~Jacobs.
\newblock Many-sorted coalgebraic modal logic: a model-theoretic study.
\newblock {\em Theor.\ Inform.\ Appl.}, 35, 2001.

\bibitem{jani:auto95}
D.~Janin and I.~Walukiewicz.
\newblock Automata for the modal $\mu$-calculus and related results.
\newblock In {\em Proc.~MFCS'95}.

\bibitem{kkp:cmcs04}
C.~Kupke, A.~Kurz, and D.~Pattinson.
\newblock Algebraic semantics for coalgebraic logics.
\newblock In {\em CMCS'04}.

\bibitem{kupk:coal08}
C.~Kupke and Y.~Venema.
\newblock Coalgebraic automata theory: basic results.
\newblock {\em Logical Methods in Computer Science}, to appear.

\bibitem{kurz:cmcs98-j}
A.~Kurz.
\newblock Specifying coalgebras with modal logic.
\newblock {\em Theoret.\ Comput.\ Sci.}, 260, 2001.

\bibitem{kurz:sigact06}
A.~Kurz.
\newblock Coalgebras and their logics.
\newblock {\em SIGACT News}, 37, 2006.

\bibitem{kurz-petr:cmcs08}
A.~Kurz and D.~Petri{\c s}an.
\newblock Functorial coalgebraic logic: The case of many-sorted varieties.
\newblock In {\em CMCS'08}.

\bibitem{moss:cl}
L.~Moss.
\newblock Coalgebraic logic.
\newblock {\em Ann.\ Pure Appl.\ Logic}, 96, 1999.

\bibitem{palm:nabl07}
A.~Palmigiano and Y.~Venema.
\newblock Nabla algebras and {C}hu spaces.
\newblock In {\em CALCO'07}.

\bibitem{pattinson:cml-j}
D.~Pattinson.
\newblock Coalgebraic modal logic: Soundness, completeness and decidability of
  local consequence.
\newblock {\em Theoret.\ Comput.\ Sci.}, 309, 2003.

\bibitem{roessiger:ml98-j}
M.~R{\"o}{\ss}iger.
\newblock From modal logic to terminal coalgebras.
\newblock {\em Theoret.\ Comput.\ Sci.}, 260, 2001.

\bibitem{rutten:uc-j}
J.~Rutten.
\newblock Universal coalgebra: {A} theory of systems.
\newblock {\em Theoret.\ Comput.\ Sci.}, 249, 2000.

\bibitem{sant:comp07}
L.~Santocanale and Y.~Venema.
\newblock Completeness for flat modal fixpoint logics (extended abstract).
\newblock In {\em LPAR'07}.

\bibitem{schroeder:fossacs05}
L.~Schr\"oder.
\newblock Expressivity of {C}oalgebraic {M}odal {L}ogic: {T}he {L}imits and
  {B}eyond.
\newblock In {\em FoSSaCS'05}.

\bibitem{siko:theo48}
R.~Sikorski.
\newblock A theorem on extensions of homomorphisms.
\newblock {\em Annals of Polish Mathematical Society}, 21, 1948.

\bibitem{vene:alge06}
Y.~Venema.
\newblock Algebras and coalgebras.
\newblock In P.~Blackburn, J.~van Benthem, and F.~Wolter, editors, {\em
  Handbook of Modal Logic}, pages 331--426. Elsevier, 2006.

\bibitem{venema:coalg-aut}
Y.~Venema.
\newblock Automata and fixed point logics: a coalgebraic perspective.
\newblock {\em Inf.\ Comput.}, 204, 2006.

\end{thebibliography}


%% or
%% for the proceedings please include the bbl here
%%
% \begin{thebibliography}{99}
% %to be in "plain style"
% \end{thebibliography}

\begin{description}
%\bigskip\noindent
\item[\rm Clemens Kupke:]
Centrum voor Wiskunde en Informatica (CWI),
Kruislaan 413,
1098 SJ Amsterdam.
\emph{E-mail:} kupke@cwi.nl
\item[\rm Alexander Kurz:]
Department of Computer Science,
University of Leicester, UK.
\emph{E-mail:} kurz@mcs.le.ac.uk 
\item[\rm Yde Venema:]
Institute for Logic, Language and Computation,
Universiteit van Amsterdam,
Plantage Muidergracht 24,
1018 TV Amsterdam.
\emph{E-mail:} yde@science.uva.nl
\end{description}

\end{document}

